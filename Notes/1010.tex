\documentclass[12pt]{report} 
\usepackage[utf8]{inputenc}
\usepackage{geometry}
\geometry{letterpaper}
\usepackage{graphicx} 
\usepackage{parskip}
\usepackage{booktabs}
\usepackage{array} 
\usepackage{paralist} 
\usepackage{verbatim}
\usepackage{subfig}
\usepackage{fancyhdr}
\usepackage{sectsty}
\usepackage[shortlabels]{enumitem}

\pagestyle{fancy}
\renewcommand{\headrulewidth}{0pt} 
\lhead{}\chead{}\rhead{}
\lfoot{}\cfoot{\thepage}\rfoot{}


%%% ToC (table of contents) APPEARANCE
\usepackage[nottoc,notlof,notlot]{tocbibind} 
\usepackage[titles,subfigure]{tocloft}
\renewcommand{\cftsecfont}{\rmfamily\mdseries\upshape}
\renewcommand{\cftsecpagefont}{\rmfamily\mdseries\upshape} %

\usepackage{amsmath}
\usepackage{amssymb}
\usepackage{mathtools}
\usepackage{empheq}
\usepackage{xcolor}

\usepackage{tikz}
\usepackage{tikz-cd}
\usepackage{pgfplots}
\pgfplotsset{compat=1.18}

\renewcommand{\hat}[1]{\widehat{#1}}
\renewcommand{\bar}[1]{\overline{#1}}

\newcommand{\F}[1]{\mathcal{F}(#1)}
\renewcommand{\P}{\mathbb{P}}
\newcommand{\R}{\mathbb{R}}
\newcommand{\E}{\mathbb{E}}
\newcommand{\Z}{\mathbb{Z}}
\newcommand{\Q}{\mathbb{Q}}
\newcommand{\N}{\mathbb{N}}
\newcommand{\C}{\mathbb{C}}
\newcommand{\mfX}{\mathfrak{X}}

\newcommand{\ind}{\mathbbm{1}}
\newcommand{\qed}{\quad \blacksquare}

\newcommand{\vecs}[1]{\langle #1\rangle}
\newcommand{\brak}[1]{\left\langle #1 \right\rangle}
\newcommand{\bra}[1]{\left\langle #1 \right\vert}
\newcommand{\ket}[1]{\left\vert #1 \right\rangle}
\newcommand{\abs}[1]{\left\vert #1 \right\vert}
\newcommand{\norm}[1]{\left\vert \left\vert #1 \right\vert \right\vert}
\newcommand{\st}{\text{ s.t. }} 

\newcommand{\ep}{\varepsilon}
\DeclareMathOperator{\gcf}{GCF}

\usepackage{tcolorbox}
\tcbuselibrary{breakable, skins}
\tcbset{enhanced}
\newenvironment*{tbox}[2][gray]{
    \begin{tcolorbox}[
        parbox=false,
        colback=#1!5!white,
        colframe=#1!75!black,
        breakable,
        title={#2}
    ]}
    {\end{tcolorbox}}


\title{Math 1010: One-Variable Analysis}
\author{Milan Capoor}
\date{Spring 2024}

\renewcommand*{\thesection}{Lecture \arabic{section} - \hspace*{-1em}}
\renewcommand*{\thesubsection}{\arabic{section}}

\begin{document}
\maketitle
\chapter{The Real Numbers}
\section{Jan 24:} 
    \subsection*{Preliminaries}
        \begin{enumerate}
            \item Sets
            
            \textbf{Definition:} A \emph{set} is a collection of objects.

            \textbf{De Morgan's Laws:}
            \begin{align*}
                (A \cap B)^c  = A^c \cup B^c\\
                (A \cup B)^c = A^c \cap B^c
            \end{align*}

            \emph{Proof:} HW

            \item Functions
            
            \textbf{Definition:} Given two sets $A, B$, a \emph{function} $f: A \to B$ is a rule that assigns to each $a \in A$ a unique element $f(a) \in B$.

            The \emph{domain} of $f$ is $A$. The \emph{range} of $f$ is a subset of $B$.

            Examples:
            \begin{enumerate}
                \item Dirichlet Function:
                \[g(x) = \begin{cases}
                    1 \quad x \in \Q\\
                    0 \quad x \notin \Q
                \end{cases}\]
                (Its domain is $\R$ and its range is $\{0, 1\}$)

                \item Absolute value function:
                \[\abs{x} = \begin{cases}
                    x \quad &x \geq 0\\
                    -x \quad &x < 0
                \end{cases}\]

                Properties:
                \begin{align*}
                    \abs{ab} &= \abs{a} \cdot \abs{b}\\ 
                    \abs{a + b} &\leq \abs{a} + \abs{b} \qquad (\text{Triangle Inequality})
                \end{align*}
            \end{enumerate}

            \item Proofs 
            
            Types of Proofs:
            \begin{itemize}
                \item \emph{Direct Proof} - Start with a valid statement (usually the hypothesis) and proceed by logical steps 
                
                \item \emph{Indirect Proof (Proof by Contradiction)} - Begin by negating the conclusion and proceed by logical steps to a contradiction.
            \end{itemize}
            
            \begin{tbox}{\textbf{Theorem:} Let $a, b \in \R$. Then $a = b \iff \forall \varepsilon > 0, \abs{a - b} < \varepsilon$}
                \emph{Proof:} We have two statements:
                \begin{itemize}
                    \item If $a = b \implies \forall \varepsilon > 0, \abs{a - b} < \varepsilon$
                    \item If $\forall \varepsilon > 0, \abs{a - b} < \varepsilon \implies a = b$
                \end{itemize}

                \emph{Proof of first statement:} Suppose $a = b$. Then $\abs{a - b} = 0$. Thus, $\forall \varepsilon > 0,\; \abs{a - b} < \varepsilon$.

                \emph{Proof of second statement:} Assume $a \neq b$. Then $\exists\, \varepsilon_0 > 0$ s.t. $\abs{a - b} = \varepsilon_0$ But this is contradiction by hypothesis. $\qed$
            \end{tbox}

            \emph{Proof by induction:}

            \begin{tbox}{Example: Let $x_1 = 2$ and $\forall n \in \N$, define $x_{n+1} = \frac{x_n + 5}{3}$, $n \geq 1$. Prove that $x_n$ is increasing.}
                \emph{Proof:} 
                \begin{enumerate}
                    \item Base Case:
                    \[x_1 = 2 < x_2 = \frac{7}{3} \quad \checkmark\]
                    \item Inductive Step: Assume $x_n \leq x_{n+1}$. Then 
                    \[\underbrace{\frac{x_n + 5}{3}}_{x_{n+1}} \leq \underbrace{\frac{x_{n+1} + 5}{3}}_{x_{n+2}} \implies x_{n+1} \leq x_{n+2} \qed\]
                \end{enumerate}
            \end{tbox}
        \end{enumerate}
    
    \subsection*{Axioms for the real numbers}
        \begin{itemize}
            \item \textbf{Field Axioms:} $\forall a, b, c \in \R$
            \begin{enumerate}
                \item $(a + b) + c = a + (b + c)$ (Additive Associativity)
                \item $\exists\, 0 \in \R$ s.t. $a + 0 = a$ (Additive Identity)
                \item $\exists\, -a \in \R$ s.t. $a + (-a) = 0$ (Additive Inverse)
                \item $a \cdot b = b \cdot a$ (Commutativity)
                \item $(a \cdot b) \cdot c = a \cdot (b \cdot c)$ (Multiplicative Associativity)
                \item $\exists\, 1 \in \R$ s.t. $a \cdot 1 = a$ (Multiplicative Identity)
                \item $\exists\, a^{-1} \in \R$ s.t. $a \cdot a^{-1} = 1$ (Multiplicative Inverse)
                \item $a \cdot (b + c) = a \cdot b + a \cdot c$ (Distributivity)
            \end{enumerate}

            \item \textbf{Order Axioms:} there exists a subset of positive numbers $P$ such that 
            \begin{enumerate}
                \setcounter{enumi}{9}
                \item exclusively either $a \in P$ or $-a \in P$ or $a = 0$ (Trichotomy) 
                \item $a, b \in P \implies a + b \in P$ (Closure under addition)
                \item $a, b \in P \implies a \cdot b \in P$ (Closure under multiplication)
            \end{enumerate}

            \item \textbf{Completeness Axiom:} a least upper bound of a set $A$ is a number $x$ such that $x \geq y$ for all $y \in A$, and such that if $z$ is also an upper bound of $A$, then $z \geq x$. 
            \begin{enumerate}
                \setcounter{enumi}{12}
                \item Every nonempty set $A$ which is bounded above has a least upper bound. 
            \end{enumerate}
        \end{itemize}

        We will call Properties 1-12, and anything that follows from them, \emph{elementary arithmetic}. These alone imply that $\Q$ is a subfield of $\R$ and basic properties of inequalities under addition and multiplication. 

        Adding Property 13 uniquely determines the real numbers. The standard proof is to identify each $x \in \R$ with the subset of rationals $\{y \in \Q: y < x\}$, \emph{the Dedekind cut}. This can also construct the reals from the rationals. 

\section{Jan 30:}
    \subsection*{Axiom of Completeness} 
        \begin{enumerate}
            \item $\R$ is an ordered field. 
            \item There is a least upper bound and a greatest lower bound 
        \end{enumerate}

        \emph{Note:} the axiom of completeness is only true for $\R$

        \textbf{Definition:} Let $A \subseteq \R$ be a set. Then: 
        \begin{enumerate}
            \item $A$ is \emph{bounded above} if $\exists\, b \in \R$ s.t. $a \leq b$ for all $a \in A$. Conversely, then $b$ is an \emph{upper bound} of $A$. 
            \item $A$ is \emph{bounded below} if $\exists\, l \in \R$ s.t. $a \geq l$ for all $a \in A$. Conversely, then $l$ is a \emph{lower bound} of $A$.
        \end{enumerate}

        \textbf{Definition:} $s \in \R$ is \emph{least upper bound} of $A \subseteq \R$ if
        \begin{enumerate}
            \item $s$ is an upper bound of $A$
            \item if $b$ is any upper bound for $A$, then $s \leq b$
        \end{enumerate} 

        $s$ is called \emph{the supremum of $A$} and is denoted $s := \sup A$. Further, it is unique. 

        Similarly, $\inf A$ (the \emph{infimum}) is the greatest lower bound of $A$.

        \emph{Example:} $A = \{\frac{1}{n}: n \in \N\} = \{1, \frac{1}{2}, \frac{1}{3}, \frac{1}{4}, \dots\}$. Then $\sup A = 1$.

        \emph{Proof:} 
        \begin{enumerate}
            \item $1 \geq \frac{1}{n}$ for all $n \in \N$ $\checkmark$ 
            \item Assume $b$ is another upper bound. Since $1 \in A$, $1 \leq b$ $\qed$
        \end{enumerate}

        \textbf{Remark:} $\sup A$ and $\inf A$ do not have to be elements of $A$. 
        \begin{itemize}
            \item When $\sup A \in A$, we call it the \emph{maximum}
            \item When $\inf A \in A$, we call it the \emph{minimum}
        \end{itemize}

        \emph{Example:} In the example above, $\inf A = 0 \notin A$.

        \emph{Example:}
        \begin{align*}
            (0, 2) &= \{x \in \R: \underbrace{0}_{\inf} < x < \underbrace{2}_{\sup}\}\\ 
            [0, 2] &= \{x \in \R: \underbrace{0}_{\min} \leq x \leq \underbrace{2}_{\max}\}
        \end{align*}
        
        \begin{tbox}{\textbf{Theorem:} There is no rational number whose square is 2}
            \emph{Proof:} Suppose $\exists, p, q \in \Z \st (\frac{p}{q})^2 = 2$. We further assume that $q \neq 0$ and $\gcf(p, q) = 1$. 

            Then 
            \[(\frac{p}{q})^2 = 2 \implies \frac{p^2}{q^2} = 2 \implies p^2 = 2q^2\]
            Thus, $p^2$ is even so $p$ is even (because the product of two odd numbers is odd). 

            Thus, we can write $p = 2r, \; r \in \Z$. Substituting, 

            \[(2r)^2 = 2q^2 \implies 4r^2 = 2q^2 \implies 2r^2 = q^2\]
            
            By similar logic, $q$ is even. But this contradicts our assumption that $\gcf(p, q) = 1$. $\qed$
        \end{tbox}

        This allows us to show that $\Q$ has gaps (it is incomplete). Consider: 
        \[S = \{r \in \Q: r^2 < 2\}\] 
        A sensible upper bound is $\sqrt 2 \approx 1.4142\dots$. Since $\sqrt 2 \notin \Q$, we need to approximate it with rational numbers. We can get infinitely close, 
        \[\frac{3}{2}, \frac{142}{100}, \frac{1415}{1000}, \dots\] 
        but because we need infinitely many terms, we do not have a least upper bound (the next term will always be closer). 

        \begin{tbox}{\textbf{Lemma:} Let $s \in \R$ be an upper bound for a set $A \subseteq \R$. Then $s = \sup A$ iff $\forall \varepsilon > 0\; \exists a \in A \st s - \varepsilon < a$}
            \emph{Proof:} 
            \begin{enumerate}
                \item Suppose $s = \sup A$. Consider any $s - \varepsilon$ with $\varepsilon > 0$. From the definition of supremum, $s - \varepsilon$ is not an upper bound for $A$ (because $s - \varepsilon < \sup A$). Thus, $\exists a \in A \st s - \varepsilon < a$ 

                \item Suppose $\forall \varepsilon > 0\; \exists\, a \in A \st s - \varepsilon < a$. 
                
                Since $s - \varepsilon < a$, it cannot be an upper bound by definition. Thus, for any $b < s$, $b$ is not an upper bound. Therefore, any upper bound $b'$ must satisfy $s \leq b'$. This is precisely the definition of $\sup A$. $\qed$
            \end{enumerate}
        \end{tbox}

\section{Feb 1:}
    \subsection*{Recall}
        \begin{itemize}
            \item $\R$ is an ordered field satisfying the Axiom of Completeness 
            \item $\Q$ is an ordered field but does not satisfy the Axiom of Completeness
            \item $\Z$ satisfies the AOC but is not a field (so we ignore it in analysis)
            \item $s = \sup A \implies a \leq b$ for any other upper bound $b$ 
        \end{itemize}

    \subsection*{Consequences of Completeness}
        \begin{tbox}{\textbf{Theorem (Nested interval property):} For each $n \in \N$, assume we are given a closed interval $I_n = [a_n, b_n] = \{x \in \R: a_n \leq x \leq b_n\}$. Assume also that $I_n$ contains $I_{n+1}$. Then the resulting nested sequence $I_1 \supseteq I_2 \supseteq I_3 \supseteq \dots$ has a nonempty intersection $\bigcap_{n=1}^{\infty} I_n \neq \emptyset$}
            \emph{Proof:} 

            Let $A = \{a_n: n \in \N\}$ be the set of all left endpoints of the intervals $I_n$. Then $A$ is nonempty and bounded above by the $b$ (right) endpoints. 

            Consider $x = \sup A$. We know $a_n \leq x \leq b_n$ for all $n \in \N$ by the fact that $x$ is an upper bound for $A$ and that it is the \emph{least} upper bound for $A$.  

            And indeed, this is exactly the intersection of the intervals. $\qed$
        \end{tbox}

        Note that the theorem does not hold for $\Q$! Imagine the series of intervals centered at $\frac{1}{\sqrt{2}}$ -- all are non-empty but their intersection is empty (because there are rational numbers infinitely close to $\frac{1}{\sqrt{2}}$ but that final interval would be empty).

        \begin{tbox}{\textbf{Theorem (Archimedian Property):} Given any number $x \in \R$, $\exists n \in \N$ satisfying $n > x$. (i.e. $\N$ is \emph{not} bounded above)}
            \emph{Proof by contradiction:} 

            Suppose $\N$ is bounded above. By the axiom of completeness, $\N$ has a least upper bound $\alpha = \sup \N$. By definition of supremum, $\alpha - 1 < n \implies \alpha < n + 1$. But $n + 1 \in \N$, so $\alpha$ is not an upper bound. $\qed$
        \end{tbox}

        \textbf{Consequence:} Given any real number $y > 0$, $\exists n \in \N$ satisfying $\frac{1}{n} < y$.

        \emph{Proof:} Let $x = \frac{1}{y}$. By the Archimedean Property, $\exists n \in \N$ satisfying $n > x$. Then $n > \frac{1}{y} \implies y < \frac{1}{n}$

        \begin{tbox}{\textbf{Theorem (Density of $\Q$ in $\R$):} For every two real numbers $a$ and $b$ with $a < b$, $\exists r \in \Q$ s.t. $a < r < b$}
            \emph{Proof:} 

            We want to show that $\exists m \in \Z, n \in \N: a < \frac{m}{n} < b$. 

            First note that we can choose $m \in \Z, n \in \N$ to bound $a$. We choose $n$ such that 
            \[\frac{m-1}{n} < a < \frac{m}{n}\]
            and $m$ to be the smallest integer greater than $na$: 
            \[m - 1 \leq na < m\] 
            The RHS inequality gives $a < \frac{m}{n}$. 

            By Archimedean property, we can pick $n \in \N$ such that $\frac{1}{n} < b - a$. Equivalently, $a < b - \frac{1}{n}$.

            The LHS gives 
            \[m \leq na + 1 < n(b - \frac{1}{n}) + 1 = nb \implies m < nb \implies \frac{m}{n} < b\]

            Thus, 
            \[a < \frac{m}{n} < b \qed\]
        \end{tbox}

        \textbf{Corollary:} Density of Irrationals ($\mathbb{I}$) in $\R$ 

    \subsection*{Cardinality}
        \textbf{Definition:} \emph{Cardinality} is the size of a set 

        \textbf{Definition:} 
        \begin{itemize}
            \item A function $f: A \to B$ is \emph{injective} (or \emph{one-to-one}) if $a_1 \neq a_2$ in $A$ implies $f(a_1) \neq f(a_2)$. 
            \item A function $f: A \to B$ is \emph{surjective} (\emph{onto}) if, given any $b \in B$, it is possible to find an element $a \in A$ for which $f(a)= b$ (all elements in $B$ have a pre-image in $A$)
            \item A function $f: A \to B$ is \emph{bijective} (has a ``1-to-1 correspondence'') if it is both injective and surjective
        \end{itemize}

        \textbf{Definition:} The set $A$ has the same cardinality as the set $B$ if there exists a bijection $f: A \to B$.
        
        \emph{Example:} $E = \{2, 4, 6, 8, \dots\}$. We create an equivalence relation $\N \sim E$ induced by $f: \N \to E$ given by $f(n) = 2n$. Thus $\N$ and $E$ have the same cardinality.

        \emph{Example:} $\N \sim \Z$. Consider
        \[f(n) = \begin{cases}
            \frac{n-1}{2} &n \text{ is odd}\\
            -\frac{n}{2} &n \text{ is even}
        \end{cases}\]
        Proof of bijection is left as an exercise.

        \emph{Example:} $(a, b) \sim \R$ 

\section{Feb 6:}
    \subsection*{Countable Sets}
        \textbf{Definition:} A set $A$ is \emph{countable} if $A \sim \N$ (it has the same cardinality as $\N$)

        \begin{tbox}{\textbf{Theorem:} $\Q$ is countable}
            \emph{Proof:} 
                It suffices to construct a bijection $\phi: \N \to \Q$. 

                Consider $A_1 =\{0\}$ and for each $n \geq 2$,
                \[A_n = \{\pm \frac{p}{q}: p, q \in \N \quad \text{ with p/q in lowest term with } p + q = n\}\]
                i.e., $A_2 = \{1, -1\}, \; A_3 = \{\frac{1}{2}, -\frac{1}{2}, 2, -2\}, \; A_4 = \{\pm \frac{1}{3}, \pm 3\}$

                We know that each $A_n$ is finite. Further, every rational number appears \emph{exactly} once in these sets. 

                We can then define $\phi: \N \to \Q$ by the one-to-one correspondence between the natural numbers and each element of the $A_n$'s
                \[\begin{array}{ccccccccc}
                    \N: & 1 & 2 & 3 & 4 & 5 & 6 & 7 & \dots\\ 
                    \Q: & 0 & 1 & -1 & \frac{1}{2} & -\frac{1}{2} & 2 & -2 & \dots\\[-4pt]
                        & \multicolumn{1}{c}{\underbrace{\rule{0.25cm}{0pt}}_{A_1}} & \multicolumn{2}{c}{\underbrace{\rule{1cm}{0pt}}_{A_2}} & \multicolumn{4}{c}{\underbrace{\rule{2.5cm}{0pt}}_{A_3}} &            
                \end{array}\]
                The correspondence is onto: every rational will appear. (e.g. $\frac{22}{7}\in A_{29}$) 

                The correspondence is 1-1: each rational appears exactly once. $\qed$
        \end{tbox}

        \begin{tbox}{\textbf{Theorem:} $\R$ is uncountable}
            \emph{Proof:} Assume $\R$ is countable. Then $\R = \{x_1, x_2, \dots\}$

            Let $I_1$ be a closed interval which does not contain $x_1$. Then $I_2 \subseteq I_1$ and does not contain $x_2$. By induction, $I_{n+1} \subseteq I_n,\; x_n \notin I_n$ 

            Consider $\bigcap_{n=1}^{\infty} I_n$. If $x_{n_0}$ is in the list, $\exists\, I_{n_0} \st x_{n_0} \notin I_{n_0}$. But then 
            \[\bigcap_{n=1}^{\infty} I_n = \emptyset\]

            However, by the nested interval property, $\bigcap_{n=1}^{\infty} I_n \neq \emptyset$.
        \end{tbox}

        \begin{tbox}{\textbf{Theorem:} If $A \subseteq B$ and $B$ is countable, then $A$ is countable or finite}
            \emph{Proof:} HW
        \end{tbox}

        \begin{tbox}{\textbf{Theorem:} 
            \color{white}
            \begin{enumerate}
                \item If $A_1, A_2, \dots, A_m$ are countable, then $\bigcup_{n=1}^{m} A_n$ is countable
                \item If $A_1, A_2, \dots$ are countable, then $\bigcup_{n=1}^{\infty} A_n$ is countable
            \end{enumerate} }
            \emph{Proof:} HW
        \end{tbox}

\chapter{Sequences and Series}
\section{Feb 6 (Continued):} 
    \subsection*{The Limit of a Sequence}
        \textbf{Definition:} A \emph{sequence} is a function whose domain is $\N$

        \emph{Examples:}
        \begin{itemize}
            \item $(1, \frac{1}{2}, \frac{1}{3}, \frac{1}{4}, \dots) = (\frac{1}{n})_{n \in \N}$ 
            \item $(\frac{1+n}{n})^{\infty}_{n=1} = (2, \frac{3}{2}, \frac{4}{3}, \frac{5}{4}, \dots)$
            \item $x_1 = 2, \; x_{n+1} = \frac{x_n + 1}{2}$
        \end{itemize}

        \textbf{Definition (convergence of a sequence):} A sequence $(a_n)$ \emph{converges} to a real number $a$ if, for every positive number $\varepsilon$, there exists a $N \in \N$ such that $n \geq N$ implies $\abs{a_n - a} < \varepsilon$:
        \begin{align*}
            \lim_{n\to \infty} a_n = a &\iff a_n \to a \\ 
                &\iff \forall \varepsilon > 0, \; \exists\, N \in \N \st n \geq N \implies \abs{a_n - a} < \varepsilon
        \end{align*}

        \textbf{Definition ($\varepsilon$-neighborhood):} The \emph{$\varepsilon$-neighborhood} of $a \in \R$ (given $\varepsilon > 0$) is the set $V_{\varepsilon}(a) = \{x \in \R: \abs{x - a} < \varepsilon\}$

        Here, $\varepsilon$ is the radius about the center $a$. 

        \textbf{Definition:} A sequence $(a_n)$ converges to $a$ if, given any $\varepsilon$-neighborhood $V_{\varepsilon}(a)$ if $a$, there exists a point in the sequence after which all the terms are in $V_{\varepsilon}(a)$

\section{Feb 08:}
    \subsection*{Convergence}
        \textbf{Example:} Let $a_n = \frac{1}{\sqrt n}$. Show $\lim_{n\to \infty} a_n = 0$. 

        First we try a few values of epsilon:
        \begin{itemize}
            \item $\varepsilon = \frac{1}{10}$: $(0 - \frac{1}{10}, 0 + \frac{1}{10}) = (-\frac{1}{10}, \frac{1}{10})$
            
            When $n = 100 \implies a_{100} = \frac{1}{10}$. So the first element in the interval is $a_{101}$. 

            \item $\varepsilon = \frac{1}{50}$: $(-\frac{1}{50}, \frac{1}{50})$ 
            
            Here, the first element in the interval is $a_{2501}$.
        \end{itemize}

        Now for the rigorous version: Let $\varepsilon > 0$. Choose $N \in \N$ such that $N > \frac{1}{\varepsilon^2} \implies \frac{1}{\sqrt{N}} < \varepsilon$. 

        Let $n \geq N$. Then 
        \[n > \frac{1}{\varepsilon^2} \implies \frac{1}{\sqrt{N}} < \varepsilon \implies \abs{\frac{1}{\sqrt{n} - 0}} < \varepsilon\]

        \textbf{A template for convergence proofs:}
            \begin{enumerate}
                \item Let $\varepsilon > 0$
                \item Demonstrate a choice for $N \in \N$
                \item Verify $N$
                \item With $N$ well chosen, it should be possible to get $\abs{x_n- x} < \varepsilon$ 
            \end{enumerate}

        \textbf{Example:} Prove that $\lim \frac{n+1}{n} = 1$ 

        We want $\abs{\frac{n+1}{n} - 1} < \varepsilon$. This is equivalent to $\abs{\frac{1}{n}} < \varepsilon$. So we choose $N \in \N > \frac{1}{\varepsilon}$. 

        The actual proof then reads: Let $\varepsilon > 0$. Choose $N \in \N \st N \frac{1}{\varepsilon}$. Let $n \geq N$. 
        \[n > \frac{1}{\varepsilon} \implies \frac{1}{n} < \varepsilon \implies \abs{\frac{n+1}{n} - 1} < \varepsilon\]

        \begin{tbox}{\textbf{Theorem (Uniqueness of limits):} The limit of a sequence, when it exists, is unique}
            \emph{Proof:} HW         
        \end{tbox}

    \subsection*{The algebraic and order limit theorems}
        \textbf{Definition:} A sequence $(x_n)$ is bounded if there exists a number $M > 0$, such that $\abs{x_n} \leq M$ for all $n \in \N$. 

        \begin{tbox}{\textbf{Theorem:} Every convergent sequence is bounded}
            \emph{Proof:} Assume $(x_n)$ converges to $l$. 

            Given $\varepsilon > 0$, 
            \[\exists\, N \in \N \st x_n \in (l - \varepsilon, l + \varepsilon) \; \forall n \geq N\]

            Since we do not know if $l$ is positive or negative, we can only say 
            \[\abs{x_n} < \abs{l} + \varepsilon\] 

            From this we know $x$ is bounded for $n \geq N$. Now we check the case $n < N$. Luckily, this is a finite number of cases. 

            By construction, $M = \max\{\abs{x_1}, \abs{x_2}, \dots, \abs{x_{N-1}}, \abs{l} + 1\}$. Then $\abs{x_n} \leq M$ for all $n \in \N$. $\qed$   
        \end{tbox}

        \begin{tbox}{\textbf{Theorem (Algebraic Limit Theorems):} Let $\lim a_n = a$, $\lim b_n = b$
            \begin{enumerate}
                \item $\lim (ca_n) = ca, \quad \forall c \in \R$
                \item $\lim (a_n + b_n) = a + b$
                \item $\lim (a_n \cdot b_n) = a \cdot b$
                \item $\lim \frac{a_n}{b_n} = \frac{a}{b}$, provided $b \neq 0$
            \end{enumerate}}

            \emph{Proof:} 
            \begin{enumerate}
                \item Let $\varepsilon > 0$. We want to show $\abs{ca_n - ca} < \varepsilon$. Notice 
                \[\abs{ca_n - ca} = \abs{c} \cdot \abs{a_n - a}\]
                Since $a_n$ is convergent, we can make $\abs{a_n - a}$ arbitrarily small.

                We choose $N \in \N \st \abs{a_n - a} < \frac{\varepsilon}{\abs{c}}$ so $\forall n > N$, 
                \[\abs{ca_n - ca} < \abs{c} \frac{\varepsilon}{\abs{c}} = \varepsilon \quad \checkmark\]

                \item Let $\varepsilon > 0$. We want to show $\abs{a_n + b_n - (a + b)} < \varepsilon$. We can say $\abs{a_n - a + b_n - b} \leq \abs{a_n - a} + \abs{b_n - b}$ by the Triangle inequality. Then since $a_n$ and $b_n$ are convergent, we note that 
                \begin{align*}
                    \exists &N_1 \in \N, \st \forall n \geq N_1: \quad \abs{a_n - a} < \frac{\varepsilon}{2}\\ 
                    \exists &N_2 \in \N, \st \forall n \geq N_2: \quad \abs{b_n - b} < \frac{\varepsilon}{2}
                \end{align*} 
                Choose $N = \max\{N_1, N_2\}$ so 
                \[\forall n \geq N: \quad \abs{(a_n + b_n) - (a + b)} < \frac{\varepsilon}{2} + \frac{\varepsilon}{2} = \varepsilon \quad \checkmark\]
                
                \item Let $\varepsilon > 0$. We want to show that $\abs{a_n \cdot b_n - a \cdot b} < \varepsilon$. We can say
                \begin{align*}
                    \abs{a_nb_n - ab_n + ab_n - ab} &\leq \abs{a_nb_n - ab_n} + \abs{ab_n - ab}\\ 
                        &= \abs{b_n} \cdot \abs{a_n - a} + \abs{a} \cdot \abs{b_n - b}
                \end{align*}

                Since $a_n$ and $b_n$ are convergent, $\exists N_1 \in \N, \st \forall n \geq N_1: \quad \abs{b_n - b} < \frac{\varepsilon}{2\abs{a}}$. Note then that $b_n$ is convergent so bounded: $\abs{b_n} \leq M$. Then $\exists N_2, \st \forall n \geq N_2: \quad \abs{a_n - a} < \frac{\varepsilon}{2M}$ 

                So with $N= \max{N_1, N_2}$, $\forall n \geq N$, we have
                \[\abs{a_nb_n - ab} \leq M \cdot \frac{\ep}{2M} + \abs{a} \cdot \frac{\ep}{2\abs{a}} = \ep\]

                \item Let $\varepsilon > 0$. We want to show that $\abs{\frac{a_n}{b_n} - \frac{a}{b}} < \varepsilon$. This is the same as showing $a_n \cdot \frac{1}{b_n} \to a\cdot\frac{1}{b}$ so it suffices to show that $\frac{1}{b_n} \to \frac{1}{b}$ and apply the multiplicative limit theorem. 
                
                Observe:
                \[\abs{\frac{1}{b_n} - \frac{1}{b}} = \abs{\frac{b - b_n}{b_n \cdot b}} = \frac{\abs{b - b_n}}{\abs{b_n} \cdot \abs{b}}\]
            
                Intuitively, finding a lower bound for $b_n$ gives an upper bound for $1/b_n$. \emph{Trick:} Choose a large $n$ such that $\abs{b_n - b} > \abs{b_n - 0} \implies \abs{b_n} > \frac{\abs{b}}{2}$. By convergence of $(b_n)$, $\exists N_1 \in \N \st \forall n \geq N: \abs{b_n - b} < \frac{\abs{b}}{2}$. Then $\abs{b_n} > \frac{\abs{b}}{2}$. 

                Now bound $\abs{b_n - b} < \frac{\ep \abs{b}^2}{2}$ by convergence at $N_2 \in \N$. 
                
                Finally, let $N = \max\{N_1, N_2\}$ then for $n > N$, 
                \[\abs{\frac{1}{b_n} - \frac{1}{b}} = \frac{\abs{b - b_n}}{\abs{b_n} \cdot \abs{b}} < \frac{\ep \abs{b}^2}{2} \cdot \frac{2}{\abs{b}} \cdot \frac{1}{\abs{b}} = \ep \qed\]
            \end{enumerate}
        \end{tbox}

\section{Feb 15:}
    \begin{tbox}{\textbf{Theorem (Order Limit Theorem):} Assume $(a_n) \to a$, $(b_n) \to b$. 
        \begin{enumerate}
            \item If $a_n \geq 0 \quad \forall n \in \N$, then $a \geq 0$
            \item If $a_n \leq b_n \quad \forall n \in \N$, then $a \leq b$
            \item If $\exists c \in \R \st c \leq b_n \quad \forall n \in \N$, then $c \leq b$
        \end{enumerate}}

        \emph{Proof:} 
            \begin{enumerate}
                \item Suppose $a < 0$. Consider $\ep = \abs{a}$ so $\exists N \in \N \st \forall n \geq N: \abs{a_n - a} < \abs{a}$. However, since $a <0$, this tells us 
                \[a < a_n - a < -a \implies a_n < 0\]
                But this contradicts the fact that $a_n \geq 0$.

                \item By the Algebraic limit theorem, $(b_n - a_n) \to b - a$. Since $a_n \leq b_n$ for all $n \in \N$, $b_n - a_n \geq 0$, by part 1, $b - a \geq 0 \implies b \geq a$
                
                \item Take $a_n = c \quad \forall n \in \N$. Then $(a_n) \to c$. The result follows from part 2. $\qed$
            \end{enumerate}
    \end{tbox}

    \subsection*{Monotone Convergence Theorem}
        \textbf{Definition:} A sequence $(a_n)$ is \emph{increasing} if $a_n \leq a_{n+1}$ for all $n \in \N$. It is \emph{decreasing} if $a_n \geq a_{n+1} \quad \forall n \in \N$. 

        A sequence is \emph{monotone} if it is either increasing or decreasing for all $n \in \N$.
        
        \begin{tbox}{\textbf{Theorem (Monotone Convergence Theorem):} If a sequence is monotone and bounded, then it is convergent} 
            \emph{Proof:} Let $(a_n)$ be monotone and bounded. Assume WLOG that $(a_n)$ is increasing. Consider the set $A = \{a_n: n \in \N\}$. SInce $(a_n)$ is bounded, $sup A$ exists. 

            We claim $\lim_{n\to \infty} a_n = \sup A$. Let $\ep > 0$. Since $\sup A$ is the least upper bound, $\sup A - \ep$ is not an upper bound. Thus, $\exists N \in \N \st a_N > \sup A - \ep$. Since $a_n$ is monotone, $a_n > \sup A - \ep \quad \forall n \geq N$. Further, $a_n \leq \sup A + \ep$ so 
            \[\abs{a_n - \sup A} < \ep\]
        \end{tbox}

    \subsection*{Series Introduction} 
        \textbf{Definition (Convergence of Series):} Let $(b_n)$ be a sequence. A \emph{infinite series} is an expression of the form 
        \[\sum_{n=1}^{\infty} b_n = b_1 + b_2 + \dots\]

        The series \emph{converges} to $S$ if the sequence of \emph{partial sums} $(S_n)$ given by 
        \[S_m = \sum_{n=1}^m b_n = b_1 + \dots + b_m\]
        converges to $S$.

        \textbf{Example:} Consider $\sum_{n=1}^{\infty} \frac{1}{n^2}$. 
        \[S_m = 1 + \frac{1}{4} + \frac{1}{9} + \dots + \frac{1}{m^2}\]
        
        We seek an upper bound for $(S_m)$. Notice 
        \begin{align*}
            S_m &= \frac{1}{2 \cdot 2} + \frac{1}{3 \cdot 3} + \frac{1}{4 \cdot 4} + \dots + \frac{1}{m \cdot m}\\ 
                &< \frac{1}{1 \cdot 2} + \frac{1}{2 \cdot 3} + \frac{1}{3 \cdot 4} + \dots + \frac{1}{(m-1) \cdot m}\\ 
                &= 1 + (1 - \frac{1}{2}) + (\frac{1}{2} - \frac{1}{3}) + (\frac{1}{3} - \frac{1}{4}) + \dots + (\frac{1}{m-1} - \frac{1}{m})\\
                &= 2 - \frac{1}{m} < 2
        \end{align*}

        Since $(S_m)$ has an upper bound and is increasing, it is convergent to some limit $s$. 

        \textbf{Example (Harmonic Series):} Consider $\sum_{n=1}^{\infty} \frac{1}{n}$. Taking partial sums, 
        \[S_m = 1 + \frac{1}{2} + \frac{1}{3} + \dots + \frac{1}{m}\] 

        Is $S_m$ bounded? No! 
        \[S_4 = 1 + \frac{1}{2} + \frac{1}{3} + \frac{1}{4} = 2\] 
        But 
        \[S_8 > 2 + \frac{1}{2}\]
        and 
        \[S_{2^k} > 1 + k(\frac{1}{2})\]
        and this is unbounded! 

\section{Feb 22:}
    \begin{tbox}{\textbf{Theorem (Cauchy Condensation Test):} Suppose $(b_n)$ is decreasing and $b_n \geq 0 \quad \forall n \in \N$. Then $\sum_{n=1}^{\infty} b_n$ converges iff $\sum_{n=1}^{\infty} 2^n b_{2^n} = b_1 + 2b_2 + 4b_4 + 8b_8 + \dots$ converges }
        \emph{Proof:} Omitted. 

        \textbf{Remark:} This is a mostly useless theorem used only for showing the harmonic series diverges. 
    \end{tbox}

    \textbf{Corollary:} The series $\sum_{n=1}^{\infty} \frac{1}{n^p}$ converges iff $p > 1$.

    \subsection*{Subsequences}
        \textbf{Definition:} Let $(a_n)$ be a sequence and let $n_1 < n_2 < n_3 < \dots$ be an increasing sequence of natural numbers. Then the sequence $(a_{n_1}, a_{n_2}, a_{n_3}, \dots)$ is a \emph{subsequence} of $(a_n)$ and is denoted by $(a_{n_k})$ where $k \in \N$ is the index.

        \textbf{Example:} 
        \[\begin{array}{c|ccccccccc}
            a_1 & a_2 & a_3 & a_4 & a_5 & a_6 & a_7 & \dots\\
            \hline
            1 & 5 & -3 & 10 & 0 & -8 & 12 & \dots
        \end{array}\]
        If we choose $n_1 = 3, n_2 = 4, n_3 = 6, \dots$ then $(a_{n_k}) = (-3, 10, -8, \dots)$

        \textbf{Note:} The order of the terms in the subseq is the same as in the original sequence. Further, no repetitions are allowed. 

        \textbf{Examples:} $(a_n) = (1, \frac{1}{2}, \frac{1}{3}, \frac{1}{4}, \frac{1}{5}, \dots)$
        \begin{itemize}
            \item $(\frac{1}{2}, \frac{1}{4}, \frac{1}{6}, \frac{1}{8})$ is a subsequence
            \item $(\frac{1}{10}, \frac{1}{100}, \frac{1}{1000}, \frac{1}{10000}, \dots)$ is a subsequence 
            \item $(\frac{1}{10}, \frac{1}{5}, \frac{1}{100}, \frac{1}{5}, \dots)$ is \emph{not} a subsequence
            \item $(1, 1, \frac{1}{3}, \frac{1}{3}, \frac{1}{5}, \frac{1}{5}, \dots)$ is \emph{not} a subsequence
        \end{itemize}

        \begin{tbox}{\textbf{Theorem:} A subsequence of a convergent sequence converges to the same limit as the original sequence}
            \emph{Proof:} Assume $(a_n) \to a$. Let $(a_{n_k})$ be a subsequence. Given $\ep > 0$, $\exists N \in \N \st \forall n \geq N: \abs{a_n - a} < \ep$. Since $n_k \geq k \quad \forall k$, the same $N$ will suffice for the subsequence. Then, 
            \[\abs{a_{n_k} - a} < \ep \quad \forall k \geq N\qed\]
        \end{tbox}

        \textbf{Example:} Let $0 < b < 1$. Then $b > b^2 > b^3 > \dots > 0$. Therefore, $(b^n)$ is decreasing and bounded below. By the Monotone Convergence Theorem, $(b^n) \to l$. $(b^{2n})$ is a subsequence so by the Theorem above, $(b^{2n}) \to l$. However, 
        \[b^{2n} = b^n \cdot b^n \to l \cdot l \implies l^2 = l \implies l = 0\]
        Therefore, $(b_n) \to 0$. 

        \textbf{Example:} Consider the sequence $(1 , -\frac{1}{2}, \frac{1}{3}, -\frac{1}{4}, \frac{1}{5}, -\frac{1}{5}, \frac{1}{5}, -\frac{1}{5}, \dots)$. Does it converge? 

        Consider: 
        \begin{itemize}
            \item $(\frac{1}{5}, \frac{1}{5}, \frac{1}{5}, \frac{1}{5}, \dots) \to \frac{1}{5}$
            \item $(-\frac{1}{5}, -\frac{1}{5}, -\frac{1}{5}, -\frac{1}{5}, \dots) \to -\frac{1}{5}$
        \end{itemize}
        Since the subsequences do not converge to the same limit, the original sequence does not converge.

        \begin{tbox}{\textbf{Theorem (Bolzano-Weierstrass):} Every bounded sequence contains a convergent subsequence}
            \emph{Proof:} Let $(a_n)$ be a bounded subsequence. $\exists M > 0 \st \abs{a_n} \leq M \quad \forall n \in \N$.

            Split $[-M, M]$ into equal intervals $[-M, 0]$ and $[0, M]$. At least one these intervals must contain infinitely many terms of $(a_n)$. Call this interval $I_1$. WLOG, suppose $I_1 = [-M, 0]$. 

            Let $(a_{n_1})$ to be some term of $(a_n)$ which lies in $I_1$. Now we repeat: $I_1 = [-M, \frac{M}{2}] \cup [-\frac{M}{2}, 0]$. Label the interval with infinite terms $I_2$ and pick $(a_{n_2})$ from $I_2$ with $n_2 > n_1$. 

            In general, construct the closed $I_k$ by taking the half of $I_{k-1}$ containing infinitely many terms of $(a_n)$. Select $n_{k} > n_{k-1} > n_{k-2} > \dots > n_1$ such that $a_{n_k} \in I_k$. 

            Notice that the sets $I_1 \supseteq I_2 \supseteq I_3 \supseteq \dots$ are nested and closed. By the Nested Interval Property, $\exists x \in \R$ which lies in every $I_k$. Intuitively, this is a good limit candidate. 

            Now we seek to show that $(a_{n_k}) \to x$. Let $\ep > 0$. By construction, each $I_k$ has length $M(\frac{1}{2})^{k-1} \to 0$. $\exists N \in \N \st \forall k \geq N$, the length of $I_k$ is less than $\ep$. Since $x \in I_k$ and $a_{n_k} \in I_k$, $\abs{a_{n_k} - x} <\ep$. 

            Therefore, $(a_{n_k})$ is a convergent subsequence of the bounded sequence $(a_n)$. $\qed$
        \end{tbox}

\section{Feb 27:}
    \textbf{Recall:}
    \begin{itemize}
        \item A \emph{subsequence} of $(a_n)$ is a sequence $(a_{n_k})$ where $n_1 < n_2 < n_3 < \dots$
        \item Any subsequence of a convergent sequence converges to the same limit as the original sequence
        \item If two convergent subsequences converge to different limits, the original sequence diverges 
        \item \emph{Bolzano-Weierstrass Theorem:} Every bounded sequence contains a convergent subsequence
    \end{itemize}

    \subsection*{The Cauchy Criterion}
        \textbf{Definition:} A sequence $(a_n)$ is called a Cauchy sequence if $\forall \ep > 0$, 
        \[\exists N \in \N \st \abs{a_n - a_m} < \ep \quad \forall n, m \geq N\]

        \begin{tbox}{\textbf{Theorem:} Every convergent sequence is a Cauchy sequence}
            \emph{Proof:} Assume $(x_n)$ converges to $x$. To prove $(x_n)$ is a Cauchy sequence, we need to find a point in the sequence after which $\abs{x_n - x_m} < \ep$. 
            
            Since $(x_n) \to x$, $\forall \ep> 0$, $\exists N \in \N$ such that $\abs{x_n - x} < \frac{\ep}{2}$. 
            \[\abs{x_n - x_m} = \abs{(x_n - x) + (x - x_m)} \leq \abs{x_n - x} + \abs{x_m - x} < \ep\]
        \end{tbox}

        \begin{tbox}{\textbf{Lemma:} Cauchy sequences are bounded}
            \emph{Proof:} Set $\ep = 1$. Then $\exists N \in \N$ such that $\forall m, n \geq N$,
            \[\abs{x_n - x_m} < 1 \implies \abs{x_n} < \abs{x_N} + 1 \quad \forall n \geq N\]

            Then 
            \[M = \max\{\abs{x_1}, \abs{x_2}, \dots, \abs{x_{N-1}}, \abs{x_N} + 1\}\]
            is a bound for the sequence. $\qed$
        \end{tbox}

        \begin{tbox}{\textbf{Theorem (Cauchy Criternion):}  A sequence converges iff it is a Cauchy sequence}
            \emph{Proof:} The first direction follows from the fact that every convergent sequence is Cauchy. 

            For the other direction, assume $(x_n)$ is a Cauchy sequence. Then $(x_n)$ is bounded by the Lemma. By the Bolzano-Weierstrass Theorem, $(x_n)$ contains a convergent subsequence $(x_{n_k}) \to x$. 

            Since $(x_n)$ is Cauchy, $\forall \ep > 0$, $\exists N \in \N \st \abs{x_n - x_m} < \frac{\ep}{2} \quad \forall n, m \geq N$. 

            Since $(x_{n_k}) \to x$, choose $x_{n_k}$ with $n_k \geq N$. Then, 
            \[\abs{x_{n_k} - x} > \frac{\ep}{2}\]

            Now 
            \[\abs{x_n - x} \leq \abs{x_n - x_{n_k}} + \abs{x_{n_k} - x} < \frac{\ep}{2} + \frac{\ep}{2} = \ep \qed\]
        \end{tbox}

    \subsection*{Properties of Infinite Series}
        \textbf{Recall:}
        \begin{itemize}
            \item For a sequence $(a_1, a_2, a_3, \dots)$, the sequence of partial sums is given by 
            \[(S_m) = (S_1, S_2, S_3, \dots,) = (a_1, a_1 + a_2, a_1 + a_2 + a_3, \dots)\]

            \item A series $\sum_{n=1}^{\infty} a_n$ converges to $A$ if $\lim (S_m) = A$ 
        \end{itemize}

        \begin{tbox}{\textbf{Theorem (Algebraic Limit Theory for Series):} If $\sum_{k=1}^{\infty} = A$ and $\sum_{k=1}^{\infty} b_k = B$, then 
            \begin{enumerate}
                \item $\sum_{k=1}^{\infty} ca_k = cA, \quad \forall c\in \R$ 
                \item $\sum_{k=1}^{\infty} (a_k + b_k) = A + B$
            \end{enumerate}}
            
            \emph{Proof:} 
            \begin{enumerate}
                \item Since $\sum_{k=1}^{\infty}a_k = A$, $(S_m) = \sum_{k=1}^{m} a_k \to A$. Then $\lim(cS_m) = c\lim S_m = cA$ by the Algebraic Limit Theorem for Sequences (ALT). Then, by definition, $\sum_{k=1}^{\infty} ca_k = cA$
                
                \item Let $S_m = \sum_{k=1}^m a_k$ and $T_m = \sum_{k=1}^m b_k$. Then $S_m + T_m = \sum_{k=1}^m (a_k + b_k)$. Since $(S_m) \to A$ and $(T_m) \to B$, $(S_m + T_m) \to A + B$ by the ALT. Then $\sum_{k=1}^{\infty} (a_k + b_k) = A + B$ $\qed$
            \end{enumerate}
        \end{tbox}

        \begin{tbox}{\textbf{Theorem (Cauchy Criterion for Series):} The series $\sum_{k=1}^{\infty} a_k$ converges iff $\forall \ep > 0$, $\exists N \in \N \st \forall m \geq n \geq N$ we have 
            \[\abs{\sum_{k=m+1}^n a_k} = \abs{a_{m+1} + a_{m+2} + \dots + a_n} < \ep\]
        }
            \emph{Proof:} Define $S_n = a_1 + a_2 + \dots + a_n$. Observe that 
            \[\sum_{k=1}^{\infty} a_k \text{ converges } \iff (S_n) \text{ converges } \overset{*}{\iff} (S_n) \text{ Cauchy seq} \]
            where $\overset{*}{\iff}$ follows from the Cauchy Criterion for sequences.
        
            Further, if and only if $(S_n)$ is Cauchy, $\forall \ep > 0$, $\exists N \in \N \st \forall n > m \geq N$,
            \[\abs{S_n - S_m} = \abs{a_{m+1} + a_{m+2} + \dots + a_n} < \ep \qed\]
        \end{tbox}

\section{Feb 29:}
    \begin{tbox}{\textbf{Theorem:} If the series $\sum_{k=1}^{\infty} a_k$ converges, then $(a_k) \to 0$.}
        \emph{Proof:} Pick $n = m+1$ in previous theorem: for $m > N$, 
        \[\abs{a_{m+1}} < \ep\]
    \end{tbox}

    \textbf{Remark:} The converse is \emph{not} true! Consider the harmonic series: $a_n = \frac{1}{n} \to 0$ but $\sum_{n=1}^{\infty} a_n = \infty$ 

    \begin{tbox}{\textbf{Theorem (Comparison Test):} Assume $(a_k)$ and $(b_k)$ are sequences satisfying $0 \leq a_k \leq b_k$ for all $k \in \N$. Then 
        \begin{enumerate}
            \item If $\sum_{k=1}^{\infty} b_k$ converges, then $\sum_{k=1}^{\infty} a_k$ converges. 
            \item If $\sum_{k=1}^{\infty} a_k$ diverges, then $\sum_{k=1}^{\infty} b_k$ diverges.
        \end{enumerate}    
    }
        \emph{Proof:} Apply Cauchy Criterion for series and observe that 
        \[\abs{a_{m+1} + a_{m+2} + \dots + a_n} \leq \abs{b_{m+1} + \dots + b_n}\]
    \end{tbox}

    \textbf{Example (Geometric Series):} A series is called a \emph{geometric series} if it is of the form
    \[\sum_{k=0}^{\infty} ar^k = a + ar + ar^2 + ar^3 + \dots\] 

    If $r \geq 1$ and $a \neq 0$, then the series diverges. If $r \neq 1$, we use the identity 
    \[(1 - r)(1 + r + r^2 + r^3 + \dots + r^{m-1}) = 1 - r^m\]

    Then for partial sums  
    \[S_m = a + ar + ar^2 + \dots + ar^{m-1} = a(1 + r + r^2 + \dots + r^{m-1}) = a\frac{1 - r^m}{1 - r}\]

    If $\abs{r} < 1$, $a \frac{1 - r^m}{1 - r} \to \frac{a}{1 - r}$. Therefore, for $\abs{r} < 1$, 
    \[\sum_{k=0}^{\infty} ar^k = \frac{a}{1-r}\]

    \begin{tbox}{\textbf{Theorem (Absolute Convergence Test):} If the series $\sum_{n=1}^{\infty} \abs{a_n}$ converges, then $\sum_{n=1}^{\infty} a_n$ converges.}
        \emph{Proof:} Since $\sum_{n=1}^{\infty} \abs{a_n}$ converges, by Cauchy Criterion, given $\ep > 0$, $\exists N \in \N$ such that for all $n > m \geq N$,
        \[\abs{a_{m+1}} + \abs{a_{m+1}} + \dots + \abs{a_n} < \ep\]

        By triangle inequality, 
        \[\abs{a_{m+1} + a_{m+2} + \dots + a_n} \leq \abs{a_{m+1}} + \abs{a_{m+2}} + \dots + \abs{a_n} < \ep\]
    \end{tbox}

    \textbf{Remark:} The converse is not true! Consider the alternating harmonic series: 
    \[\sum_{n=1}^{\infty} \frac{(-1)^{n+1}}{n} \text{ converges}, \quad \sum_{n=1}^{\infty} \abs{\frac{(-1)^{n+1}}{n}} = \sum_{n=1}^{\infty} \frac{1}{n} \text{ diverges}\]

    \begin{tbox}{\textbf{Theorem (Alternating Series Test):} Let $(a_n)$ be a sequence satisfying
        \begin{enumerate}[label=(\alph*)]
            \item $a_1 \geq a_2 \geq \dots \geq a_n \geq a_{n+1} \geq \dots$ (Decreasing)
            \item $(a_n) \to 0$ (Converges to 0)
        \end{enumerate}
        Then the alternating series $\sum_{n=1}^{\infty} (-1)^{n+1}a_n$ converges.}

        \emph{Proof:} From conditions (i) and (ii), we have that $a_n \geq 0$. We want to show that the sequence of partial sums $(S_n)$ converges by showing that $(S_n)$ is Cauchy. Let $\ep > 0$ be arbitrary. We need to find an $N$ such that $n > m \geq N$ implies $\abs{S_n - S_m} < \ep$. 
        \[\abs{S_n - S_m} = \abs{a_{m+1} - a_{m+2} + a_{m+3} - \dots \pm a_n}\] 

        Since $(a_n)$ is decreasing and all the terms are positive, we can use an induction argument to show $\abs{S_n - S_m} \leq \abs{a_{m+1}}$ for all $n > m$. 
        
        Sketch: 
        \[\abs{a_{m+3}} \leq \abs{a_{m+2}} \leq \abs{a_{m+1}} \implies a_{m+1} - a_{m+2} + a_{m+3} \leq a_{m+1}\]

        Since $(a_n) \to 0$, we can choose $N$ such that $m \geq N$ implies $\abs{a_m} < \ep$. Then 
        \[\abs{S_n - S_m} \leq \abs{a_{m+1}} < \ep\]
        Therefore, $(S_n)$ is Cauchy so it converges $\qed$
    \end{tbox}

    \textbf{Definition:} 
    \begin{itemize}
        \item If $\sum_{n=1}^{\infty} \abs{a_n}$ converges, then $\sum_{n=1}^{\infty} a_n$ \emph{converges absolutely}
        \item If $\sum_{n=1}^{\infty} a_n$ converges but $\sum_{n=1}^{\infty} \abs{a_n}$ diverges, then $\sum_{n=1}^{\infty} a_n$ \emph{converges conditionally}
    \end{itemize}

    \textbf{Definition:} Let $\sum_{n=1}^{\infty} a_n$ be a series. A series $\sum_{n=1}^{\infty} b_n$ is called a rearrangement of the original series if there exists $f:\begin{tikzcd}
        \N \arrow[hook, two heads]{r} &[-10pt] \N
    \end{tikzcd}$ such that $b_{f(n)} = a_n$ for all $n \in \N$. 

    \emph{Note:} the bijectivity means that every term eventually appears and there are no repetitions.

    \begin{tbox}{\textbf{Theorem:} If $\sum_{n=1}^{\infty} a_n$ converges absolutely, then every rearrangement of the series converges to the same limit.}
        \emph{Proof:} Omitted
    \end{tbox}

\chapter{Basic Topology on $\R$}
\section*{March 05:}
    \textbf{Recall:} an \emph{$\ep$-neighborhood} of a point $x \in \R$ is the set
    \[V_{\ep}(a) = \{x \in \R: \abs{x - a} < \ep\}\]

    \textbf{Definition:} A set $O \subseteq \R$ is \emph{open} if for all points $a \in O$, there exists an $\ep$-neighborhood of $a$ such that $V_{\ep}(a) \subseteq O$.

    \textbf{Examples:}
    \begin{itemize}
        \item $\R$ is open
        \item $\emptyset$ is open
        \item $(c, d) = \{x \in \R: c < x < d\}$ is open (\emph{Proof:} Let $x \in (c, d)$. Then $V_{\min\{x-c, d-x\}}(x) \subseteq (c, d)$)
    \end{itemize}

    \begin{tbox}{\textbf{Theorem:} 
        \begin{enumerate}
            \item The union of an arbitrary collection of open sets is open
            \item The intersection of a finite collection of open sets is open
        \end{enumerate}}
        \emph{Proof:} 
        \begin{enumerate}
            \item Let $\{O_{\lambda}: \lambda \in \Lambda\}$ be a collection of open sets. 
            
            Let $O = \bigcup_{\lambda \in \Lambda} O_{\lambda}$. We need an $\ep$-neighborhood of an arbitrary $a \in O$ to be completely contained in $O$. 

            Notice that $a \in O \implies a \in O_{\lambda'}$ for some $\lambda' \in \Lambda$. Since $O_{\lambda'}$ is open, $\exists \ep > 0$ such that $V_{\ep}(a) \subseteq O_{\lambda'} \subseteq O$. 

            \item Let $\{O_1, O_2, \dots, O_n\}$ be a finite collection of open sets. Denote $O = \bigcap_{k=1}^n O_k$. We need to show that $O$ is open.
            
            Let $a \in O$. Then $a \in O_k$ for all $k = 1, 2, \dots, n$. Since $O_k$ is open, $\exists \ep_k > 0$ such that $V_{\ep_k}(a) \subseteq O_k$ for all $k$. 

            Now, we have different $\ep$-neighborhoods in each $O_k$. We want an $\ep$-neighborhood which is contained in \emph{every} $O_k$. 

            Let $\ep = \min\{\ep_1, \dots, \ep_n\}$. Then $V_{\ep}(a) \subseteq O_k$ for all $k = 1, 2, \dots, n$. Therefore, $V_{\ep}(a) \subseteq \bigcap_{k=1}^n O_k$. $\qed$
        \end{enumerate}
    \end{tbox}

    \textbf{Definition:} A point $x$ is a \emph{limit point} (cluster point/accumulation point) of a set $A$ if every $\ep$-neighborhood of $x$ intersects $A$ at some point other than $x$. 

    \begin{tbox}{\textbf{Theorem:} A point $x$ is a limit point of a set $A$ iff there exists a sequence $(a_n)$ in $A$ such that $(a_n) \to x$ and $a_n \neq x$ for all $n \in \N$}
        \emph{Proof:} 
        
        Assume $x$ is a limit point of $A$. We need a sequence $(a_n)$ in $A$ such that $(a_n) \to x$. By definition, every $\ep$-neighborhood of $x$ intersects $A$ at some point other than $x$. Pick $\ep = \frac{1}{n}$. Then for all $n \in \N$, pick 
        \[a_n \in V_{1/n}(x) \cap A, \quad a_n \neq x\]

        Now we want $(a_n) \to x$. Given $\ep > 0$ choose $N$ such that $\frac{1}{N} < \ep$ so $\abs{a_n  - x} < \ep$ for all $n \in N$

        Now, suppose there exists a sequence $(a_n)$ in $A$ such that $(a_n) \to x$ and $a_n \neq x$ for all $n \in \N$. We need to show that $x$ is a limit point of $A$.

        Let $V_{\ep}(x)$ be an arbitrary $\ep$-neighborhood. By definition of convergence, $\exists N \in \N$ such that for all $n \geq N$, $\abs{a_n - x} < \ep$. Then $a_n \in V_{\ep}(x)$ for all $n \geq N$. $\qed$
    \end{tbox}

    \textbf{Definition:} A point $a \in A$ is an isolated point of $A$ if it is \emph{not} a limit point of $A$ 

    \textbf{Note:} An isolated point is \emph{always} a point in the set. A limit point does not necessarily belong to the set. 

    \textbf{Definition:} a set $F \subseteq \R$ is closed if it contains its limit points.

    \begin{tbox}{\textbf{Theorem:} A set $F \subseteq \R$ is closed iff every Cauchy sequence contained in $F$ has a limit in $F$}
        \emph{Proof:} HW
    \end{tbox}

    \textbf{Example:} Let $A = \{\frac{1}{n}: n \in \N\}$. Show each point in $A$ is isolated. 
    
    Given $\frac{1}{n} \in A$, choose $\ep = \frac{1}{n} - \frac{1}{n+1}$. Therefore, $V_{\ep}(\frac{1}{n}) \cap A = \{\frac{1}{n}\}$ so $\frac{1}{n}$ is an isolated point and not a limit point. 

    Further, the limit of $A$ is $0$. Therefore, $\forall \ep > 0$, $V_{\ep}(0)$ contains points in $A$. Since $0 \notin A$, $A$ is not closed. 

    However, we can create a closed set $F = A \cup \{0\}$. This is the \emph{closure} of $A$. 

    \textbf{Example:} Show $[c, d] = \{x \in \R: c \leq x \leq d\}$ is closed. 
    
    If $x$ is a limit point, then $\exists (x_n) \in [c, d]$ with $(x_n) \to x$. We want to show that $x \in [c, d]$. Since $c\leq x_n \leq d$, by the Order Limit Theorem, 
    \[c \leq \lim x_n \leq d \implies \lim x_n \in [c, d] \implies x \in [c, d]\]
    so the set is closed. 

    \textbf{Example:} $\Q \subseteq \R$. The set of all limit point in $\Q$ is $\R$. 

    \emph{Proof:} Let $y \in \R$. Consider any neighborhood $V_{\ep}(y) = (y - \ep, y + \ep)$. From the density of $\Q$ in $\R$, $\exists r \neq y$ such that $y - \ep < r < y + \ep$. Therefore, $r \in V_{\ep}(y)$ so $y$ is a limit point of $\Q$. 

\section{March 7:} 
    \textbf{Definition:} given a set $A \subseteq \R$, let $L$ be the set of all limit points of $A$. The \emph{closure} of $A$ is the set $\overline{A} = A \cup L$.

    \textbf{Example:}
    \begin{itemize}
        \item  $\overline{\Q} = \R$
        \item $A = (a, b) \implies \overline A = [a, b]$ 
        \item If $A$ is closed, $\overline A = A$ 
    \end{itemize}

    \begin{tbox}{\textbf{Theorem:} For any $A \subseteq \R$, the closure $\bar A$ is a closed set and it is the smallest closed set containing $A$}
        \emph{Proof:} Let $L$ be the set of limit points of $A$. Then $\bar A = A \cup L$ is closed (it contains all its limit points, obviously). Any closed set containing $A$ must contain $L$. Therefore $\bar A$ is the smallest closed set containing $A$. $\qed$
    \end{tbox}

    \textbf{Complement:} Recall that $A^c = \{x \in \R: x \notin A\}$ 

    \begin{tbox}{\textbf{Theorem:} 
        \begin{enumerate}
            \item A set $O$ is open $\iff O^c$ is closed
            \item A set $F$ is closed $\iff F^c$ is open
        \end{enumerate}}
        \emph{Proof:} 
        \begin{enumerate}
            \item Let $O \subseteq \R$ be open. We want to show $O^c$ is closed. By definition, if $x$ is a limit point of $O^c$, then every $\ep$-neighborhood of $x$ contains some point of $O^c$. Thus, any $\ep$-neighborhood of $x$ cannot be a subset of $O$ so $x \notin O$. Since $x \in O^c$, $O^c$ is closed.
            
            Now assume $O^c$ is closed. We want to show that $O$ is open, i.e. for any $x \in O$, $\exists V_{\ep}(x) \subseteq O$. By definition, $O^c$ is closed so $x$ is not a limit point of $O^c$. Therefore, $\exists V_{\ep}(x)$ which does not intersect $O^c$. Then $V_{\ep}(x) \subseteq O$.

            \item $(E^c)^c = E$. The rest of the proof follows from 1). 
        \end{enumerate}
    \end{tbox}

    \begin{tbox}{\textbf{Theorem:} 
        \begin{enumerate}
            \item The union of a finite collection of closed sets is closed
            \item The intersection of an arbitary collection of closed sets is closed
        \end{enumerate}}
        \emph{Proof:} Follows from previous theorem and de Morgan's laws: 
        \[\left(\bigcup_{\lambda \in \Lambda} E_{\lambda}\right)^c = \bigcap_{\lambda \in \Lambda} E_{\lambda}^c, \quad \left(\bigcap_{\lambda \in \Lambda} E_{\lambda}\right)^c = \bigcup_{\lambda \in \Lambda} E_{\lambda}^c\]
    \end{tbox}

\subsection*{Compact Sets}
    \textbf{Motivation:} Bring ``finite'' quality to infinite arguments. 

    \textbf{Definition:} A set $K \subseteq \R$ is \emph{compact} if every sequence in $K$ has a convergent subsequence whose limit is in $K$.

    \textbf{Example:} $[c, d]$ is compact. \emph{Proof:} if $(a_n) \in [c, d]$, then it is bounded so by Bolzano-Weierstrass, $\exists (a_{n_k})$ which converges to $a$. Further $a \in [c, d]$ since $[c, d]$ is closed.

    \textbf{Definition:} A set $A \subseteq \R$ is bounded if $\exists M > 0$ such that $\abs{a} < M$ for all $a \in A$. 

    \begin{tbox}{\textbf{Theorem (Characterization of compactness in $\R$):} A set $K \subseteq \R$ is compact iff it is closed and bounded}
        \emph{Proof:} Assume $K$ is compact. Suppose $K$ is not bounded. Since $K$ is not bounded: 
        \[\forall n \in \N: \quad \exists x_n \in K, \st \abs{x_n} > n\]

        Since $K$ is compact, $(x_n)$ should have a convergent subsequence. However, $(x_n)$ is unbounded so $(x_{n_k})$ is unbounded. Therefore, there is no convergent subsequence in $(x_n)$. This is a contradiction of compactness so $K$ is bounded.

        Now we want to show $K$ is closed. Let $x = \lim x_n$ with $(x_n) \in K$. It suffices to show $x \in K$. By definition, $K$ is compact so $(x_n)$ has a convergent subsequence $(x_{n_k})$ which converges to $x$ and lies in $K$. $(x_{n_k}) \to x \implies x \in K \implies$ K is closed. 

        It remains to prove that $K$ is compact if it is closed and bounded. This is left for HW. 
    \end{tbox}

    \begin{tbox}{\textbf{Theorem (Nested Compact Set Property):} If $K_1 \supseteq K_2 \supseteq K_3 \supseteq \dots$ is a nested sequence of nonempty compact sets, then $\bigcap_{n=1}^{\infty} K_n \neq \emptyset$}
        \emph{Proof:} Use compactness of $K_n$ to produce a sequence that belongs to each set. $\forall n \in \N$, pick $x_n \in K_n$. Therefore, $(x_n) \in K_1 \implies \exists (x_{n_k}) \in K_1$ with $\lim x_{n_k} = x \in K_1$. 

        Given an $n_0 \in \N$, the terms of $(x_n)$ are contained in $K_{n_0}$ as long as $n > n_0$. We now ignore the finite number of terms for which $n_k < n_0$. Therefore, $(x_{n_k}) \in K_{n_0}$ so $\lim x_{n_k} = x \in K_{n_0}$. Since $n_0$ was arbitrary, 
        \[x \in \bigcap_{n=1}^{\infty} K_n\]
    \end{tbox}

\section*{March 12:}
    \textbf{Definition:} Let $A \subseteq \R$. An \emph{open cover} of $A$ is a (possibly infinite) collection of open sets $\{O_{\lambda}: \lambda \in \Lambda\}$ such that 
    \[A \subseteq \bigcup_{\lambda \in \Lambda} O_{\lambda}\]

    Given an open cover for $A$, a \emph{finite subcover} is a finite collection of open sets from the original open cover, whose union still contains $A$

    \textbf{Example:} Find an open cover for $(0, 1)$. 

    $\forall x \in (0, 1)$, let $O_x$ be the open interval $(\frac{x}{2}, 1)$ so we have the infinite collection 
    \[\{O_x: x \in (0, 1) \text{ covering } (0, 1)\}\]

    However, it is impossible to find a finite subcover for $(0, 1)$ using this open cover: Construct $\{O_{x_1}, O_{x_2}, \dots, O_{x_n}\}$ and set $x' = \min\{x_1, \dots, x_n\}$. But then any $y \in \R$ with $0 < y \leq \frac{x'}{2}$ is not in $\bigcup_{i=1}^{n} O_{x_i}$ 

    \textbf{Example:} Find an open cover for $[0, 1]$. 

    Naturally, we can use the same open cover as $(0, 1)$. However, this does not include the endpoints. Now let $\ep > 0$ and define $O_0 = \{-\ep, \ep\}, O_1 = (1 - \ep, 1 + \ep)$. Then
    \[\{O_0, O_1, O_x: x \in (0, 1)\}\] 
    is an open cover if $[0, 1]$. 

    To find a finite subcover, choose $x'$ such that $\frac{x'}{2} < \ep$:  
    \[\{O_0, O_1, O_{x'}\}\]

    \begin{tbox}{\textbf{Theorem (Heine-Borel):} For $K \subseteq \R$, then the following are equivalent: 
        \begin{enumerate}[label=(\roman*)]
            \item $K$ is compact
            \item $K$ is closed and bounded 
            \item Every open cover of $K$ has a finite subcover
        \end{enumerate} }
        \emph{Proof:} (i) $\iff$ (ii) follows from the Characterization of compactness in $\R$. 

        It suffices to show (ii) $\iff$ (iii): 

        Assume that every open cover of $K$ has a finite subcover. We want to show that $K$ is closed and bounded. Let $O_x = \{\abs{x - a} < 1: a \in \R\} = V_1(x)$. Since $\{O_x: x\in K\}$ must have finite subcover, $\exists x_1, x_2, \dots, x_n \in K$ such that $\{O_{x_1}, O_{x_2}, \dots, O_{x_n}\}$ is a finite subcover of $K$. 

        Since $K$ is contained in a finite collection of sets, it is bounded. 

        To show $K$ is closed, let $(y_n)$ be a Cauchy sequence is $K$ with $(y_n) \to y$. Suppose $y \notin K$, i.e. $\forall x \in K$, $x$ lies some positive distance away from $y$. 

        Construct an open cover by taking $O_x$ to be the interval of radius $\frac{\abs{x-y}}{2}$ around $x \in K$. By (iii), we have a finite subcover $\{O_{x_1}, O_{x_2}, \dots, O_{x_n}\}$.
    
        Let $\ep_0 = \min \left\{\frac{\abs{x_i - y}}{2}: 1 \leq i \leq n\right\}$. Since $(y_n) \to y$, $\exists y_N$ such that $\abs{y_N - y} < \ep_0$. 

        This means that $y_N$ must be excluded from each $O_x$ so certainly, $y \notin \bigcup_{i=1}^n O_{x_i}$. Therefore, this finite collection cannot be a subcover since it does not contain all of $K$. This is a contradiction so $K$ contains every limit point, and therefore $K$ is closed.
        
        The other direction, (ii) $\implies$ (iii), is left for homework. $\qed$
    \end{tbox}

\chapter{Functional Limits and Continuity}
\section*{March 12 (Continued)}
    \textbf{Definition (Functional limit):} Let $f: A \to \R$ be a function and let $c$ be a limit point of the domain $A$. We say $\lim f(x) = L$ if $x \to c$. 

    Then $\forall \ep > 0$, $\exists \delta > 0$, such that whenever $0 < \abs{x - c} < \delta$ and $x \in A$, we have $\abs{f(x) - L} < \ep$.

    \usetikzlibrary{math}
    \tikzmath{\c = 2.5; \L = 1.6; \d = 0.5; \e = 0.5; \cm = \c - \d; \cp = \c + \d; \lm = 1.5; \lp = 1.7;}
    \begin{center}
        \begin{tikzpicture}
            \begin{axis}[
                axis lines = left,
                domain=0:6,
                range=0:6,
                samples=100,
                xticklabel={\empty},
                yticklabel={\empty},
                extra x ticks = {\cm, \cp}, 
                extra x tick labels = {$c - \delta$, $c + \delta$},
                extra y ticks = {\lm, \lp}, 
                extra y tick labels = {$L - \ep$, $L + \ep$}
                ]
                \draw[red!50, fill=red!50] (axis cs: \cm, \lm) -- (axis cs: \cp, \lm) -- (axis cs: \cp, \lp) -- (axis cs: \cm, \lp) -- cycle;

                \addplot[smooth, thick, blue]{sin(deg(x))+1};
                \node[circle,fill,inner sep=2pt] (p) at (axis cs:\c, \L) {};
                \node[above right] at (p) {$(c,L)$};
            
                \draw[dotted, red] (axis cs:\cm, 0) -- (axis cs: \cm, \lm);
                \draw[dotted, red] (axis cs: \cp, 0) -- (axis cs: \cp, \lm);
                
                \draw[dotted, red] (axis cs: 0, \lp) -- (axis cs: \cm, \lp);
                \draw[dotted, red] (axis cs: 0, \lm) -- (axis cs: \cm, \lm);
            \end{axis}
        \end{tikzpicture}
    \end{center}
    

    \textbf{Topological Definition:} Let $c$ be a limit point in $A$ of $f: A \to \R$. We say that 
    \[\lim_{x\to c} f(x) = L\]
    if $\forall V_{\ep}(L)$, there exists $V_{\delta}(c)$ such that $\forall x \in V_{\delta}(c)$, $f(x) \in V_{\ep}(L)$ 

    \textbf{Example:} Show $\lim_{x \to 2} f(x) = 7$ with $f(x) = 3x + 1$. 

    Let $\ep > 0$. We need to produce a $\delta > 0$ such that $0 < \abs{x - 2} < \delta$ implies $\abs{f(x) - 7} < \ep$. 
    \[\abs{f(x) - 7} = \abs{3x + 1 - 7} = \abs{3x - 6} = 3\abs{x - 2}\]

    Choose $\delta = \frac{\ep}{3}$ so $0 < \abs{x - 2} < \delta \implies \abs{f(x) - 7} < 3\delta = \ep$.

    \textbf{Example:} Show $\lim_{x\to 2} g(x) = 4$, $g(x) = x^2$. 

    Let $\ep > 0$. We want $\abs{g(x) - 4} < \ep$ by restricting $\abs{x - 2} < \delta$. 

    Notice
    \[\abs{g(x) - 4} = \abs{x^2 - 4} = \abs{x + 2}\abs{x - 2}\] 

    So we construct a $\delta$-neighborhood around $c = 2$ with radius no bigger than $\delta = 1$: 
    \[\abs{x + 2} \leq \abs{3 + 2} = 5\] 

    Choose $\delta = \min\{1, \frac{\ep}{5}\}$. Then when $0 < \abs{x - 2} < \delta$, we have 
    \[\abs{g(x) -4} < \ep\]

\section{March 19:}
    \begin{tbox}{\textbf{Theorem (Sequential Criterion for Functional Limits):} Given $f: A \to \R$ and $c$ is a limit point of $A$, then the following are equivalent:
        \begin{enumerate}
            \item $\lim_{x\to c} f(x) = L$
            \item For every sequence $(x_n)$ in $A$ with $(x_n) \to c$ and $x_n \neq c$, we have $f(x_n) \to L$
        \end{enumerate} }
        \emph{Proof:} 

        Assume $\lim_{x\to c} f(x) = L$. Let $(x_n)$ be a sequence in $A$ with $(x_n) \to c$ and $x_n \neq c$. We want to show that $\forall \ep >0$, $\exists V_{\delta}(c)$ such that $\forall x \in V_{\delta}(c)$, $f(x) \in V_{\ep}(L)$. 

        We assume that $(x_n) \to c \implies \exists N \in \N$ such that $x_n \in V_{delta}(c)$ for all $n \geq N$. Then for all $n \geq N$, $f(x_n) \in V_{\ep}(L)$.
        
        For the other direction, we argue the contrapositive statement: 
        \[\lim_{x \to c} f(x) \neq L \implies \exists \ep_0 \st \forall \delta > 0, \exists x \in V_{\delta}(c) \st f(x) \notin V_{\ep_0}(L)\]

        Consider $\delta_n = \frac{1}{n}$. Then $\exists x_n \in V_{\delta_n}(c)$ such that $f(x_n) \notin V_{\ep_0}(L)$. Then $(x_n) \to c$ but $f(x_n) \nrightarrow L$. $\qed$
    \end{tbox}

    \begin{tbox}{\textbf{Corollary (ALT for Functional Limits):} Let $f$ and $g$ be functions defined on a domain $A \subseteq \R$ and assume $\lim_{x \to c} f(x) = L$ and $\lim_{x \to c} g(x) = M$. Then 
        \begin{enumerate}
            \item $\lim_{x \to c} kf(x) = kL$ for any $k \in \R$
            \item $\lim_{x \to c} (f(x) + g(x)) = L + M$
            \item $\lim_{x \to c} (f(x)g(x)) = LM$
            \item $\lim_{x \to c} \frac{f(x)}{g(x)} = \frac{L}{M}$, provided $M \neq 0$
        \end{enumerate}  }
        \emph{Proof:} Direct consequence of ALT for sequences and Sequential Criterion for Functional Limits.
    \end{tbox}

    \begin{tbox}{\textbf{Corollary (Divergence Criterion):} If $f: A \to \R$ with $c$ a limit point of $f$, if $\exists (x_n) \to c$ and $(y_n)\to c \in A$ but $\lim_{x_n \to c} f(x_n) \neq \lim_{y_n \to c} f(y_n)$, then $\lim_{x \to c} f(x)$ does not exist. }
        \emph{Proof:} Omitted
    \end{tbox}

    \textbf{Example:} To show $\lim_{x \to 0} \sin(\frac{1}{x})$ does not exist, consider the sequences $(x_n) = \frac{1}{2n\pi}$ and $(y_n) = \frac{1}{2n\pi + \frac{\pi}{2}}$.

    Clearly, $\lim_{n\to \infty} x_n = \lim_{n\to\infty} y_n = 0$. However, $\sin(\frac{1}{x_n}) = 0$ and $\sin(\frac{1}{y_n}) = 1$ so $\lim_{x\to 0} \sin(\frac{1}{x})$ does not exist.

    \textbf{Definition:} A function $f: A \to \R$ is \emph{continuous} at a point $c \in A$ if $\forall \ep > 0$, $\exists \delta > 0$ such that for $x \in V_{\delta}(c)$ (and $x \in A$), it follows that 
    \[\abs{f(x) - f(c)} < \ep\]

    If $f$ is continuous at every point in $A$, then we say $f$ is \emph{continuous} on $A$. 

    \begin{tbox}{\textbf{Theorem (Characterization of continuity):} Let $f: A \to \R$ and $c \in A$. The following definition of continuity of $f$ at $c$ are equivalent:
        \begin{enumerate}
            \item $\forall \ep > 0$, $\exists \delta > 0$, such that $\abs{x - c} < \delta \implies \abs{f(x) - f(c)} <\ep$ 
            \item $\forall V_{\ep}(f(c)), \exists \delta > 0$ such that $x \in V_{\delta}(c) \implies f(x) \in V_{\ep}(f(c))$
            \item If for $x_n \in A$ we have $(x_n) \to c$, then $f(x_n) \to f(c)$
        \end{enumerate}}
        \emph{Proof:} 
    \end{tbox}

    Equivalently, if $c$ is a limit point of $A$, then $f$ is continuous at $c$ if $\lim_{x\to c} f(x) = f(c)$.

    \begin{tbox}{\textbf{Corollary (Criterion for discontinuity):} $f: A \to \R$, $c \in A$ be a limit point of $A$. If $\exists (x_n) \in A$ with $(x_n) \to c$ but with $f(x_n) \nrightarrow f(c)$, then $f$ is not continuous at $c$. }
        \emph{Proof:} Direct from Characterization of continuity
    \end{tbox}

    \begin{tbox}{\textbf{Algebraic Continuity Theorem:} Assume $f: A \to \R$, $g: A \to \R$ are continuous at $c \in A$. Then 
        \begin{enumerate}
            \item $kf(x)$ is continuous at $c$ for any $k \in \R$
            \item $f(x) + g(x)$ is continuous at $c$ for any continuous functions $f$ and $g$
            \item $f(x)g(x)$ is continuous at $c$ 
            \item $\frac{f(x)}{g(x)}$ is continuous at $c$ provided $g(c) \neq 0$
        \end{enumerate}}
        \emph{Proof:} Direct from sequential criterion and sequences
    \end{tbox}

    \textbf{Example:} All polynomials (and in fact all rational functions) are continuous. Consider 
    \[g(x) = x \implies \abs{g(x) - g(c)} = \abs{x - c}\] 
    so $\forall \ep > 0$, pick $\delta = \ep$ and then $\abs{x - c} < \delta \implies \abs{g(x) - g(c)} < \ep$ so $g(x) = x$ is continuous. 

    Now consider $f(x) = k$. Clearly, with $\ep > 0$ and $\delta =1$, $k$ is continuous. 

    Notice the general form of a polynomial: 
    \[p(x) = a_0 + a_1 x + a_2 x^2 + \dots + a_n x^n\]
    Each term is continuous by the Algebraic Continuity Theorem so the sum is continuous.

    \textbf{Example:} Consider 
    \[g(x) = \begin{cases}
        x \sin(\frac{1}{x}) & x \neq 0 \\
        0 & x = 0
    \end{cases}\]
    Then we can estimate 
    \[\abs{g(x) - g(0)} = \abs{x \sin(\frac{1}{x}) - 0} \leq x\]
    since $\abs{\sin x} \leq 1$. 

    Then, $\forall \ep > 0$, set $\delta = \ep$ so whenever $\abs{x - 0} = \abs{x} < \delta$, 
    \[\abs{g(x) - g(0)} < \ep \implies g \text{ is continuous at 0}\]

\section{March 21:} 
    \textbf{Example:} $f(x) = \sqrt{x}$ on $A = \{x \in \R: x \geq 0\}$. Show that $f(x)$ is continuous in $A$. 

    Let $\ep > 0$. We need to show that 
    \[\abs{f(x) - f(c)} < \ep\]
    for all $x$ in some $V_{\delta}(c)$.

    \emph{Case 1 ($c = 0$):} 
    \[\abs{f(x) - \sqrt{0}} = \sqrt{x} < \ep \implies x < \ep^2\]

    Choosing $\delta = \ep^2$, we have $\abs{x - 0} < \delta \implies \abs{f(x) - f(0)} < \ep$.

    \emph{Case 2 ($c > 0$):}
    \[\abs{f(x) - f(c)} = \abs{\sqrt{x} - \sqrt{c}} = \frac{\abs{x - c}}{\sqrt{x} + \sqrt{c}} \leq \frac{\abs{x - c}}{\sqrt{c}}\]

    Choose $\delta = \sqrt{c} \cdot \ep$. Then 
    \[\abs{x - c} < \delta \implies \abs{\sqrt{x} - \sqrt{c}} < \frac{\ep\sqrt{c}}{\sqrt{c}} = \ep\]

    \begin{tbox}{\textbf{Theorem (Composition of Continuous Functions):} Given $f: A \to \R$ and $g: B \to \R$ with $f(A) \subseteq B$, if $f$ is continuous at $c \in A$ and $g$ is continuous at $f(c) \in B$, then $g \circ f$ is continuous at $c$. }
        \emph{Proof:} HW
    \end{tbox}

    \subsection*{Continuous Functions on Compact Sets} 
        Let $f: A \to \R$ with $B \subseteq A$. We say $f(B) = \{f(x): x \in B\}$ is the \emph{image} of $B$ under $f$.

        \textbf{Question:} does a continuous function map open sets to open sets? 

        No! $f(x) = x^2$ maps $(-1, 1)$ to $[0, 1)$ which is not open. In fact, this property is also not true for closed sets: $g(x)  \frac{1}{1 + x^2}$ maps $[0, \infty)$ to $(0, 1)$ which is not closed. 

        This leads to another natural question: is there any property which is preserved under continuous maps? 

        \begin{tbox}{\textbf{Theorem (Preservation of Compact Sets):} Let $f: A \to \R$ be continuous on $A$. If $K \subseteq A$ is compact, then $f(K)$ is compact.}
            \emph{Proof:} Let $(y_n) \in f(K)$. It suffices to find a subsequence $(y_{n_k})$ which converges to a limit contained in $f(K)$. 

            Note that $(y_n) \in f(K) \implies \forall n \in \N$, $\exists x_n \in K$ such that $f(x_n) = y_n$. 
            
            Since $K$ is compact, $\exists (x_{n_k})$ such that $x_{n_k} \to x \in K$.

            Since $f$ is continuous on $A$, clearly $f$ is continuous at $x$. Therefore,
            \[(x_{n_k}) \to x \implies y_{n_k} = f(x_{n_k}) \to f(x) \in f(K)\]
        \end{tbox}

        \begin{tbox}{\textbf{Extreme Value Theorem:} If $f: K \to \R$ is continuous on a compact set $K \subseteq \R$, then $f$ attain a maximum and minimum value, i.e. $\exists x_0, x_1 \in K$ such that $f(x_0) \leq f(x) \leq f(x_1)$ for all $x \in K$. }
            \emph{Proof:} Since $f(K)$ is compact, we can set $\alpha = \sup f(K)$ and we know $\alpha \in f(K)$. Therefore, $\exists x_1 \in K$ such that $f(x_1) = \alpha$ and we call it the maximum.

            The minimum follows by similar argument
        \end{tbox}

    \subsection*{Uniform Continuity}
        \textbf{Example:} 
        \begin{enumerate}
            \item $f(x) = 3x + 1$ is continuous at $c \in \R$ so 
            \[\abs{f(x) - f(c)} = \abs{3x + 1 - 3c - 1} = 3\abs{x - c}\] 
            Then, given $\ep > 0$, choose $\delta = \frac{\ep}{3}$ so $\abs{x - c} < \delta \implies \abs{f(x) - f(c)} < \ep$.

            \item $g(x) = x^2$. Given $c \in \R$, 
            \[\abs{g(x) - g(c)} = \abs{x^2 - c^2} = \abs{x - c}\abs{x + c}\]
            we need and upper bound so choose $\delta \leq 1$ which bounds $x \in (c - 1, c  1)$. 

            Then, 
            \[\abs{x + c} \leq \abs{x} + \abs{c} \leq (\abs{c} + 1) + \abs{c}= 2\abs{c} +1\]
            Let $\ep > 0$, choose $\delta = \min\{1, \frac{\ep}{2\abs{c} + 1}\}$. Finally, 
            \[\abs{x - c} < \delta \implies \abs{f(x) - f(c)} < \ep\]
        \end{enumerate}

        \textbf{Conclusion:} In example 1, we chose an arbitrary $c$. In example 2, our value of $\delta$ depended on $c$. 
 
        \textbf{Definition:} A function $f: A \to \R$ is \emph{uniformly continuous} on $A$ if $\forall \ep > 0$, $\exists \delta > 0$ such that $\forall x, y \in A$,
        \[\abs{x - y} < \delta \implies \abs{f(x) - f(c)} < \ep\] 

        \textbf{Remark:} Continuity requires only that for each $c$ there exists at least one $\delta > 0$ such that $\abs{x - c} < \delta \implies \abs{f(x) - f(c)} < \ep$.

        Uniform continuity requires that a single $\delta > 0$ works for all $c \in A$. 

        \begin{tbox}{\textbf{Seqential Criterion for Absence of Uniform Continuity:} $f: A \to \R$ fails to be uniformly continuous iff $\exists \ep_0 > 0$ and $(x_n), (y_n) \in A$ such that $\forall \delta > 0$, 
            \[\abs{x - y} \to 0 \text{ but } \abs{f(x) - f(y)} \geq \ep_0\]}
            \emph{Proof:} Suppose that $f$ fails to be uniformly continuous. Then, by definition, $\exists \ep_0 > 0$ such that $\forall \delta > 0$, $\exists x, y \in A$ such that $\abs{x - y} < \delta$ but $\abs{f(x) - f(y)} \geq \ep_0$.
            
            Then we can construct $(x_n)$ and $(y_n)$ by choosing $\delta_n = \frac{1}{n}$ ($n \in \N$) so that $\exists x_n, y_n$ with 
            \[\abs{x_n - y_n} < \frac{1}{n} \text{ but } \abs{f(x_n) - f(y_n)} \geq \ep_0\]

            For the other direction, suppose that $\exists \ep_0 > 0$ and $(x_n), (y_n) \in A$ such that $\abs{x_n - y_n} \to 0$ but $\abs{f(x_n) - f(y_n)} \geq \ep_0$. 

            Obviously, as $\abs{f(x_n) - f(y_n)} \geq \ep_0$, $f$ fails to be uniformly continuous. $\qed$
        \end{tbox}

\section{April 2:} 
    \textbf{Example of uniform continuity:} We showed that $h(x) = \sin(\frac{1}{x})$ is continuous on $(0, 1)$. However, it is not uniformly continuous: Take $\ep_0 = 2$ so 
    \[x_n = \frac{1}{\frac{\pi}{2} + 2n\pi}, \quad y_n = \frac{1}{\frac{3\pi}{2} + 2n\pi}\]
    As $n \to \infty$, $(x_n) \to 0$ and $(y_n) \to 0$ so $\abs{x_n - y_n} = 0$. However, 
    \[\abs{h(x_n) - h(y_n)} = 2\]
    so it is not uniformly continuous

    \begin{tbox}{\textbf{Theorem (Uniform Continuity on Compact Sets):} A function that is continuous on a compact set $K$ is uniformly continuous on $K$.}
        \emph{Proof (by Contradiction):} Assume $f: K \to \R$ is continuous on $K$. Suppose that $f$ is not uniformly continuous on $K$. 
        
        Then by the Criterion for Absence, $\exists \ep_0 > 0$ and $(x_n), (y_n) \in K$ such that $\abs{x_n - y_n} \to 0$ but $\abs{f(x_n) - f(y_n)} \geq \ep_0$.

        Since $K$ is compact, $(x_n)$ has a convergent subsequence $(x_{n_k})$ which converges to $x \in K$. We can then consider $(y_{n_k})$ consisting of the terms in $y_n$ that correspond to the terms in $(x_{n_k})$. We know these exist since $\abs{x_n - y_n} \to 0$.

        By the ALT, 
        \[\lim(y_{n_k}) = \lim((y_{n_k} - x_{n_k}) + x_{n_k}) = 0 + x \implies y_{n_k} \to x\]

        Since $f$ is continuous at $x$, we have $f(x_{n_k}) \to f(x)$ and $f(y_{n_k}) \to f(x)$. However, this implies 
        \[\lim (f(x_{n_k}) - f(y_{n_k})) = 0\] 
        but this contradicts the Criterion for Absence. Therefore, $f$ is uniformly continuous on $K$. $\qed$
    \end{tbox}

    \begin{tbox}{\textbf{Intermediate Value Theorem:} Let $f: [a, b] \to \R$ be continuous. If $L$ is a real number satisfying $f(a) < L < f(b)$ or $f(a) > L > f(b)$, then $\exists c \in (a, b)$ such that $f(c) = L$.}
        \emph{Proof:} HW
    \end{tbox}

\chapter{Derivatives}
\section*{April 2 (Continued):}

    \textbf{Definition:} Let $g: A \to \R$. Given $c \in A$, the \emph{derivative} of $g$ at $c$ is given 
    \[g'(c) = \lim_{x \to c} \frac{g(x) - g(c)}{x - c}\]
    provided the limit exists. In this case, we say $g$ is \emph{differentiable} at $c$.

    If $g$ is differentiable at every point in $A$, then we say $g$ is differentiable on $A$.

    \textbf{Example:}
    \begin{itemize}
        \item $f(x) = x^n$, $n \in \N$. 
        
        For $c \in \R$, we have 
        \[x^n - c^n = (x-c)(x^{n-1} + cx^{n-2} + c^2x^{n-3} + \dots + c^{n-1})\]
        so 
        \begin{align*}
            f'(c) &= \lim_{x \to c} \frac{x^n - c^n}{x - c}\\ 
            &= \lim_{x \to c} (x^{n-1} + cx^{n-2} + c^2x^{n-3} + \dots + c^{n-1})\\ 
            &= c^{n-1} + c^{n-1} + \dots + c^{n-1}\\ 
            &= nc^{n-1} 
        \end{align*} 

        \item $g(x) = \abs{x}$. 
        
        Attempting to calculate the derivative at $c = 0$, we have 
        \[g'(0) = \lim_{x \to 0} \frac{\abs{x}}{x} = \pm 1\]
        so the limit does not exists. 
    \end{itemize}

    \begin{tbox}{\textbf{Theorem:} If $g: A \to \R$ is differentiable at $c \in A$, then $g$ is continuous at $c$.}
        \emph{Proof:} Let $g$ be differentiable at $c$. Then 
        \[g'(c) = \lim_{x \to c} \frac{g(x) - g(c)}{x - c}\]
        exists. 

        It suffices to show $\lim_{x \to c} g(x) = g(c)$. By ALT, 
        \[\lim_{x \to c} g(x) - g(c) = \lim_{x \to c} \frac{g(x) - g(c)}{x - c} \cdot (x - c) = g'(c) \cdot 0 = 0\]
        so $g$ is continuous at $c$. $\qed$
    \end{tbox}

    \begin{tbox}{\textbf{Algebraic Differentiability Theorem:} Let $f, g$ be defined on an interval $A$ and assume both are differentiable at some point $c \in A$. Then:
        \begin{enumerate}
            \item $(f + g)'(c) = f'(c) + g'(c)$
            \item $(kf)'(c) = kf'(c) \quad \forall k \in \R$ 
            \item $(fg)'(c) = f'(c)g(c) + f(c)g'(c)$
            \item $\left(\frac{f}{g}\right)'(c) = \frac{f'(c)g(c) - f(c)g'(c)}{g^2(c)}$ provided $g(c) \neq 0$
        \end{enumerate} }
        \emph{Proof:} (1) and (2) come directly from the definition of the derivative and the ALT. 

        (3):
        \begin{align*}
            \frac{(fg)(x) - (fg)(c)}{x - c} &= \frac{f(x)g(c) - f(x)g(c) + f(x)g(c) - f(c)g(c)}{x - c}\\ 
            &= f(x) \frac{g(x) - g(c)}{x - c} + g(c) \frac{f(x) - f(c)}{x - c}
        \end{align*}

        Since $f$ is differentiable at $c$, it is continuous, i.e. $\lim_{x \to c} f(x) = f(c)$. Therefore, by the ALT, 
        \[\lim_{x \to c} \frac{(fg)(x) - (fg)(c)}{x - c} = f(c)g'(c) + f'(c)g(c)\]

        (4): Similar
    \end{tbox}

    \begin{tbox}{\textbf{Theorem (Chain Rule):} Let $f: A \to \R$ and $g: B \to \R$ with $f(A) \subseteq B$. If $f$ is differentiable at $c \in A$ and $g$ is differentiable at $f(c) \in B$, then $g \circ f$ is differentiable at $c$ and 
        \[(g \circ f)'(c) = g'(f(c)) \cdot f'(c)\]}
        \emph{Proof:} Since $g$ is differentiable at $f(c)$, 
        \[g'(f(c)) = \lim_{y \to f(c)} \frac{g(y) - g(f(c))}{y - f(c)}\]

        We can write 
        \begin{align*}
            d(y) &= \frac{g(y) - g(f(c))}{y - f(c)} \\ 
                &\implies \lim_{y \to f(c)} d(y) = g'(f(c))\\ 
                &\implies g(y) - g(f(c)) = d(y) (y - f(c))
        \end{align*}

        This new equation is defined $\forall y \in B$, including $g(c)$. Make the substitution $y = f(t)$ for some $t \in A$. 
        
        If $t \neq c$, 
        \[\frac{g(f(t)) - g(f(c))}{t - c} = d(f(t)) \frac{f(t) - f(c)}{t - c}\]

        Taking the limit as $t \to c$ and applying the ALT gives the result. $\qed$ 
    \end{tbox}

    \begin{tbox}{\textbf{Interior Extremum Theorem:} Let $f$ be differentiable on an open interval $(a, b)$. If $f$ attains a max value at some point $c \in (a, b)$ (i.e. $f(c) \geq f(x)$ forall $x \in (a, b)$), then $f'(c) = 0$.}
        \emph{Proof:} Since $c \in (a, b)$, construct $(x_n), (y_n)$ such that $(x_n) \to c$, $(y_n) \to c$, and $x_n < c < y_n$ forall $n \in \N$. 

        Since $f(c)$ is a max, $f(y_n) - f(c) \leq 0$ $\forall n \in \N$. So (by the Order Limit Theorem),
        \[f;(c) = \lim_{n \to \infty} \frac{f(y_n) - f(c)}{y_n - c} \leq 0\]

        Similarly, 
        \[\frac{f(x_n) - f(c)}{x_n - c} \geq 0 \implies f'(c) = \lim_{n \to \infty} \frac{f(x) - f(c)}{x - c}\geq 0\]

        Therefore, $f'(c) = 0$. $\qed$
    \end{tbox}

    \begin{tbox}{\textbf{Darboux's Theorem:} If $f$ is differentiable on an interval $[a, b]$ and if $\alpha$ satisfies $f'(a) < \alpha < f'(b)$ then $\exists c \in (a, b)$ where $f'(c) = \alpha$.}
        \emph{Proof:} HW
    \end{tbox}

    \subsection*{The Mean Value Theorem}
    \begin{tbox}{\textbf{Rolle's Theorem:} Let $f: [a, b] \to \R$ be continuous on $[a, b]$ and differentiable on $(a, b)$. If $f(a) = f(b)$, then $\exists c \in (a, b)$ such that $f'(c) = 0$.}
        \emph{Proof:} Since $f$ is continuous on a compact set, $f$ attains extrema on that set. 
        
        If both the max and min occur at the endpoints, then $f$ is constant since $f(a) = f(b)$. Trivially, $f'(c) = 0$ for all $c \in [a, b]$. 

        If either the max or the min occur some point $c \in (a, b)$, then by the Interior Extremum Theorem, $f'(c) = 0$. 
    \end{tbox}

    \begin{tbox}{\textbf{The Mean Value Theorem:} If $f: [a, b] \to \R$ is continuous on $[a, b]$ and differentiable on $(a, b)$, then $\exists c \in (a, b)$ such that 
        \[f'(c) = \frac{f(b) - f(a)}{b - a}\]} 

        \emph{Proof:} Notice that Rolle's Theorem is a special case of the MVT when $f(a) = f(b)$. We seek to reduce the general case to the special case. 
        
        The equation of the line through the points $(a, f(a))$ and $(b, f(b))$ is given by: 
        \[y = \frac{f(b) - f(a)}{b -a}(x - a) + f(a)\]

        Consider the difference between this line and the function $f(x)$: 
        \[d(x) = f(x) - \left[\frac{f(b) - f(a)}{b -a}(x - a) + f(a)\right]\] 

        Clearly, $d(x)$ is continuous on $[a, b]$. Further, it is differentiable on $(a, b)$ and $d(a) = d(b) = 0$. Therefore, we can apply Rolle's Theorem. 

        This gives that $\exists c \in (a, b)$ where $d'(c) = 0$. Since 
        \[d'(x) = f'(x) - \frac{f(b) - f(a)}{b- a} \implies 0 = f'(c) - \frac{f(b) - f(a)}{b - a} \qed\]
    \end{tbox}

    \begin{tbox}{\textbf{Corollary:} If $g: A \to \R$ is differentiable on an interval $A$ and satisfies $g'(x) = 0$ for all $x \in A$, then $g(x) = k$ with $k \in \R$.}
        \emph{Proof:} Take $x, y \in A$ and assume $x < y$. By MVT (applied to $g$ on $[x, y]$), $\exists c \in (x, y)$ such that 
        \[g'(c) =\frac{g(y) - g(x)}{y - x} = 0 \implies g(y) = g(x) = k\]

        Since $x, y$ were arbitrary, $g(x) =k \quad \forall x \in A$. $\qed$
    \end{tbox}

    \begin{tbox}{\textbf{Corollary:} If $f$ and $g$ are differentiable functions on an interval $A$ and satisfy $f'(x) = g'(x)$ $\forall x \in A$, then $f(x) = g(x) + k$ for $k \in \R$}
        \emph{Proof:} Let $h(x) = f(x) - g(x)$. $h'(x) = 0$ so by the previous corollary, $h(x) = k$. $\qed$
    \end{tbox}

    \begin{tbox}{\textbf{Theorem (Generalized MVT):} If $f$ and $g$ are continuous on the closed interval $[a, b]$ and differentiable on $(a, b)$, then $\exists c \in (a, b)$ where 
        \[[f(b) - f(a)]g'(c) = [g(b)- g(a)]f'(c)\]  
        If $g\neq 0$ on $(a, b)$, 
        \[\frac{f'(c)}{g'(c)} = \frac{f(b) - f(a)}{g(b) - g(a)}\]}
        \emph{Proof:} Follows from applying MVT to $h(x) = [f(b) - f(a)]g(x) - [g(b) - g(a)]f(x)$ 
    \end{tbox}

\section{April 9:}
    \begin{tbox}{\textbf{Theorem (L'Hopital's Rule - 0 case):} Let $f, g$ be continuous on an interval containing $a$ and assume $f, g$ are differentiable on this interval. If $f(a) = g(a) = 0$, and $g'(x) \neq 0$ for all $x \neq a$ then
        \[\lim_{x \to a} \frac{f'(x)}{g'(x)} = L \implies \lim_{x \to a} \frac{f(x)}{g(x)} = L\]}
        \emph{Proof:} Direct application of Generalized Mean Value Theorem
    \end{tbox}

    \textbf{Definition:} Give $g: A \to \R$ and a limit point $c \in A$, we say that 
    \[\lim_{x \to c} g(x) = \infty\]
    if $\forall M > 0$, $\exists \delta > 0$ such that 
    \[0 < \abs{x - c} < \delta \implies g(x) \geq M\]

    \begin{tbox}{\textbf{Theorem (L'Hopital's Rule - $\infty$ case):} Assume $f$ and $g$ are differentiable on $(a, b)$ and that $g'(x) \neq 0$ for all $x \in (a, b)$. If $\lim_{x \to a} g(x) = \infty$ (or $-\infty$), then 
        \[\lim_{x \to a} \frac{f'(x)}{g'(x)} = L \implies \lim_{x \to a} \frac{f(x)}{g(x)} = L\]}
        \emph{Proof:} Omitted (again largely by Generalized MVT)
    \end{tbox}

\chapter{Sequences and Series of Functions}
\section*{April 9 (Continued):} 
    \subsection*{Pointwise Convergence}
        \textbf{Definition:} For each $n \in \N$, let $f_n$ be defined on a set $A \subseteq \R$. The seuqnece $(f_n)$ of functions \emph{converges pointwise of A} to a function $f$ if $\forall x \in A$, the sequence of real numbers $f_n(x)$ converges to $f(x)$

        \emph{Notation:} to designate pointwise convergence, we can write 
        \begin{itemize}
            \item $f_n \to f$ 
            \item $\lim f_n = f$ 
            \item $\lim_{n \to \infty} f_n(x) = f(x)$
        \end{itemize}

        \textbf{Example:} Consider $f_n(x) = \frac{x^2 + nx}{n}$ on $\R$. 

        We compute 
        \[\lim_{n \to \infty} f_n(x) = \lim_{n \to \infty} \frac{x^2 + nx}{n} = \lim_{n \to \infty} \frac{x^2}{n} + x = x\] 

        \textbf{Example:} Let $g_n(x) = x^n$ on $[0, 1]$ 

        If $0 \leq x < 1$, then $x^n \to 0$ as $n \to \infty$. For $x = 1$, $x^n \to 1$ as $n \to \infty$. 

        Therefore, $g_n \to g$ pointwise on $[0, 1]$ where 
        \[g(x) = \begin{cases}
            0 & 0 \leq x < 1\\ 
            1 & x = 1
        \end{cases}\]

        \textbf{Remark:} This is a problem! A continuous sequence might not have a continuous limit. We will see more of this later. 

        \textbf{Example:} $h_n(x) = x^{1  \frac{1}{2n - 1}}$ on $[-1, 1]$. 

        For a fixed $x \in [-1, 1]$, we have 
        \[\lim_{n \to \infty} h_n(x) = x\cdot x^{\frac{1}{2n-1}} = x \lim_{n\to \infty} x^{\frac{1}{2n - 1}} = \abs{x}\]
        since $x^{\frac{1}{2n-1}} \to 1$ if $x > 0$ and $\to -1$ if $x < 0$. 

        \begin{tbox}{\textbf{Lemma (Failure Continuity of the Limit Function):}}
            \emph{Proof:} For $f$ to be continuous: fix $c \in A$ with $\ep > 0$. We need to find $\delta > 0$ such that 
            \[\abs{x - c} < \delta \implies \abs{f(x) - f(c)} < \delta\]

            Notice 
            \[\abs{f(x) - f(c)} \leq \abs{f(x) - f_n(x)} + \abs{f_n(x) - f_n(c)} + \abs{f_n(c) - f(c)}\]

            Choose $N \in \N$ such that $\abs{f_N(c) - f(c)} < \frac{\ep}{3}$. By continuity of $f_N$, 
            \[\abs{f_N(x) - f_N(c)} < \frac{\ep}{3}\]

            However, it might be true that $N$ is not large enough to ensure $\abs{f(x) - f_N(x)}$ converges. 
        \end{tbox}

        \textbf{Example:} Let $g_n(x) = x^n$ on $[0, 1]$ as above. Notice 
        \[\abs{g_n(\frac{1}{2}) - g(\frac{1}{2})} < \frac{1}{3} \implies n \geq 2\]
        but 
        \[\abs{g_n(\frac{9}{10}) - g(\frac{9}{10})}< \frac{1}{3} \implies n \geq 11\]

        For any chosen $n$, there are values of $x$ for which $\abs{g_n(x) - g(x)}$ might not be small enough. 
        
    \subsection*{Uniform Convergence}
        \textbf{Definition:} Let $(f_n)$ be a sequence of functions defined on $A \subseteq \R$. Then $f_n$ \emph{converges uniformly} on $A$ to a limit function $f$ defined on $A$, if $\forall \ep > 0$, $\exists N \in \N$ such that $\abs{f_n(x) - f(x)} < \ep$ whenever $n \geq N$ and $x \in A$.

        \textbf{Example:} $g_n = \frac{1}{n(1 + x^2)}$ 

        For any fixed $x \in \R$, $\lim_{n \to \infty} g_n(x) = 0 \implies g(x) = 0$ is the pointwise limit of $(g_n)$. 

        Do we have uniform convergence?

        Notice: 
        \[\frac{1}{1 + x^2} \leq 1 \quad \forall x \in \R \implies \abs{g_n(x) - g(x)} = \abs{\frac{1}{n(1 + x^2)} - 0} \leq \frac{1}{n}\]

        Given $\ep > 0$, choose $N > \frac{1}{\ep}$ so $n \geq N \implies \abs{g_n(x) - g(x)} < \ep$ for all $x \in \R$. 

        Since $N$ \emph{does not depend on $x$}, $g_n \to 0$ uniformly on $\R$. 

        \textbf{Example:} $f_n(x) = \frac{x^2 + nx}{n} \overset{p.w.}{\longrightarrow} f(x) = x$ but not uniformly! 
        \[abs{f_n(x)  f(x)} = \abs{\frac{x^2 + nx}{n} - x} = \frac{x^2}{n}\] 
        so $\abs{f_n(x) - f(x)} < \ep$ requires $N > \frac{x^2}{\ep}$ which depends on $x$. 
        
        \begin{tbox}{\textbf{Theorem (Cauchy Criterion for Uniform Convergence):} A sequence of functions $(f_n)$ defined on a set $A \subseteq \R$ converges uniformly on $A$ iff $\forall \ep > 0$, $\exists N \in \N$ such that $\abs{f_n(x) - f_m(x)} < \ep$ whenever $n, m \geq N$ and $x \in A$.}
            \emph{Proof:} HW
        \end{tbox}

        \begin{tbox}{\textbf{Theorem (Continuous Limit Theorem):} Let $(f_n)$ be a sequence of functions defined on $A \subseteq \R$ that converges uniformly on $A$ to a function $f$. If each $f_n$ is continuous at $c \in A$, then $f$ is continuous at $c$.}
            \emph{Proof:} Fix $c \in A$ and let $\ep > 0$. Choose $N \in \N$ such that 
            \[\abs{f_N(x) - f(x)} < \frac{\ep}{3}\]

            Since $f_N$ is continuous at $c$, $\exists \delta > 0$ for which 
            \[\abs{f_N(x) - f_N(c)} < \frac{\ep}{3}\]
            wheneber $\abs{x - c} < \delta$. 

            This implies:
            \begin{align*}
                \abs{f(x) - f(c)} &= \abs{f(x) - f_N(x) + f_N(x) - f_N(c) + f_N(c) - f(c)}\\
                &\leq \abs{f(x) - f_N(x)} + \abs{f_N(x) - f_N(c)} + \abs{f_N(c) - f(c)}\\
                &< \frac{\ep}{3} + \frac{\ep}{3} + \frac{\ep}{3} = \ep
            \end{align*}
        
            \textbf{Note:} we only have the convergence of the first term because of uniform continuity 
        \end{tbox}

\section{April 11:}
    \begin{tbox}{\textbf{Differentiable Limit Theorem:} Let $f_n \to f$ pointwise on the closed interval $[a, b]$ and assume that each $f_n$ is differentiable if $(f_n')$ converges uniformly on $[a, b]$ to a function $g$. Then $f$ is differentiable on $[a, b]$ and $f' = g$.}
        \emph{Proof:} We want to show that $f'(c)$ exists and equals $g(c)$. 

        Since 
        \[f'(c) = \lim_{x \to c} \frac{f(x) - f(c)}{x - c}\]
        it suffices to show that $\exists \delta > 0$ such that whenever $0 < \abs{x - c} < \delta$. 
        \[\abs{\frac{f(x) - f(c)}{x - c} - g(c)} < \ep\]

        By the triangle inequality, $\forall x\neq c$, 
        \begin{align*}
            \abs{\frac{f(x) - f(c)}{x - c} - g(c)} &\leq \abs{\frac{f(x) - f(c)}{x - c} - \frac{f_n(x) - f_n(c)}{x- c}}\\ 
            &\qquad + \abs{\frac{f_n(x) - f_n'(c)}{x - c} - f_n'(c)}\\ 
            &\qquad + \abs{f_n'(c) - g(c)}
        \end{align*}

        Start by choosing $N \in \N$ such that $\abs{f_m'(c) - g(c)} < \frac{\ep}{3}$ for all $m \geq N$.

        By uniform convergence of $(f_n')$, $\exists N_2$ such that $\forall m, n \geq N_2$, 
        \[\abs{f_m'(x) - f_n'(x)} < \frac{\ep}{3} \quad \forall x \in [a, b]\]

        Pick $N = \max\{N, N_2\}$. Since $f_n$ is differentiable at $c$, $\exists \delta > 0$ such that 
        \[0 < \abs{x - c} < \delta \implies \abs{\frac{f_N(x) - f_N(c)}{x- c} - f_N'(c)} < \frac{\ep}{3}\]

        Now all that remains is to bound the first term. Fix $x$ satisfying $0 < \abs{x - c} < \delta$ and let $m \geq n$. We can apply the MVT to $f_m - f_n$ on $[c, x]$ 

        By the MVT, $\exists \alpha \in [c, x]$ such that 
        \begin{align*}
            f_m'(\alpha) - f_N'(\alpha) &= \frac{(f_m(x) - f_N(x)) - (f_m(c) - f_N(c))}{x - c}\\ 
            &\implies \abs{\frac{f_m(x) - f_m(c)}{x - c}  -\frac{f_N(x) - f_N(c)}{x- c}} < \frac{\ep}{3}
        \end{align*}

        Since $f_m \to f$, take the limit as $m \to \infty$. By OLT, 
        \[\abs{\frac{f(x)- f(c)}{x  -c}  -\frac{f_N(x) - f_N(c)}{x - c}} < \frac{\ep}{3}\]

        Therefore, 
        \begin{align*}
            \abs{\frac{f(x) - f(c)}{x - c} - g(c)} &\leq \abs{\frac{f(x) - f(c)}{x - c} - \frac{f_n(x) - f_n(c)}{x- c}}\\ 
                &\qquad + \abs{\frac{f_n(x) - f_n'(c)}{x - c} - f_n'(c)}\\ 
                &\qquad + \abs{f_n'(c) - g(c)}\\ 
            &< \frac{\ep}{3} + \frac{\ep}{3} + \frac{\ep}{3} = \ep
        \end{align*}
    \end{tbox}

    \begin{tbox}{\textbf{Theorem:} Let $(f_n)$ be a sequence of differentiable functions defined on $[a, b]$ and assume that $(f_n')$ converges uniformly on $[a, b]$. If $\exists x_0 \in [a, b]$ where $f_n(x)$ is convergent, then $(f_n)$ converges uniformly on $[a, b]$.}
        \emph{Proof:} HW
    \end{tbox}

    \begin{tbox}{\textbf{Theorem:} Let $(f_n)$ be a sequence of differentiable functions defined on $[a, b]$ and assume $(f_n')$ converges uniformly to $g$ on $[a, b]$. If $\exists x_0 \in [a, b]$ for which $(f_n(x_0))$ converges, then $(f_n)$ converges uniformly and $f =\lim f_n$ is differentiable with $f' = g$.}
        \emph{Proof:} Follows from previous two theorems
    \end{tbox}

    \subsection*{Series of Functions} 
        \textbf{Definition:} For each $n \in \N$, let $f_n$ and $f$ be defined on a set $A \subseteq \R$. Then infinite series 
        \[\sum_{n=1}^{\infty} f_n(x) = f_1(x) + f_2(x) + \dots\]
        \emph{converges pointwise} on $A$ to $f(x)$ if the sequence $(S_k(x))$ of partial sums 
        \[S_k(x) = f_1(x) + \dots + f_k(x)\] 
        converges pointwise to $f(x)$. 

        Similarly, the series \emph{converges uniformly} on $A$ to $f(x)$ if the sequence of partial sums converges uniformly to $f(x)$.

        \begin{tbox}{\textbf{Term-by-term Continuity Theorem:} Let $f_n$ be continuous functions defined on $A \subseteq \R$ and assume $\sum_{n=1}^{\infty} f_n$ converges uniformly on $A$ to $f$. Then $f$ is continuous on $A$.}
            \emph{Proof:} Apply Continuous Limit Theorem to $(S_k)$ 
        \end{tbox}

        \begin{tbox}{\textbf{Term-by-term Differentiability Theorem:} Let $f_n$ be a sequence of differentiable functions defined on $A \subseteq \R$ and assume that $\sum_{n=1}^{\infty} f_n'$ converges uniformly on $A$ to $g$. If $\exists x_0 \in A$ where $\sum_{n=1}^{\infty} f_n(x)$ converges, then $\sum_{n=1}^{\infty} f_n(x)$ converges uniformly on $A$ to a function $f$ that is differentiable on $A$ with $f' = g$.}
            \emph{Proof:} Apply Differentiable Limit Theorem to $(S_k)$
        \end{tbox}

        \begin{tbox}{\textbf{Theorem (Cauchy Criterion for Uniform Convergence of Series):} The series $\sum_{n=1}^{\infty} f_n$ converges uniformly on $A \subseteq \R$ iff $\forall \ep > 0$, $\exists N \in \N$ such that $\abs{f_{m+1}(x) + f_{m+2}(x) + \dots + f_n(x)} < \ep$ when $n > m \geq N$ and $x \in A$.}
            \emph{Proof:} Follows immediately 
        \end{tbox}

        \begin{tbox}{\textbf{Corollary (Weierstrass M-test):} For each $n \in \N$, let $f_n$ be a function defined on $A \subseteq \R$ and $M_n > 0$ a real number satisfying 
            \[\abs{f_n(x)} \leq M_n \quad \forall x \in A\]
            If $\sum_{n=1}^{\infty} M_n$ converges, then $\sum_{n=1}^{\infty} f_n(x)$ converges uniformly on $A$}
            \emph{Proof:} HW
        \end{tbox}
    
\section{April 16:}
    \subsection*{Power Series}
        \textbf{Definition:} Functions expressed in the form 
        \[f(x) = \sum_{n=0}^{\infty} a_n x^n = a_0 + a_1x + a_2x^2 + \dots\]
        are called \emph{power series}. 
        
        \begin{tbox}{\textbf{Theorem:} If a power series $\sum_{n=0}^{\infty} a_nx^n$ converges at some point $x_0 \in \R$, then it converges absolutely for all $x$ with $\abs{x} < \abs{x_0}$}
            \emph{Proof:} If $\sum_{n=0}^{\infty} a_n x_0^n$ converges, then $(a_n x_0^n)$ converges to 0 and is bounded. 

            Since it is bounded, $\exists M > 0$ such that $\abs{a_n x_0^n} < M$ for all $n \in \N$. If $\abs{x} < \abs{x_0}$, then 
            \[\abs{a_n x^n} = \abs{a_n x_0^n} \abs{\frac{x}{x_0}}^n \leq M \abs{\frac{x}{x_0}}^n \]
            
            Since $\sum_{n=0}^{\infty} M \abs{\frac{x}{x_0}}^n$ is a geometric series with $\abs{\frac{x}{x_0}} < 1$, it converges. 
            
            By the Comparison Test, $\sum_{n=0}^{\infty} a_n x^n$ converges absolutely.
        \end{tbox}

        The set of points for which the power series converges is $\{0\}, \R$, or some interval around $0$: $(-R, R), [-R, R), (-R, R], [-R, R]$ where $R$ is the \emph{radius of convergence}

        \begin{tbox}{\textbf{Theorem:} If a power series $\sum_{n=0}^{\infty} a_n x^n$ converges absolutely at a point $x_0$, then it converges uniformly on the closed interval $[-c, c]$ where $c = \abs{x_0}$.}
            \emph{Proof:} Weierstrass M-theorem.
        \end{tbox}

        Notice! If $g(x) = \sum_{n=0}^{\infty} a_n x^n$ converges conditionally at $x = R$, then it \emph{is} possible for it to diverge when $x = -R$
         
        \emph{Example:} With $R = 1$ 
        \[g(x) = \sum_{n=1}^{\infty} \frac{(-1)^n x^n}{n} = \sum_{n=1}^{\infty} \frac{(-1)^n (-1)^n}{n = \sum_{n=1}^{\infty} \frac{(-1)^{2n}}{n}} = \sum_{n=1}^{\infty} \frac{1}{n}\]

        \begin{tbox}{\textbf{Abel's Lemma:} Let $(b_n)$ satisfy $b_1 \geq b_2 \geq \dots \geq 0$ and let $\sum_{n=0}^\infty a_n$ be a series for which the partial sums are bounded. Then, $\exists A$ such that 
        \[\abs{a_1 b_1 + a_2 b_2 + \dots + a_n b_n} \leq A\cdot b_1\]}
            \emph{Proof:} Let $s_n = a_1 + a_2 + \dots + a_n$. Recall Summation by Parts: with $s_0 = 0$,
            \[\sum_{j=m}^{n} x_j y_j = s_n y_{n+1} - s_{m-1}y_m + \sum_{j=m}^n s_j(y_j - y_{j+1})\]

            Therefore, 
            \begin{align*}
                \abs{\sum_{k=1}^n a_k b_k} &= \abs{s_n b_{n+1} + \sum_{k=1}^n s_k(b_k - b_{k+1})}\\ 
                    &\leq \abs{Ab_{n+1} + \sum_{k=1}^n A(b_k - b_{k+1})}\\
                    &\leq Ab_{n+1} + A \sum_{k=1}^n (b_k - b_{k+1})\\ 
                    &= Ab_{n+1} + A\left[(b_1 - b_2) + (b_2 - b_3) + \dots + (b_n - b_{n+1})\right]\\
                    &= Ab_{n+1} + A(b_1 - b_{n+1})\\
                    &= A b_1
            \end{align*}
        \end{tbox}

        \begin{tbox}{\textbf{Abel's Theorem:} Let $g(x) = \sum_{n=0}^{\infty} a_nx^n$ be a power series that converges at the point $x = R > 0$. Then $g(x)$ converges uniformly on $[0, R]$. A similar result holds if the series converges at $x = -R$.}
            \emph{Proof:} To be able to apply Abel's Lemma, we write 
            \[g(x) = \sum_{n=0}^{\infty} a_n x^n = \sum_{n=0}^\infty (a_n R^n) \left(\frac{x}{R}\right)^n\]

            Let $\ep > 0$. By Cauchy Criterion for uniform convergence of a series, if we can produce $N$ such that for $n > m \geq N$,
            \[\abs{(a_{m+1} R^{m+1})\left(\frac{x}{R}\right)^{m+1} + \dots + (a_n R^n) \left(\frac{x}{R}\right)^n} < \ep\]
            then we are done. 

            Since $\sum_{n=0}^\infty a_n R^n$ converges, by Cauchy Criterion for series, 
            \[\abs{a_{m+1} R^{m+1} + \dots + a_n R^n} < \frac{\ep}{2}\]
            when $n > m \geq N$. 

            Notice that $\left(\frac{x}{R}\right)^{m+j}$ is monotone decreasing so we can apply Abel's Lemma to get
            \[\abs{(a_{m+1} R^{m+1})\left(\frac{x}{R}\right)^{m+1} + \dots + (a_n R^n) \left(\frac{x}{R}\right)^n} \leq \frac{\ep}{2} \left(\frac{x}{R}\right)^{m+1} < \ep\]
        \end{tbox}

        \begin{tbox}{\textbf{Theorem:} If a power series converges pointwise on a set $A \subseteq \R$, then it converges uniformly on any compact set $K \subseteq A$. }
            \emph{Proof:} A compact set contains a max $x_1$ and a min $x_0$. By Abel's theorem, the series converges uniformly on $[x_0, x_1]$ and thus also on $k$. 
        \end{tbox}

        \textbf{Corollary:} A power series is continuous at every point at which it converges

        \begin{tbox}{\textbf{Theorem:} If $\sum_{n=0}^\infty a_nx^n$ converges $\forall x \in (-R, R)$ then the differentiated series $\sum_{n=1}^\infty na_nx^{n-1}$ converges a $x \in (-R, R)$.} 
            \emph{Proof:} HW
        \end{tbox} 

        \textbf{Corollary:} The convergence is uniform on compact sets contained in $(-R, R)$

\section{April 18:}
    \begin{tbox}{\textbf{Theorem:} Assume $f(x) = \sum_{n=0}^{\infty} a_n x^n$ converges on $A \subseteq \R$. The function $f$ is continuous on $A$ and differentiable on any open interval $(-R, R) \subseteq A$. The derivative is given by 
        \[f'(x) = \sum_{n=1}^{\infty} na_n x^{n-1}\]
        Moreover, $f$ is infinitely differentiable on $(-R, R)$ with successive derivatives obtained term-by-term differentiation.} 

        \emph{Proof:} Continuity follows from uniform convergence on compact sets (by pointwise convergence of $f$). Differentiability follows from term-by-term differentiation theorem. 
        
        The radius of convergence is constant because $\sum_{n=1}^{\infty} na_n x^{n-1}$ is also a power series with the same radius of convergence. 
        
        Infinite derivatives follow immediately by induction.
    \end{tbox}

    \subsection*{Taylor Series}
        Consider: 
        \[\frac{1}{1- x} = 1 + x + x^2 + x^3  + \dots\]
        for $\abs{x} < 1$. 

        We can equally replace $x$ with $-x^2$ and have 
        \[\frac{1}{1 + x^2} = 1 - x^2 + x^4 - x^6 + x^8 + \dots\]

        We recall from calculus that 
        \[\int \frac{1}{1 + x^2}\; dx = \arctan(x)\]
        so we can integrate term-by-term to get 
        \[\arctan(x) = x - \frac{x^3}{3} + \frac{x^5}{5} - \frac{x^7}{6} + \dots\]
        for $\abs{x} < 1$.

        \begin{tbox}{\textbf{Taylor's Formula:} Let 
            \[f(x) = a_0 + a_1 x + a_2 x^2 + \dots\]
            be defined on some nontrivial interval centered at $0$. Then 
            \[a_n = \frac{f^{n}(0)}{n!}\]}
            \emph{Proof:} HW
        \end{tbox}

        \textbf{Example:} Suppose $\sin(x)$ has a Taylor series. Then it could be written using Taylor's formula as 
        \[x - \frac{x^3}{3!} + \frac{x^5}{5!} - \frac{x^7}{7!} + \dots\]

        \textbf{Converse:} Assume $f$ is infinitely times differentiable around $0$. If $a_n = \frac{f^(n)(0)}{n!}$, does $\sum_{n=0}^\infty a_n x^n \to f(x)$?

        Potentially no! 

        Let $S_n = \sum_{k=0}^n a_k x^k$. Does $S_n \to f$? Consider $E_N(x) = f(x) - S_N(x)$. 

        \begin{tbox}{\textbf{Lagrange's Remainder Theorem:} Let $f$ be $N+1$ times differentiable on $(-R, R)$. Define 
            \[a_n = \frac{f^{(n)}(0)}{n!} \quad n = 0, 1, 2, \dots\] 
            and let $S_N(x) = a_0 + a_1x + \dots + a_N x^N$. Given $x \neq 0$ in $(-R, R)$, $\exists c$ such that $\abs{c} < \abs{x}$ where the error function $E_N(x)$ satisfies 
            \[E_N(x) = \frac{f^{(N+1)}(c)}{(N+1)!} x^{n+1}\]}
        
            \emph{Proof:} By assumption $f^{(n)} = S_n^{(n)}(0)$ for all $n$ such that $0 \leq n \leq N$. 

            Clearly, $E_N(x) = f(x) - S_N(x)$ satisfies $E_N^{(n)}(0) = 0$ for all $n = 0, 1, 2, \dots, N$.

            Assume WLOG $x > 0$. Apply the generalized MVT to $E_N(x)$ and $x^{N+1}$ on $[0, x]$ so

            \[\exists x_1 \in (0, x), \quad \frac{E_N(x)}{x^{N+1}} = \frac{E_N'(x_1)}{(N+1)x_1^N}\]

            We can apply the MVT to $E'N(x)$ and $(N+1)x^N$ on $[0, x_1]$. Then, $\exists x_2 \in (0, x_2)$ such that 
            \[\frac{E_N(x)}{x^{N+1}} = \frac{E_N'(x_1)}{(N+1)x_1^N} = \frac{E_N''(x_2)}{(N+1)(N)x_2^{N-1}}\]

            We may repeat this until we have $x_{N+1} \in (0, x_n) \subset \dots \subset (0, x)$ with 
            \[\frac{E_N(x)}{x^{N+1}} = \frac{E_N^{(N+1)}(x_{N+1})}{(N+1)!}\]

            Set $c = x_{N+1}$. Since $S_N^{(N+1)}(x) = 0$, we have 
            \[E_N^{(N+1)}(x) = f^{(N+1)}(x) \implies E_N(x) = \frac{f^{(N+1)}(c)}{(N+1)!}x^{N+1}\]
        \end{tbox}

        \textbf{Example:} How well does $S_5(x) = x - \frac{x^3}{3!} + \frac{x^5}{5!}$ approximate $\sin(x)$ for $x \in [-2, 2]$?

        By Lagrange's Remainder Theorem, 
        \[E_5(x) = \sin(x) - S_5(x) = -\frac{\sin(c)}{6!}x^6\]
        for some $c \in (-\abs{x}, \abs{x})$. 

        $\abs{\sin(c)} \leq 1$ so for $x \in [-2, 2]$, $\abs{E_5} \leq \frac{2^6}{6!} \approx 0.089$

        To prove $S_N \to \sin(x)$ uniformly on $[-2, 2]$, using $\abs{f^{(N+1)}(c)}\leq 1$, we get 
        \[\abs{E_N(x)} = \abs{\frac{f^{(N+1)(c)}}{(N+1)!} x^{N+1}} \leq \frac{2^{N+1}}{(N+1)!} \to 0\]

        But! We can replace $[-2, 2]$ by $[-R, R]$ with $R$ arbitrary. The Taylor Series thus converges to $\sin(x)$ on every $[-R, R]$.

        \begin{tbox}{\textbf{Theorem:} If $f$ is defined in some neighborhood of $a \in \R$ and infinitely differentiable at $a$, then 
            \[\sum_{n=0}^\infty c_n(x - a), \quad c_n = \frac{f^{(n)}(a)}{n!}\]}
            \emph{Proof:} By Lagrange's Remainder Theorem, $\exists c \in (a, x)$ such that 
            \[E_N(x) = \frac{f^{(N+1)}(c)}{(N+1)!}(x - a)^{N+1}\]
        \end{tbox}

\chapter{Riemann Integral}
\section{April 23:}
    Throughout this chapter, we assume $f$ is bounded on $[a, b]$: that is, there exists $M > 0$ such that $\abs{f(x)} < M$ for all $x \in [a, b]$

    \textbf{Definition:} A \emph{partition} $P$ of $[a, b]$ is a finite set of points from $[a, b]$ that includes $a$ nad $b$: 
    \[P = \{x_0, x_1, \dots, x_n\} \quad \text{such that} \quad a = x_0 < x_1 < \dots < x_n = b\]

    For each subinterval $[x_{k-1}, x_k]$ of $P$: 
    \[m_k = \inf\{f(x): x \in [x_{k-1}, x_k]\}\]
    \[M_k = \sup\{f(x): x \in [x_{k-1}, x_k]\}\]

    The lower sum of $f$ with respect to $P$ is 
    \[L(f, P) = \sum_{k=1}^n m_k(x_k - x_{k-1})\] 
    (sum of areas of rectangles with height $m_k$ and width $x_k - x_{k-1}$ which \emph{underestimates} the value)

    The upper sum of $f$ with respect to $P$ is
    \[U(f, P) = \sum_{k=1}^n M_k(x_k - x_{k-1})\]
    (sum of areas of rectangles which \emph{overestimate} the value) 

    \textbf{Definition:} A partition $Q$ is \emph{refinement} of $P$ if $P \subseteq Q$ ($Q$ contains all points of $P$)    

    \begin{tbox}{\textbf{Lemma:} If $P \subseteq Q$, then $L(f, P) \leq L(f, Q)$ and $u(f, P) \geq u(f, Q)$}
        \emph{Proof:} Consider the refinement of $P$ by considering $\{z\} \cup [x_{k-1}, x_k]$.

        Now, our lower sum is 
        \begin{align*}
            m_k(x_k - x_{k-1}) &= m_k(x_k - z) + m_k(z - x_{k-1})\\ 
            &\leq m'_k(x_k - z) + m_k'' (z - x_{k-1})
        \end{align*}
        where 
        \begin{align*}
            m_k' &= \inf\{f(x): x \in [z, x_k]\}\\
            m_k'' &= \inf\{f(x): x \in [x_{k-1}, z]\}
        \end{align*}

        By induction on $k$, we have $L(f, P) \leq L(f, Q)$

        Similar argument shows the upper sum case. 
    \end{tbox}

    \begin{tbox}{\textbf{Lemma:} If $P_1$ and $P_2$e are any two partitions of $[a, b]$, then $L(f, P_1) \leq u(f, P_2)$}
        \emph{Proof:} Let $Q = P_1 \cup P_2$ (the common refinement of $P_1$ and $P_2$) 

        Since $P_1 \subseteq Q$ and $P_2 \subseteq Q$, 
        \[L(f, P_1) \leq L(f, Q) \leq u(f, Q) \leq u(f, P_2)\]
        by the previous lemma
    \end{tbox}

    \subsection*{Integrability}
    A function is integrable if the upper and lower sums ``meet'' as partitions get more refined. 

    \emph{Idea:} instead of limits, use AoC and $\lim$/$\sup$

    \textbf{Definition:} Let $\mathcal{P}$ be the collection of all possible partitions of $[a, b]$. The \emph{upper integral} of $f$ is 
    \[U(f) = \inf\{U(f, P): P \in \mathcal{P}\}\]
    the \emph{lower integral} of $f$ is 
    \[L(f) = \sup\{L(f, P) : P \in \mathcal{P}\}\]

    \begin{tbox}{\textbf{Lemma:} For any bounded function $f$ on $[a, b]$ we always have $U(f) \geq L(f)$ }
        \emph{Proof:} HW
    \end{tbox}

    \textbf{Definition (Riemann Integrability):} A bounded function $f$ defined on $[a, b]$ is Riemann-integrable if $U(f) = L(f)$. Then we write 
    \[\int_a^b f = U(f) = L(f)\]

    \subsection*{Criterion for Integrability}
    To review:
    \begin{align*}
        \sup\{L(f, P), P \in \mathcal{P}\} &= L(f) \leq U(f)\\ 
            &= \inf\{U(f, P), P \in \mathcal{P}\}
    \end{align*}
    and $f$ is integrable if $L(f) = U(f)$

    \begin{tbox}{\textbf{Theorem (Integrability Criterion):} A bounded function $f$ is interable on $[a, b]$ iff $\forall \ep > 0$, $\exists P_{\ep}$, a partition of ${a, b}$ such that 
        \[U(f, P_{\ep}) - L(f, P_{\ep}) < \ep\] }
        \emph{Proof:} ($\impliedby$) Let $\ep > 0$. If such a partition $P_{\ep}$ exists, then $U(f) - L(f) \leq U(f, P_{\ep})  L(f, P_{\ep}) < \ep$

        Since $\ep$ is arbitrary, $U(f) = L(f)$ so $f$ is integrable. 

        ($\implies$) Since $U(f)$ is the greatest lower bound of the upper sums, given $\ep > 0$, $\exists P_1$ such that
        \[U(f, P_1) < U(f) + \frac{\ep}{2}\]
        and $\exists P_2$ such that 
        \[L(f, P_2) > L(f) - \frac{\ep}{2}\]

        Let $P_{\ep} = P_1 \cup P_2$ be the common refinement of $P_1$ and $P_2$. Then...
    \end{tbox}

    \begin{tbox}{\textbf{Theorem:} If $f$ is continuous on $[a, b]$ then it is integrable. }
        \emph{Proof:} Since $f$ is continuous on a compact set, it is bounded, and uniformly continuous. 

        Therefore, given $\ep > 0$, $\exists \delta$ such that $\abs{x - y} < \delta \implies \abs{f(x) - f(y)} < \frac{\ep}{b - a}$

        Let $P$ be a partition on $[a, b]$ where 
        \[\Delta x_k = x_k - x_{k-1} < \delta\] 

        Given a particular $[x_{k-1}, x_k]$, by the Extreme Value Theorem 
        \begin{align*}
            \sup \implies \exists M_k = f(z_k) \quad \text{for some} \quad z_k \in [x_{k-1}, x_k]\\ 
            \inf \implies \exists m_k = f(y_k) \quad \text{for some} \quad y_k \in [x_{k-1}, x_k]
        \end{align*}

        Therefore, $\abs{z_k - y_k} < \delta$ so $M_k - m_k = f(z_k) - f(y_k) < \frac{\ep}{b - a}$

        Therefore, 
        \begin{align*}
            U(f, P) - L(f, P) &:= \sum_{k=1}^n (M_k - m_k) \delta x_k\\ 
            &< \frac{\ep}{b - a} \sum_{k=1}^n \delta x_k\\ 
            &= \frac{\ep}{b - a} (b - a) = \ep
        \end{align*}
        Therefore, $f$ is integrable.
    \end{tbox}

\section{April 25:}
    \subsection*{Recall} 
        \begin{itemize}
            \item $m_k = \inf\{f(x): x \in [x_{k-1}, x_k]\},\; M_k = \sup\{f(x): x \in [x_{k-1}, x_k]\}$
            \item $L(f, P) = \sum_{k=1}^n m_k(x_k - x_{k-1}),\; U(f, P) = \sum_{k=1}^n M_k(x_k - x_{k-1})$
            \item Upper integral: $U(f) = \inf\{U(f, P): P \in \mathcal{P}\}$, Lower integral: $L(f) = \sup\{L(f, P): P \in \mathcal{P}\}$
            \item Integrable: $U(f) = L(f) \iff \forall \ep > 0, \exists P_{\ep}$ such that $U(f, P_{\ep}) - L(f, P_{\ep}) < \ep$
            \item $f$ continuous $\implies$ $f$ integrable
        \end{itemize}

    \subsection*{Integrating functions with discontinuities}
        \textbf{Example:} $f(x) = \begin{cases}
            1 & x \neq 1\\ 
            0 & x = 1
        \end{cases}$ on $[0, 2]$. 

        For \emph{any} partition $P$ of $[0, 2]$ we have $U(f, P) = \sum_{k=1}^n M_k(x_k - x_{k - 1}) < \sum_{k=1}^n 1(2) = 2$. Therefore, the lower sum is less than $2$

        We will proceed by constructing a partition that embeds $x = 1$ into a very small subinterval. Let $\ep > 0$, consider 
        \[P_{\ep} = \{0, 1 - \frac{\ep}{3}, 1 + \frac{\ep}{3}, 2\}\]
        (we are tightly constraining our discontinuity) 

        Then 
        \[L(f, P_{\ep}) = 1 \cdot (1 - \frac{\ep}{3}) + 0 + 1\cdot (1 - \frac{\ep}{3}) = 2 - \frac{2\ep}{3} < 2\]

        Since $U(f, P_{\ep}) = 2$, 
        \[U(f, P_{\ep}) - L(f, P_{\ep}) = \frac{2}{3}\ep < \ep\]

        \begin{tbox}{\textbf{Theorem:} If $f: [a,b] \to \R$ is bounded and $f$ is integrable on $[c, b]$ for all $c \in (a, b)$, then $f$ is integrable on $[a, b]$}
            \emph{Proof:} Let $\ep > 0$. It sufficesd to produce a $P$ such that $U(f, P) - L(f, P) < \ep$. 

            For any partition, we have 
            \begin{align*}
                U(f, P) - L(f, P) &= \sum_{k=1}^n (M_k - m_k)(x_k - x_{k-1})\\
                    &= \sum_{k=1}^n (M_k - m_k) \Delta x_k\\
                    &= (M_1 - m_1)(x_1 - a) + \sum_{k=2}^n (M_k - m_k) \Delta x_k
            \end{align*}
            
            We want to choose $x_1$ close enough to $a$ such that 
            \[(M_1 - m_1)(x_1 - a) < \frac{\ep}{2}\]

            Since $f$ is bounded, $\exists M > 0$ such that $\abs{f(x)} < M$ for all $x \in [a, b]$. Now $M_1 - m_1 \leq 2M$, so we can pick $x_1$ such that so 
            \[x_1 - a < \frac{\ep}{4M}\]

            By assumption, $f$ is integrable on $[x_1, b]$ so $\exists P_1$ such that 
            \[U(f, P_1) - L(f, P_1) < \frac{\ep}{2}\]

            Let $P = \{a\} \cup P$. Then 
            \[U(f, P) - L(f, P) \leq 2M(x_1 - a) + U(f, P_1)  L(f, P_1) < \frac{\ep}{2} + \frac{\ep}{2} < \ep\]
        \end{tbox}

    \subsection*{Properties of the Integral} 
        \begin{tbox}{\textbf{Theorem:} Assume $f: [a, b] \to \R$ is bounded and let $c \in (a, b)$. Then, $f$ is integrable on $[a, b]$ iff $f$ is integrable on $[a, c]$ and $[c, b]$. In that case, we may write 
            \[\int_a^b f = \int_a^c f + \int_c^b f\]}
            \emph{Proof:} ($\implies$) If $f$ is integrable on $[a, b]$, then $\forall \ep > 0$, $\exists P$ such that $U(f, P) - L(f, P) < \ep$.

            Let $P_1 = P \cap [a, c]$ (a partition of $[a, c]$) and $P_2 = P \cap [c, b]$ (a partition of $[c, b]$). 

            Then 
            \[U(f, P_1) - L(f, P_1) < \ep \quad \text{and} \quad U(f, P_2) - L(f, P_2) < \ep \]
            Therefore, $f$ is integrable on $[a, c]$ and $[c, b]$ 

            ($\impliedby$) Assume $f$ is integrable on $[a, c]$ and $[a, b]$. Then given $\ep > 0$, $\exists P_1 \subset [a, c]$ and $P_2 \subset [c, b]$ such that 
            \[U(f, P_1) - L(f, P_1) < \frac{\ep}{2} \quad \text{and} \quad U(f, P_2) - L(f, P_2) < \frac{\ep}{2} \]

            Let $P = P_1 \cup P_2$. Then
            \[U(f, P) - L(f, P) < \ep\] 
            so $f$ is integrable on $[a, b]$
            
            Finally, we want to show that $\int_a^b f = \int_a^c f + \int_c^b f$.

            We know 
            \begin{align*}
                \int_a^b \leq U(f, P) &< L(f, P) + \ep\\ 
                &= L(f, P_1) + L(f, P_2) + \ep\\ 
                &\leq \int_a^c f + \int_c^b f + \ep\\ 
                &\leq \int_a^c f + \int_c^b f
            \end{align*}
            
            Now we show the other direction: 
            \begin{align*}
                \int_a^c f + \int_c^b f &\leq U(f, P_1) + U(f, P_2)\\
                    &< L(f, P_1) + L(f, P_2) + \ep\\ 
                    &= L(f, P) + \ep\\ 
                    &\leq \int_a^c f + \ep\\ 
                    &\leq \int_a^b f
            \end{align*}

            Therefore, $\int_a^b f = \int_a^c f + \int_c^b f$
        \end{tbox}

        \begin{tbox}{\textbf{Theorem:} Assume $f, g$ are integrable on $[a, b]$. Then 
            \begin{enumerate}[label=(\roman*)]
                \item $f + g$ is integrable on $[a, b]$ with $\int_a^b(f + g) = \int_a^b f + \int_a^b g$ 
                \item $kf$ is integrable on $[a, b]$ for any $k \in \R$ with $\int_a^b kf = k\int_a^b f$
                \item If $m \leq f(x) \leq M$ on $[a, b]$, then $m(b - a) \leq \int_a^b f \leq M(b - a)$
                \item $f(x) \leq g(x)$ for all $x \in [a, b]$ implies $\int_a^b f \leq \int_a^b g$
                \item $\abs{f}$ is integrable and $\abs{\int_a^b f} \leq \int_a^b \abs{f}$
            \end{enumerate}}
            \emph{Proof:} $f$ is integrable $\iff$ $\exists (P_n)$ such that 
            \[\lim_{n \to \infty} [U(f, P_n) - L(f, P_n)] = 0 \qquad \text{(\emph{Proof:} HW)}\]

            Therefore
            \[\int_a^b f = \lim U(f, P_n) = \lim L(f, P_n)\]

            The rest of the results follow:
            \begin{enumerate}[label=(\roman*)]
                \item HW 
                
                \item For $k \geq 0$, from properties of $\sup$ and $\inf$ 
                \[U(kf, P) = kU(f, P), \quad L(kf, P) = kL(f, P)\]
                since $f$ is integrable, $\exists (P_n)$ such that 
                \[\lim_{n\to \infty}[U(kf, P_n) - L(kf, P_n)] = \lim_{n \to \infty} k(U(f, P_n) - L(f, P_n)) = 0\]

                For $k < 0$, similar except $U(kf, P) = kL(f, P)$ and $L(kf, P) = kU(f, P)$

                \item Since $U(f, P )\geq L(f, P)$ for all $P$, pick $P = [a, b]$ so 
                \[M(b- a) \geq \int_a^b f \geq m(b - a)\] 

                \item Set $h = g - f$ and apply (i), (ii), (iii) 
                
                \item HW
            \end{enumerate}
        \end{tbox}

        \textbf{Defintion:} If $f$ is integrable on $[a, b]$, define 
        \[\int_a^b = -\int_b^a f\]

        Moreover $\forall c \in [a, b]$: $\int_c^c f = 0$ 

        \textbf{Question:} If $f_n \to f$, does $\int f_n = \int f$? No! Unless the convergence is uniform as we will see. 

        \subsection*{Interlude: Arzela-Ascoli Theorem}
            Recall the Bolzano-Weierstrass Theorem for sequences: $a_n$ bounded implies $\exists a_{n_k}$ convergent. 

            Does this hold for functions, i.e. does $f_n(x)$ bounded imply $\exists f_{n_k}$ convergent? 

            Not quite! We can show $f_n(x)$ bounded on a countable set $A$ implies $\exists f_{n_k}$ pointwise convergent on $A$ (Exercise 6.2.13)

            \textbf{Definition:} $(f_n)$ is \emph{equicontinuous} on $E \subset \R$ if $\ep > 0$, $\exists \delta > 0$ such that $\abs{x - y} < \delta \implies \abs{f_n(x) - f_n(y)} < \ep$ for all $n \in \N$

            \begin{tbox}{\textbf{Theorem:} If $f_n(x)$ is bounded and equicontinuous then $\exists f_{n_k}$ uniformly convergent on $A$.}
                \emph{Proof:} HW
            \end{tbox}

            \textbf{Remark:} This is a much stronger result!  

\section{April 30:} 
    \textbf{Question:} If $(f_n(x)) \to f(x)$ of integrable functions on $[a, b]$, do we have that $\int f_n \to \int f$? 

    \emph{Answer:} When $(f_n(x)) \to f(x)$ uniformly, then yes! 

    \begin{tbox}{\textbf{Integrable Limit Theorem:} Assume that $f_n \to f$ uniformly on $[a, b]$ and that each $f_n$ is integrable on $[a, b]$. Then $f$ is integrable and \[\lim_{n \to \infty} \int_a^b f_n = \int_a^b f\]}
        \emph{Proof:} Proof that $f$ is integrable is HW. 

        From the properties of integrals,
        \[\abs{\int_a^b f_n - \int_a^b f} = \abs{\int_a^b f_n - f} \leq \int_a^b \abs{f_n - f}\]
        But $f_n \to f$ uniformly so for all $\ep > 0$, $\exists N$ such that $\abs{f_n - f} < \frac{\ep}{b - a}$ for all $n \geq N$ so 
        \[\abs{\int_a^b f_n - \int_a^b f} \leq \int_a^b \abs{f_n(x) - f(x)} < \int_a^b \frac{\ep}{b - a} = \ep \qed\]
    \end{tbox}

    \begin{tbox}{\textbf{Fundamental Theorem of Calculus:} 
        \begin{enumerate}[label=(\roman*)]
            \item If $f: [a, b] \to \R$ is integrable and $F: [a, b] \to \R$ satisfies $F'(x) = f(x)$ for all $x \in [a, b]$, then 
            \[\int_a^b f = F(b) - F(a)\]
            \item Let $g: [a, b] \to \R$ be integrable and define $G(x) = \int_a^x f$ for all $x \in [a, b]$. Then: $G$ is continuous on $[a, b]$. If $g$ is continuous at some point $c \in [a, b]$, then $G$ is differentiable at $c$ and $G'(c) = g(c)$
        \end{enumerate}}
        \emph{Proof:} 

        \begin{enumerate}[label=(\roman*)]
            \item Let $P$ be a partition of $[a, b]$ and apply the MVT to $F$ on a subinterval $[x_{k-1}, x_k]$ of $P$ to get $\exists t \in (x_{k - 1}, x_k)$ such that 
            \begin{align*}
                F(x_k) - F(x_{k-1}) &= F'(t_k)(x_k - x_{k-1})\\ 
                &= f(t_k) (x_k - x_{k-1})
            \end{align*}

            Now, consider the upper and lower sums $U(f, P)$ and $L(f, P)$. Since 
            \[m_k = \inf_{t_k \in (x_{k-1}, x_k)} f(t_k) \leq f(t_k) \leq M_k = \sup_{t_k \in (x_{k-1}, x_k)} f(t_k)\]
            we have
            \begin{align*}
                m_k &\leq f(t_k) \leq M_k\\ 
                \sum m_k(x_k - x_{k-1}) &\leq \sum f(t_k)(x_k - x_{k-1})\leq  \sum M_k(x_k - x_{k-1})\\ 
                L(f, P) &\leq \sum F(x_k) - F(x_{k-1}) \leq U(f, P)
            \end{align*}
            \[L(f, P) \leq \sum_{k=1}^n [F(x_k) - F(x_{k-1})] \leq U(f, P)\]
            but this is a telescoping sum! so 
            \[L(f, P) \leq F(b) - F(a) \leq U(f, P)\]
            but this is independent of $P$ which gives 
            \[L(f) \leq F(b) - F(a) \leq U(f)\]

            Since $f$ is integrable, $U(f) = L(f)$ so $F(b) - F(a) = \int_a^b f$

            \item Take $x > y$ in $[a, b]$. Observe: 
            \[\abs{G(x) - G(y)} = \abs{\int_a^x g - \int_a^y g} = \abs{-\int_x^a g - \int_a^y g} = \abs{\int_y^x g} \leq \int_y^x \abs{g} \leq M(x - y)\]
            where $M > 0$ is a bound on $\abs{g}$. 

            Therefore, $G$ is Lipschitz on $[a, b]$ so $G$ is uniformly continuous on $[a, b]$.

            Assume $g$ is continuous at $c \in [a, b]$.

            By definition, 
            \[G'(c) = \lim_{x \to c} \frac{G(x) - G(c)}{x- c} = \lim_{x \to c} \frac{1}{x - c} \left(\int_a^x g(t) \; dt - \int_a^c g(t)\; dt\right) = \lim_{x \to c} \frac{1}{x - c} \int_c^x g(t) \; dt\]

            It suffices to show this limit equals $g(c)$. 

            Given $\ep > 0$, we want a $\delta > 0$ such that $\abs{x -c}< \delta$ implies \[\abs{\frac{1}{x - c} \int_c^x g(t)\; dt - g(c)} < \ep\]

            By continuity of $g$, $\exists \delta> 0$ such that $\abs{x - c} < \delta$ implies $\abs{g(x) - g(c)} < \ep$. 

            We write 
            \[g(c) = \frac{1}{x - c} \int_c^x g(c) \; dt\]
        
            Since $\abs{x  -c} \geq \abs{t - c}$, we have $\forall \abs{x - c} < \delta$, 
            \begin{align*}
                \abs{\frac{1}{x - c} \int_c^x g(t)\; dt - g(c)} &= \abs{\frac{1}{x - c} \int_c^x g(t)\; dt - \frac{1}{x - c} \int_c^x g(c) \; dt}\\ 
                &= \abs{\frac{1}{x - c} \int_c^x [g(t) - g(c)]\; dt}\\
                &\leq \frac{1}{\abs{x - c}} \int_c^x \abs{g(t) - g(c)}\; dt\\
                &< \frac{1}{x - c} \int_c^x \ep \; dt = \ep \qed
            \end{align*}
        \end{enumerate}
    \end{tbox}
    
\end{document}  






