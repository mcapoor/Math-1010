\documentclass[12pt]{article} 
\usepackage[utf8]{inputenc}
\usepackage{geometry}
\geometry{letterpaper}
\usepackage{graphicx} 
\usepackage{parskip}
\usepackage{booktabs}
\usepackage{array} 
\usepackage{paralist} 
\usepackage{verbatim}
\usepackage{subfig}
\usepackage{fancyhdr}
\usepackage{sectsty}
\usepackage[shortlabels]{enumitem}

\pagestyle{fancy}
\renewcommand{\headrulewidth}{0pt} 
\lhead{}\chead{}\rhead{}
\lfoot{}\cfoot{\thepage}\rfoot{}

\geometry{
    left=0.5in, 
    right=0.5in,
    top=0.5in,
    bottom=0.5in
}


%%% ToC (table of contents) APPEARANCE
\usepackage[nottoc,notlof,notlot]{tocbibind} 
\usepackage[titles,subfigure]{tocloft}
\renewcommand{\cftsecfont}{\rmfamily\mdseries\upshape}
\renewcommand{\cftsecpagefont}{\rmfamily\mdseries\upshape} %

\usepackage{amsmath}
\usepackage{amssymb}
\usepackage{mathtools}
\usepackage{empheq}
\usepackage{xcolor}

\usepackage{tikz}
\usepackage{pgfplots}
\pgfplotsset{compat=1.18}

\newcommand{\ans}[1]{\boxed{\text{#1}}}
\newcommand{\vecs}[1]{\langle #1\rangle}
\renewcommand{\hat}[1]{\widehat{#1}}
\newcommand{\F}[1]{\mathcal{F}(#1)}
\renewcommand{\P}{\mathbb{P}}
\newcommand{\R}{\mathbb{R}}
\newcommand{\E}{\mathbb{E}}
\newcommand{\Z}{\mathbb{Z}}
\newcommand{\N}{\mathbb{N}}
\newcommand{\Q}{\mathbb{Q}}
\newcommand{\ind}{\mathbbm{1}}
\newcommand{\qed}{\quad \blacksquare}
\newcommand{\brak}[1]{\left\langle #1 \right\rangle}
\newcommand{\bra}[1]{\left\langle #1 \right\vert}
\newcommand{\ket}[1]{\left\vert #1 \right\rangle}
\newcommand{\abs}[1]{\left\vert #1 \right\vert}
\newcommand{\mfX}{\mathfrak{X}}
\newcommand{\ep}{\varepsilon}

\usepackage{tcolorbox}
\tcbuselibrary{breakable, skins}
\tcbset{enhanced}
\newenvironment*{tbox}[2][gray]{
    \begin{tcolorbox}[
        parbox=false,
        colback=#1!5!white,
        colframe=#1!75!black,
        breakable,
        title={#2}
    ]}
    {\end{tcolorbox}}


\title{Math 1010 - Homework 8}
\author{}
\date{}

\begin{document}
\maketitle
\vspace*{-1in}

\section*{Problem 1 }
Assume $h:\mathbb{R}\to\mathbb{R}$ is continuous on $\mathbb{R}$ and let $K=\{x:h(x)=0 \}$. Show that $K$ is a closed set.

    \color{blue}
        We proceed by cases. If $K = \emptyset$, then certainly it is closed.

        Suppose $K$ is non-empty and finite. Then $K = \{x_1, x_2, \dots, x_n\} = \bigcup_{i =1}^{\infty} \{x_{i}\}$. Since $\{x_i\}$ is a singleton set, it is closed. (\emph{Proof:} $\{x_i\}^c = (-\infty, x_i) \cup (x_i, \infty)$ and the union of an arbitrary collection of open sets is open so $\{x_i\}$ is closed). Then, the union of a finite collection of closed sets is closed, so $K$ is closed. 
        
        Now suppose $K$ is non-empty and infinite. Let $(x_n) \in K$ and suppose $x_n \to x$. By continuity of $f$, $f(x_n) \to f(x)$. Since $x_n \in K$ for all $n$, $f(x_n) = 0 \implies f(x) = 0$. Therefore, $x \in K$. Since every convergent sequence in $K$ has its limit in $K$, $K$ is closed. $\qed$
    \color{black}

\pagebreak

\section*{Problem 2 (Contraction Mapping Theorem)}
Let $f$ be a function defined on all of $\mathbb{R}$ and assume there is a constant $c$ such that $0<c<1$ and 
\begin{equation*}
	\abs{f(x)-f(y)}\leq c\abs{x-y}
\end{equation*}
for all $x,y\in\mathbb{R}$.
\begin{enumerate}
	\item Show that $f$ is continuous on $\mathbb{R}$.
    
        \color{blue}
            Let $\ep > 0$. Choose $\delta = \frac{\ep}{c}$ Suppose $\abs{x - y} < \delta$. Then 
            \[\abs{f(x) - f(y)} \leq c\abs{x- y} < c\delta = \ep\]
            Therefore, $f(x)$ is continuous at all $y \in \R$. $\qed$
        \color{black}

	\item Pick some point $y_1\in\mathbb{R}$ and construct the sequence $$(y_1,f(y_1),f(f(y_1)),\cdots).$$
	In general, if $y_{n+1}=f(y_n)$, show that the resulting sequence $(y_n)$ is a Cauchy sequence. Hence we may let $y=\lim y_n$.

        \color{blue}
            We seek to show that $\forall \ep > 0$, $\exists N \in \N$ such that $\forall m,n \geq N$, 
            \[\abs{y_m - y_n} < \ep\]

            In the case $m = n$, $\abs{y_m - y_n} = 0 < \ep$ and we are done.

            Notice that 
            \begin{align*}
                \abs{y_{n+1} - y_n} &= \abs{f(y_n) - f(y_{n-1})}\\
                &\leq c\abs{y_n - y_{n-1}}\\
                &\leq c^2\abs{y_{n-1} - y_{n-2}}\\
                &\leq \cdots\\
                &\leq c^{n-1}\abs{y_2 - y_1}
            \end{align*}

            Since $0 < c < 1$, $\lim_{n\to\infty} c^{n-1} \abs{y_2-y_1} = 0$. 
           
            Thus for all $\ep_0 > 0$, choose $N$ such that $\abs{y_{n+1} - y_n} \leq c^{N-1}\abs{y_2 - y_1} < \ep$. Then $\forall m,n \geq N$,
            \begin{align*}
                \abs{y_m - y_n} &= \abs{y_m - y_{m-1} + y_{m-1} - y_{m-2} + \cdots + y_{n+1} - y_n}\\
                &\leq \abs{y_m - y_{m-1}} + \abs{y_{m-1} - y_{m-2}} + \cdots + \abs{y_{n+1} - y_n}\\
                &\leq \ep_0 + \ep_0 + \cdots + \ep_0\\ 
                &= (m - n + 1)\ep_0
            \end{align*}

            Let $\ep > 0$. Since $\ep_0$ is arbitrary, we can say $\ep = \frac{\ep_0}{m -n + 1}$ so $\abs{y_m - y_n} < \ep$. Therefore, $(y_n)$ is a Cauchy sequence. $\qed$        
        \color{black}

	\item Prove that $y$ is a fixed point of $f$ (i.e. $f(y)=y$) and that it is unique in this regard.
	
        \color{blue}
            From (2), $(y_n) \to y$. Notice, though, that 
            \[f(y_n) = (f(y_1), f(f(y_1)), f(f(f(y_1))), \dots)\]
            is a subsequence of $(y_n)$ so $f(y_n) \to y$. 

            By continuity of $f$, $(y_n) \to y \implies f(y_n) \to f(y)$. Since the limit of a sequence is unique, $f(y) = y$. 

            Now suppose that $\exists x\neq y$ for which $f(x) = x$. Then 
            \[\abs{f(x) - f(y)} = \abs{x - y} \leq c\abs{x - y}\]
            by definition of $f$. However, this implies $c= 1$ which is a contradiction of the condition $0 < c < 1$. Therefore, $y$ is the only fixed point of $f$. $\qed$
        \color{black}

	\item Finally, prove that if $x$ is any arbitrary point in $\mathbb{R}$, then the sequence $(x, f(x),f(f(x)),\cdots )$ converges to $y$ defined in (2).
	
        \color{blue}
            Let $x \neq y_1 \in \R$. Repeating the argument in (2), we see that $(x_n) = (x, f(x), f(f(x)), \dots)$ is a Cauchy sequence with $\lim x_n = x$. However, by the same subsequence argument in (3), $f(x) = x$. By the continuity of $f$ and the uniqueness of the fixed point $y$ in (3), $x = y$. Therefore, the sequeunce $(x, f(x), f(f(x)), \dots) \to y$ for all $x \in \R$. $\qed$
        \color{black}

\end{enumerate}

\pagebreak


\section*{Problem 3}
\begin{enumerate}
	\item Show that $f(x)=x^3$ is continuous on all of $\mathbb{R}$
	
        \color{blue}
            Let $\ep > 0$. Denote $g(x) = x$. Choose $\delta = \ep$. Suppose $\abs{x - y} < \delta$. Then
            \[\abs{g(x) - g(y)} = \abs{x - y} < \delta = \ep\]
            Therefore, $g(x)$ is continuous at all $y \in \R$.

            By the Algebraic Continuity theorem, $x^3 = g(x)g(x)g(x)$ is continuous at all $x \in \R$. $\qed$
        \color{black}

	\item Argue, using the sequential criterion for absence of uniform continuity, that $f$ is not uniformly continuous on $\mathbb{R}$
	
        \color{blue}
            By the Sequential Criterion for Absence of Uniform Continuity, $f$ fails to be uniformly continuous if $\exists \ep > 0$ and $(x_n), (y_n) \in \R$ such that $\forall \delta > 0$, $\abs{x - y} \to 0$ but $\abs{f(x) - f(y)} \geq 0$.
        
            Consider $x_n = \frac{1}{n}$ and $y_n =-\frac{1}{n}$. Clearly, $\abs{x_n - y_n} = \frac{2}{n} \to 0$. However, 
            \begin{align*}
                \abs{f(x_n) - f(y_n)} &= \abs{\frac{1}{n^3} + \frac{1}{n^3}} = \frac{2}{n^3}
            \end{align*}

            Therefore, we can choose an $\ep_0$ for which $\abs{f(x_n)- f(x_n)} \geq \ep_0$ so $f$ is not uniformly continuous. $\qed$
        \color{black}

	\item Show that $f$ is uniformly continuous on any bounded subset of $\mathbb{R}$
            
        \color{blue}
            Let $A$ be a bounded subset of $\R$. Then there exists $M > 0$ such that $\abs{x} < M$ for all $x \in A$. 

            Let $\ep > 0$. Choose $\delta = \frac{\ep}{3M^2}$. Suppose $\abs{x - y} < \delta$. Then
            \begin{align*}
                \abs{f(x) - f(y)} &= \abs{x^3 - y^3}\\ 
                    &= \abs{x - y}\,\abs{x^2 + xy + y^2}\\ 
                    &\leq \abs{x - y}\,\abs{M^2 + M^2 + M^2}\\ 
                    &= 3M^2\abs{x - y}\\ 
                    &< 3M^2\delta\\
                    &= \ep
            \end{align*}
            Therefore, $f(x)$ is uniformly continuous on $A$. $\qed$
        \color{black}

\end{enumerate}

\pagebreak 

\section*{Problem 4 (Lipschitz Functions)}
A function $f:A\to\mathbb{R}$ is called \textit{Lipschitz} if there exists a bound $M>0$ such that
\begin{equation*}
	\bigg|\frac{f(x)-f(y)}{x-y} \bigg|\leq M
\end{equation*}
for all $x\neq y\in A$. Geometrically speaking, a function $f$ is Lipschitz if there is a uniform bound on the magnitude of the slopes of lines drawn through any two points on the graph of $f$.
\begin{enumerate}
	\item Show that if $f:A\to\mathbb{R}$ is Lipschitz, then it is uniformly continuous on A.
    
        \color{blue}
            Let $\ep > 0$. If $f: A \to \R$ is Lipschitz, then 
            \[\abs{f(x) - f(y)} \leq M \abs{x - y}\]

            Choose $\delta = \frac{\ep}{M}$. If $\abs{x - y} < \delta$, 
            \[\abs{f(x) - f(y)} \leq M\abs{x - y} < M\delta = \ep\]
            Therefore, $f(x)$ is uniformly continuous on $A$. $\qed$
        \color{black}

	\item Is the converse statement true? Are all uniformly continuous functions necessarily Lipschitz?
	
        \color{blue}
            No. We proceed by contradiction: if $f: A \to \R$ is uniformly continuous, then for all $\ep > 0$, there exists $\delta > 0$ such that $\forall x, y \in A$,
            \[\abs{x - y} < \delta \implies \abs{f(x) - f(y)} < \ep\] 

            In the case $\abs{x - y} < \delta$, certainly 
            \[\frac{\abs{f(x) - f(y)}}{\abs{x - y}} = \abs{\frac{f(x) - f(y)}{x - y}} < \frac{\ep}{\delta} := M\]

            However, when $\abs{x - y} \geq \delta$, the Lipschitz condition is not necessarily satisfied. Since the condition may not hold for all $x \neq y \in A$, we cannot conclude that all uniformly continuous functions are necessarily Lipschitz. $\qed$
        \color{black}

\end{enumerate}

\pagebreak
\section*{Problem 5}
Solve ONE of the following two:
\begin{enumerate}
	\item Finish the proof of the Intermediate Value Theorem using the Axiom of Completeness, started on page 138
	
        \color{blue}
            The IVT states: Let $f:[a,b]\to\mathbb{R}$ be continuous. If $L \in \R$ satisfying $f(a) < L < f(b)$ or $f(a) > L > f(b)$, then there exists $c \in (a, b)$ such that $f(c) = L$.

            Suppose $f: [a, b] \to \R$ is continuous and $f(a) < L < f(b)$. Let $K = \{x \in [a, b]: f(x) \leq L\}$. $K$ is bounded above by $b$ and $a \in K$ so $K$ is non-empty. By the Axiom of Completeness, $c = \sup K$ exists.

            There are now three cases:
            \begin{enumerate}
                \item $f(c) > L$: Let $\ep = f(c) - L > 0$> Since $f$ is continuous, $\exists \delta_0 > 0$ such that $\abs{x - c} < \delta_0 \implies \abs{f(x) - f(c)} < \ep$. 
                
                Therefore, we can choose a $\delta$ such that $0 < \delta < \delta_0$ so 
                \[\abs{(c - \delta) - c} < \delta_0 \implies \abs{f(c - \delta) - f(c)} < \ep\]
                
                But 
                \begin{align*}
                    \abs{f(c - \delta) - f(c)} &= \abs{-f(c - \delta) + f(c)} < \ep\\ 
                    &\implies -\ep < f(c) - f(c - \delta) < \ep\\ 
                    &\implies f(c) - f(c - \delta) < f(c) - L\\ 
                    &\implies -f(c - \delta) < -L\\ 
                    &\implies f(c - \delta) > L
                \end{align*}
                But then $c - \delta$ is an upper bound for $K$ less than $c$ which contradicts the fact that $c$ is the least upper bound. This case is impossible.

                \item $f(c) < L$: Again since $f$ is continuous, $f(x) \in V_{\ep}(f(c))$ for all $\ep > 0$. Therefore, $f(c + \delta) \in V_{\ep}(f(c))$ for some $\delta > 0$. We can choose $\ep$ such that $f(c + \delta) < L$ but then $c + \delta \in K$ which contradicts the fact that $c$ is an upper bound for $K$. This case is impossible. 
                
                \item $f(c) = L$. Clearly $b$ is an upper bound for $K$ and $L < f(b)$ so $a < c < b$ and we are done. $\qed$
            \end{enumerate}
        \color{black}

	\item Finish the proof of the Intermediate Value Theorem using the Nested Interval Property, started on page 138-139.
\end{enumerate}



\end{document}