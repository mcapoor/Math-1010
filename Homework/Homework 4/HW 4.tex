\documentclass[12pt]{article} 
\usepackage[utf8]{inputenc}
\usepackage{geometry}
\geometry{letterpaper}
\usepackage{graphicx} 
\usepackage{parskip}
\usepackage{booktabs}
\usepackage{array} 
\usepackage{paralist} 
\usepackage{verbatim}
\usepackage{subfig}
\usepackage{fancyhdr}
\usepackage{sectsty}
\usepackage[shortlabels]{enumitem}

\pagestyle{fancy}
\renewcommand{\headrulewidth}{0pt} 
\lhead{}\chead{}\rhead{}
\lfoot{}\cfoot{\thepage}\rfoot{}

\geometry{
    left=0.5in, 
    right=0.5in,
    top=0.75in,
    bottom=0.75in
}

%%% ToC (table of contents) APPEARANCE
\usepackage[nottoc,notlof,notlot]{tocbibind} 
\usepackage[titles,subfigure]{tocloft}
\renewcommand{\cftsecfont}{\rmfamily\mdseries\upshape}
\renewcommand{\cftsecpagefont}{\rmfamily\mdseries\upshape} %

\usepackage{amsmath}
\usepackage{amssymb}
\usepackage{mathtools}
\usepackage{empheq}
\usepackage{xcolor}

\usepackage{tikz}
\usepackage{pgfplots}
\pgfplotsset{compat=1.18}

\newcommand{\ans}[1]{\boxed{\text{#1}}}
\newcommand{\vecs}[1]{\langle #1\rangle}
\renewcommand{\hat}[1]{\widehat{#1}}
\newcommand{\F}[1]{\mathcal{F}(#1)}
\renewcommand{\P}{\mathbb{P}}
\newcommand{\R}{\mathbb{R}}
\newcommand{\E}{\mathbb{E}}
\newcommand{\Z}{\mathbb{Z}}
\newcommand{\N}{\mathbb{N}}
\newcommand{\Q}{\mathbb{Q}}
\newcommand{\ind}{\mathbbm{1}}
\newcommand{\qed}{\quad \blacksquare}
\newcommand{\brak}[1]{\left\langle #1 \right\rangle}
\newcommand{\bra}[1]{\left\langle #1 \right\vert}
\newcommand{\ket}[1]{\left\vert #1 \right\rangle}
\newcommand{\abs}[1]{\left\vert #1 \right\vert}
\newcommand{\mfX}{\mathfrak{X}}
\newcommand{\ep}{\varepsilon}

\usepackage{tcolorbox}
\tcbuselibrary{breakable, skins}
\tcbset{enhanced}
\newenvironment*{tbox}[2][gray]{
    \begin{tcolorbox}[
        parbox=false,
        colback=#1!5!white,
        colframe=#1!75!black,
        breakable,
        title={#2}
    ]}
    {\end{tcolorbox}}


\title{Math 1010 - Homework 4}
\author{Milan Capoor}
\date{}

\begin{document}
\maketitle
        
\section{Problem 1 (Calculating square roots)}
Let $x_1=2$ and define
\begin{equation*}
x_{n+1}=\frac{1}{2}\left(x_n+\frac{2}{x_n} \right).
\end{equation*}
\begin{enumerate}
	\item Show that $x_n^2$ is always greater than or equal to $2$, and then use this to prove that $x_n-x_{n+1}\geq0$. Conclude that $\lim x_n=\sqrt{2}$.
	
        \color{blue}
            First observe $x_1^2 = 4 > 2$. Then, 
            \begin{align*}
                x_{n+1}^2 &= \left[\frac{1}{2}\left(x_n + \frac{2}{x_n}\right)\right]^2\\ 
                    &= \frac{1}{4}\left(x_n + \frac{2}{x_n}\right)^2\\ 
                    &= \frac{1}{4}(x_n^2 + 4 + \frac{4}{x_n^2})
            \end{align*}

            Suppose $x_i^2 \geq 2$ for all $1 \leq i \leq n$. Then 
            \[x_{n+1}^2 \geq \frac{1}{4}((2)^2 + 4 + \frac{4}{(2)^2}) = \frac{9}{4} \geq 2\]
            
            Thus, by induction, $x_n^2 \geq 2$ for all $n \in \mathbb{N}$. 
            
            \vspace*{10pt}
            \hrule 
            \vspace*{10pt}

            Now consider 
            \begin{align*}
                x_n - x_{n+1} &= x_n - \frac{1}{2}\left(x_n + \frac{2}{x_n}\right)\\
                    &= x_n - \frac{1}{2}x_n - \frac{1}{x_n}\\
                    &= \frac{x_n}{2} - \frac{1}{x_n} = \frac{x_n^2 - 2}{2x_n} \geq \frac{2 - 2}{2x_n} = 0
            \end{align*}

            Thus, $x_n - x_{n+1} \geq 0$ for all $n \in \mathbb{N}$.

            \vspace*{10pt}
            \hrule 
            \vspace*{10pt}

            Since $x_n - x_{n+1} \geq 0$, the sequence $(x_n)$ is decreasing. Since $x_1 = 2$, the sequence is bounded above by $2$.

            Since the sequence is bounded and monotone, it is convergent. Let $\lim x_n = L$. Then
            \begin{align*}
                L &= \lim \frac{1}{2}\left(x_n + \frac{2}{x_n}\right)\\ 
                &= \frac{1}{2}\lim x_n + \lim \frac{1}{x_n}
            \end{align*}
            
            Define the sequence $y_n = 1$. Trivially, $(y_n) \to 1$. Then by the Algebraic Limit Theorem,
            \[\lim \frac{1}{x_n} = \lim \frac{y_n}{x_n} = \frac{1}{L}\]

            Then, substituting above 
            \[L = \frac{L}{2} + \frac{1}{L} = \frac{L^2 + 2}{2L} \implies 2L^2 = L^2 + 2 \implies L^2 = 2\]

            Finally note that while $(x_n)$ is decreasing, its terms are strictly positive and $x_1 = 2 > 0$ so $L = \lim x_n = \sqrt{2}$. $\qed$
        \color{black}

	\item Modify the sequence $(x_n)$ so that it converges to $\sqrt{c}$.
	
        \color{blue}
            Let $(x_n) \to L$ given by sequence given by
            \[x_{n+1} = \frac{1}{2}\left(x_n + \frac{c}{x_n}\right)\]

            Then, as above, 
            \begin{align*}
                \lim x_{n+1} &= \lim \frac{L}{2} + \lim \frac{c}{2x_n}\\ 
                L &= \frac{L}{2} + \frac{c}{2L} = \frac{L^2 + c}{2L}\\ 
                2L^2 &= L^2 + c\\
                L^2 &= c \implies \lim x_n = \sqrt{c} \qed
            \end{align*}
        \color{black}
\end{enumerate}

\pagebreak
\section{Problem 2 (Limit Superior)}
Let $(a_n)$ be a bounded sequence.
\begin{enumerate}
	\item Prove that the sequence defined by $y_n=\sup\{a_k:k\geq n \}$ converges. (You are allowed to use the fact that for two nonempty sets $A,B$ bounded above with $A\subset B$, we have $\sup A\leq \sup B$.)
    
        \color{blue}
            Let $\abs{a_n} \leq M$. Then $A = \{a_k: k \geq n\}$ is a nonempty set bounded above by $M$. Therefore, $\sup A$ exists. Now denote $B = \{a_k: k \geq 1\}$. Clearly, $\sup B \leq M$. But since $A \subset B$, we have $\sup A \leq \sup B \leq M$. Thus, $y_n = \sup A$ is a bounded sequence.

            Now notice 
            \[\{a_k : k \geq n+1\} \subset \{a_k: k \geq n\}\]
            so 
            \[y_n = \sup\{a_k: k \geq n\} \geq \sup\{a_k: k \geq n+1\} = y_{n+1}\]
            Therefore, $(y_n)$ is decreasing sequence.

            Since it is monotonic and bounded, it is convergent. $\qed$ 
        \color{black}

	\item The \textit{limit superior} of $(a_n)$, or $\limsup a_n$. is defined by
	\begin{equation*}
		\limsup a_n=\lim y_n,
	\end{equation*}
	where $y_n$ is the sequence from part 1). Provide a reasonable definition for $\liminf a_n$ and briefly explain why it always exists for any bounded sequence.

        \color{blue}
            Denote
            \[y_n = \inf a_n = \inf\{a_k: k \geq n\}\]
            and consider $\lim y_n = \liminf \{a_k: k \geq n\}$.

            As $\abs{a_n} \leq M$, $\{a_k: k \geq n\}$ is a nonempty set bounded below by $-M$. Therefore, $\inf\{a_k: k \geq n\}$ exists. Further, 
            \[\{a_k: k \geq n\} \subset \{a_k: k \geq 1\} \implies \inf \{a_k: k \geq n\} \geq \inf\{a_k: k \geq 1\}\]
            so $\inf\{a_k : k \geq n\} \geq \inf a_n \geq -M \implies y_n \geq -M$.

            Then, since $y_n$ is non-empty, $\inf\{a_k: k\geq n\} \leq \sup \{a_k: k \geq n\} \leq M$, so $\abs{y_n} \leq M$ and $y_n$ is bounded.

            Finally, 
            \[\{a_k: k \geq n+1\} \subset \{a_k: k \geq n\} \implies \inf\{a_k: k \geq n + 1\} \geq \inf\{a_k: k \geq n\} \implies y_n \leq y_{n+1} \implies y_n \text{ increasing}\]

            Since the sequence is bounded and monotonic, it is convergent and so $\lim y_n = \liminf a_n$ exists. $\qed$
        \color{black}

	\item Prove that $\liminf a_n\leq \limsup a_n$ for every bounded sequence, and give an example of a sequence for which the inequality is strict.
	
        \color{blue}
            Let $A_n = \{a_k: k \geq n\}$. By definition of bounds, 
            \[\inf A_n \leq A_n \leq \sup A_n \implies \inf a_n \leq \sup a_n\]

            From parts 1) and 2), we have that $\liminf a_n$ and $\limsup a_n$ exist since $(a_n)$ is bounded. Thus, we can take limits of the inequality to get
            \[\liminf a_n \leq \limsup a_n \qed\]
        \color{black}


	\item Show that $\liminf a_n=\limsup a_n$ if and only if $\lim a_n$ exists. In this case, all three share the same value. 
	
        \color{blue}
            Suppose $\lim a_n = a$ so $\exists N \in \N$ such that $\abs{a_n - a} < \ep$ for all $n \geq N$. Then, for all $n \geq N$, 
            \[\abs{a_n - a} = \abs{a_n - \limsup a_n + \limsup a_n - a} \leq \abs{a_n - \limsup a_n} + \abs{\limsup a_n - a} < \ep\]
            Since $\abs{a_n - \limsup a_n} > 0$, we have that $\abs{\limsup a_n - a} < \ep$ so $\limsup a_n = a$.
            
            Similarly, 
            \[\abs{a_n - a} = \abs{a_n - \liminf a_n + \liminf a_n - a} \leq \abs{a_n - \liminf a_n} + \abs{\liminf a_n - a} < \ep\]
            and $\abs{a_n - \liminf a_n} > 0 \implies \abs{\liminf a_n - a} < \ep$ so $\liminf a_n = a$.

            Therefore, $\liminf a_n = \limsup a_n = a$.

            \vspace*{10pt}
            \hrule 
            \vspace*{10pt}
            Now, suppose $\liminf a_n = \limsup a_n = a$. Then, for $\ep >0$, $\exists N \in \N$ such that for all $n > N$, 
            \[\abs{\,\inf\{a_k: k \geq n\} - a} < \ep, \qquad \abs{\,\sup\{a_k: k \geq n\} - a} < \ep\]

            We want to show that $\abs{a_n - a} < \ep$ for all $n > N$. Let $M > N$, then for all $m > M$, 
            \[\{a_k: k \geq n\} = a_m \implies \inf\{a_k: k\geq n\} = a_m = \sup \inf\{a_k: k\geq n\}\]
            so $\lim a_m = a$. $\qed$
        \color{black}
\end{enumerate}


\pagebreak

\section{Problem 3 }
Assume $(a_n)$ is a bounded sequence with the property that every convergent subsequence of $(a_n)$ converges to the same limit $a\in\mathbb{R}$. Show that $(a_n)$ must converge to $a$.

    \color{blue}
        Let $(a_{n_k}) \to a$ and $(a_{n_j}) \to a$ be two subsequences of $(a_n)$. Let $\ep > 0$. We want to show that $\exists N \in \N$ such that $\abs{a_n - a} < \ep$ for all $n \geq N$.

        Since $(a_{n_k})$ converges, we can say that $\exists K \in \N$ such that $\abs{a_{n_k} - a} < \frac{\ep}{2}$ for all $k \geq K$. Similarly, $\exists J \in \N$ such that $\abs{a_{n_j} - a} < \frac{\ep}{2}$ for all $j \geq J$. Let $N = \max{K, J}$. Pick $m \in \{n_k\}$ and $n \in \{n_j\}$ such that $m > n \geq N$.
        \[\abs{a_m - a_n} = \abs{a_m - a - a_n + a} \leq \abs{a_m - a} + \abs{a_n - a} \leq \frac{\ep}{2} + \frac{\ep}{2} < \ep\]
        Therefore, $(a_n)$ is Cauchy and so converges. 
        
        Since it is convergent and all subsequences converge to $a$, $(a_n) \to a$. $\qed$
    \color{black}

\pagebreak

\section{Problem 4}
Let $(a_n)$ be a bounded sequence, and define the set
\begin{equation*}
	S=\{x\in\mathbb{R}: x<a_n \text{ for infinitely many terms }a_n \}.
\end{equation*}
Show there exists a subsequence $(a_{n_k})$ converging to $s=\sup S$. This is a direct proof of the Bolzano-Weierstrass Theorem using the Axiom of Completeness.

    \color{blue}
        By the axiom of completeness, $s= \sup S$ exists and is the least upper bound for $S$. That is, for all $x < a_n$, $x \leq s < a_n$. 
        
        Let $\ep > 0$. Then $s + \ep \notin S$ so $s + \ep \geq a_n$ for infinitely many terms $a_n$. Therefore, we can create a subsequence $(a_{n_k})$ from the set $\{a_n \; | \; s < a_n < s + \ep\}$ with elements chosen such that $n_1 < n_2 < \cdots$.

        Now we want to show that $(a_{n_k}) \to s$. By the choice of $(a_{n_k})$, we have that for all $n_k \geq 1$,
        \[\abs{a_{n_k} - s} < \abs{(s + \ep) - s} = \abs{\ep} = \ep\]

        Therefore, $(a_{n_k}) \to s$. $\qed$
    \color{black}

\end{document}

\end{document}