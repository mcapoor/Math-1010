\documentclass[12pt]{article} 
\usepackage[utf8]{inputenc}
\usepackage{geometry}
\geometry{letterpaper}
\usepackage{graphicx} 
\usepackage{parskip}
\usepackage{booktabs}
\usepackage{array} 
\usepackage{paralist} 
\usepackage{verbatim}
\usepackage{subfig}
\usepackage{fancyhdr}
\usepackage{sectsty}
\usepackage[shortlabels]{enumitem}

\pagestyle{fancy}
\renewcommand{\headrulewidth}{0pt} 
\lhead{}\chead{}\rhead{}
\lfoot{}\cfoot{\thepage}\rfoot{}

\geometry{
    left=0.5in, 
    right=0.5in,
    top=0.5in,
    bottom=0.5in
}

%%% ToC (table of contents) APPEARANCE
\usepackage[nottoc,notlof,notlot]{tocbibind} 
\usepackage[titles,subfigure]{tocloft}
\renewcommand{\cftsecfont}{\rmfamily\mdseries\upshape}
\renewcommand{\cftsecpagefont}{\rmfamily\mdseries\upshape} %

\usepackage{amsmath}
\usepackage{amssymb}
\usepackage{mathtools}
\usepackage{empheq}
\usepackage{xcolor}

\usepackage{tikz}
\usepackage{pgfplots}
\pgfplotsset{compat=1.18}

\newcommand{\ans}[1]{\boxed{\text{#1}}}
\newcommand{\vecs}[1]{\langle #1\rangle}
\renewcommand{\hat}[1]{\widehat{#1}}
\newcommand{\F}[1]{\mathcal{F}(#1)}
\renewcommand{\P}{\mathbb{P}}
\newcommand{\R}{\mathbb{R}}
\newcommand{\E}{\mathbb{E}}
\newcommand{\Z}{\mathbb{Z}}
\newcommand{\N}{\mathbb{N}}
\newcommand{\Q}{\mathbb{Q}}
\newcommand{\ind}{\mathbbm{1}}
\newcommand{\qed}{\quad \blacksquare}
\newcommand{\brak}[1]{\left\langle #1 \right\rangle}
\newcommand{\bra}[1]{\left\langle #1 \right\vert}
\newcommand{\ket}[1]{\left\vert #1 \right\rangle}
\newcommand{\abs}[1]{\left\vert #1 \right\vert}
\newcommand{\mfX}{\mathfrak{X}}
\newcommand{\ep}{\varepsilon}

\usepackage{tcolorbox}
\tcbuselibrary{breakable, skins}
\tcbset{enhanced}
\newenvironment*{tbox}[2][gray]{
    \begin{tcolorbox}[
        parbox=false,
        colback=#1!5!white,
        colframe=#1!75!black,
        breakable,
        title={#2}
    ]}
    {\end{tcolorbox}}


\title{Math 1010: Homework 7}
\author{}
\date{}

\begin{document}
\maketitle
\vspace*{-1in}

\section*{Problem 1 }
For each stated limit, find the largest possible $\delta$-neighborhood that is a proper response to the given $\varepsilon$ challenge
\begin{enumerate}
	\item $\lim_{x\to 3}(5x-6)=9$, where $\varepsilon=1$
	
        \color{blue}
            We need a $\delta > 0$ such that the system of inequalities
            \[\begin{cases}
                \abs{x - 3} < \delta\\ 
                \abs{5x - 6 - 9} < 1
            \end{cases}\]
            is satisfied. From the second inequality, we have 
            \[\abs{5x - 15} < 1 \implies -1 < 5x - 15 < 1 \implies \frac{14}{5} < x < \frac{16}{5}\]

            Therefore, 
            \[-\frac{1}{5} < \abs{x - 3} < \frac{1}{5} \implies \boxed{\delta = \frac{1}{5}}\]
                
        \color{black}

	\item $\lim_{x\to 4}\sqrt{x}=2$, where $\varepsilon=1$. 
            
        \color{blue}
            \[\begin{cases}
                \abs{x - 4} < \delta\\ 
                \abs{\sqrt{x} - 2} < 1
            \end{cases}\]

            From the second inequality, we have
            \[-1 < \sqrt{x} - 2 < 1 \implies 1 < \sqrt{x} < 3 \implies 1 < x < 9\]
            so 
            \[(1 - 4) < x - 4 < (9 - 4) \implies -3 < x - 4 < 5 \implies \abs{x - 4} < 3\]
            so $\boxed{\delta = 3}$. 

        \color{black}

\end{enumerate}
Use the definition of functional limits to supply a proper proof for the following limit statements
\begin{enumerate}
	\item $\lim_{x\to 2}(x^2+x-1)=5$ 
	
        \color{blue}
            Let $\ep > 0$. We want to show that $\exists \delta > 0$, such that for $\abs{x - 2} < \delta$,
            \[\abs{(x^2 + x - 1) - 5} = \abs{x^2 + x - 6} = \abs{x - 2} \, \abs{x + 3} < \ep\]

            We construct a $\delta$-neighborhood around $c = 2$ with radius no bigger than $\delta = 1$ so 
            \[\abs{x + 3} \leq 6\]

            Choose $\delta = \min\{1, \frac{\ep}{6}\}$ so $\abs{x - 2} < \delta$ implies
            \[\abs{(x^2 + x - 1) - 5} \leq 6\abs{x - 2} < 6\delta = \ep \qed\]

        \color{black}

	\item $\lim_{x\to 0}x^3=0$
    
        \color{blue}
            Let $\ep > 0$. $\abs{x - 0} < \delta$ implies 
            \[\abs{x^3 - 0} < \ep\]
            if $\delta = \sqrt[3]{\ep}$:
            \[\abs{x} < \delta \implies \abs{x^3} = \abs{x}^3 < \delta^3 = \ep \qed\]

        \color{black}

	\item $\lim_{x\to 3}\frac{1}{x}=\frac{1}{3}$.
    
        \color{blue}
            Let $\ep > 0$. We want to show that $\exists \delta > 0$, such that for $\abs{x - 3} < \delta$, $\abs{\frac{1}{x} - \frac{1}{3}} < \ep$. 

            We have 
            \[\abs{\frac{1}{x} - \frac{1}{3}} = \abs{\frac{3 - x}{3x}} = \frac{\abs{x - 3}}{\abs{3x}}\]

            Construct a $\delta$-neighborhood of at most $2$ around $x = 3$ so $x > 1$ and 
            \[\frac{\abs{x-3}}{\abs{3x}} < \frac{\abs{x - 3}}{\abs{3}}\]

            Let $\delta = \min\{2, 3\ep\}$ so $\abs{x - 3} < \delta$ implies 
            \[\abs{\frac{1}{x} - \frac{1}{3}} < \frac{\abs{x-3}}{3} < \frac{\delta}{3} = \ep \qed\]
        \color{black}
\end{enumerate}


\pagebreak


\section*{Problem 2}
Are the following claims true or false and give a justification for each conclusion.
\begin{enumerate}
	\item If a particular $\delta$ has been constructed as a suitable response to a particular $\varepsilon$ challenge, then any smaller positive $\delta$ will also suffice
	
        \color{blue}
            True. Suppose $\lim_{x \to c} f(x) = L$, that is $\exists \delta > 0$ such that for $\abs{x - c} < \delta$, $\abs{f(x) - L} < \ep$ for all $\ep > 0$.
            
            If $\delta' < \delta$, then $\abs{x - c} < \delta'$ implies $\abs{x - c} < \delta$ so $\abs{f(x) - L} < \ep$. Therefore, any smaller $\delta$ will also suffice. $\qed$
        \color{black}

	\item If $\lim_{x\to a}f(x)=L$ and $a$ happens to be in the domain of $f$, then $L=f(a)$
    
        \color{blue}
            False. Consider 
            \[f(x) = \begin{cases}
                1 & \text{if } x \neq a\\
                0 & \text{if } x = a
            \end{cases}\]

            $\lim_{x \to a} f(x) = 1$ but $f(a) = 0$. $\qed$
        \color{black}

	\item If $\lim_{x\to a}f(x)=0$, then $\lim_{x\to a} f(x)g(x)=0$ for any function $g$ (with domain equal to the domain of $f$)
	
        \color{blue}
            False. For any continuous function $g$, the limit $\lim_{x \to a} f(x)g(x) = 0$ by the ALT for functional limits. However, if $g$ is not continuous at $a$, then the limit may not exist. For example, consider $g(x) = \frac{1}{x - a}$. Then $\lim_{x \to 0} f(x)g(x) = \frac{0}{0}$ which is indeterminate.
        \color{black}

	\item The limit $\lim_{x\to 2}\frac{\abs{x-2}}{x-2}$ exists (compute it if it does, or prove that it doesn't) 
	
        \color{blue}
            False. Notice that 
            \[f(x) = \frac{\abs{x-2}}{x - 2} = \begin{cases}
                1 & \text{if } x > 2\\
                -1 & \text{if } x < 2
            \end{cases}\]

            Consider the sequences $x_n = 2 + \frac{1}{n}$ and $y_n = 2 - \frac{1}{n}$. Clearly, $x_n \to 2$ and $y_n \to 2$. However, $\lim_{x_n \to 2} f(x_n) = 1$ and $\lim_{x_n \to 2} f(y_n) = -1$ so $f(x_n)$ and $f(y_n)$ do not converge to the same value. Therefore, by the Divergence Criterion, the limit does not exist. $\qed$
        \color{black}

	\item The limit $\lim_{x\to 7/4}\frac{\abs{x-2}}{x-2}$ exists (compute it if it does, or prove that it doesn't)
	
        \color{blue}
            True. Let $f(x) = \frac{\abs{x - 2}}{x - 2}$. 
            \[f(\frac{7}{4}) = \frac{\abs{\frac{7}{4} - 2}}{\frac{7}{4} - 2} = \frac{\abs{-1/4}}{-1/4} = -1\]
                        
            Let $\ep > 0$. Let $\delta = \frac{1}{4} - \ep$ so for $x \in V_{\delta}(\frac{7}{4})$, $x < 2$. Since $f(x) = -1$ for all $x < 2$, $f(\frac{7}{4}) \in V_{\ep}(-1)$. Therefore, $\lim_{x \to 7/4} f(x) = -1$. $\qed$            
        \color{black}

\end{enumerate}
\pagebreak

\section*{Problem 3 (Squeeze Theorem for functions)}
Let $f,g$ and $h$ satisfy $f(x)\leq g(x)\leq h(x)$ for all $x$ in some common domain $A$. If $\lim_{x\to c}f(x)=L$ and $\lim_{x\to c}h(x)=L$ at some limit point $c$, show that $\lim_{x\to c}g(x)=L$ as well.

    \color{blue}
        By the Sequential Criterion for Functional Limits, there exist a sequence $(x_n) \to c$ such that $f(x_n) \to L$ and $h(x_n) \to L$. Since $f(x) \leq g(x) \leq h(x)$, we have 
        \[f(x_n) \leq g(x_n) \leq h(x_n)\] 
        for all $n$. By the Squeeze Theorem for sequences, we have $g(x_n) \to L$ as well. Therefore, by the Sequential Criterion for Functional Limits, $\lim_{x\to c}g(x) = L$. $\qed$
    \color{black}

\pagebreak
\section*{Problem 4}
Let $g(x)=\sqrt[3]{x}$.
\begin{enumerate}
	\item Prove that $g$ is continuous at $c=0$. 
	
        \color{blue}
            Let $\ep > 0$. We want to show that $\exists \delta > 0$, such that for $\abs{x} < \delta$,
            \[\abs{g(x) - 0} = \abs{\sqrt[3]{x}} < \ep\]

            A natural choice is $\delta = \ep^3$ so $\abs{x} < \delta$ implies 
            \[\abs{g(x) - 0} = \abs{\sqrt[3]{x}} = \sqrt[3]{\abs{x}} < \sqrt[3]{\delta} = \sqrt[3]{\ep^3} = \ep \implies \abs{g(x) - 0} < \ep \qed\]
        \color{black}

	\item Prove that $g$ is continuous at a point $c\neq0$. (The identity $a^3-b^3=(a-b)(a^2+ab+b^2)$ will be helpful).
    
        \color{blue}
            As above, let $\ep > 0$. We want to show that $\abs{x - c} < \delta$ (for $\delta > 0$) implies 
            \[\abs{\sqrt[3]{x} - \sqrt[3]{c}} < \ep\]

            Notice that 
            \[\abs{\sqrt[3]{x} - \sqrt[3]{c}} = \abs{\frac{x - c}{\sqrt[3]{x^2} + \sqrt[3]{xc} + \sqrt[3]{c^2}}}\] 

            If $x$ and $c$ have the same sign, then $\abs{\sqrt[3]{x^2} + \sqrt[3]{xc} + \sqrt[3]{c^2}} = \sqrt[3]{x^2} + \sqrt[3]{xc} + \sqrt[3]{c^2} > \sqrt[3]{c^2}$ so 
            \[\abs{\sqrt[3]{x} - \sqrt[3]{c}} = \abs{\frac{x - c}{\sqrt[3]{x^2} + \sqrt[3]{xc} + \sqrt[3]{c^2}}} < \frac{\abs{x-c}}{\sqrt[3]{c^2}}\] 

            A natural approach is to construct a $\delta$-neighborhood around $c$ such that $x$ and $c$ have the same sign. 

            Let $\delta = \min\{\frac{\abs{c}}{2}, \ep \abs{c}^{2/3}\}$ so $\abs{x - c} < \delta$ implies
            \[\abs{\sqrt[3]{x} - \sqrt[3]{c}} < \frac{\abs{x- c}}{\sqrt[3]{c^2}} < \frac{\delta}{\sqrt[3]{c^2}} = \frac{\ep \sqrt[3]{c^2}}{\sqrt[3]{c^2}} = \ep \qed\]
                
        \color{black}

\end{enumerate}

\pagebreak
\section*{Problem 5}
\begin{enumerate}
	\item Supply a proof for Theorem 4.3.9 using the $\varepsilon-\delta$ characterization of continuity 
    
        \color{blue}
            Theorem 4.3.9 states that given $f: A \to \R$ and $g: B \to \R$ with $f(A) \subseteq B$, if the range $f(A) = \{f(x): x \in A\}$ is contained in the domain $B$ so $g \circ f(x) = g(f(x))$ is defined on $A$ and if $f$ is continuous at $c \in A$ and $g$ is continuous at $f(c) \in B$, then $g \circ f$ is continuous at $c$.

            Since $f$ is continuous at $c$, $\forall \ep > 0$, $\exists \delta_1 > 0$ such that for $\abs{x - c} < \delta_1$, $\abs{f(x) - f(c)} < \ep_1$. Similarly, since $g$ is continuous at $f(c)$, $\forall \ep > 0$, $\exists \delta_2 > 0$ such that for $\abs{y - f(c)} < \delta_2$, $\abs{g(y) - g(f(c))} < \ep$.

            We want to show that there exists a $\delta$ such that $\abs{x - c} < \delta$ implies that 
            \[\abs{g(f(x)) - g(f(c))} < \ep\]

            By continuity of $f$, 
            \[\abs{x - c} < \delta_1 \implies \abs{f(x) - f(c)} < \ep_1\]

            Let $\delta_2 = \ep_1$. Since $f(x)$ and $f(c)$ are in the domain of $g$, the continuity of $g$ gives that
            \[\abs{f(x) - f(c)} < \delta_2 \implies \abs{g(f(x)) - g(f(c))} < \ep\]

            Therefore, for $\delta = \delta_1$, 
            \[\abs{x - c} < \delta \implies \abs{g(f(x)) - g(f(c))} < \ep \qed\]
        \color{black}

	\item Give another proof of this theorem using the sequential characterization of continuity.
	
        \color{blue}
            Since $f$ is continuous at $c$, $\forall (x_n) \in A$ such that $(x_n) \to c$, $f(x_n) \to f(c)$. Similarly, by the continuity of $g$, $\forall (y_n) \to f(c)$, $g(y_n) \to g(f(c))$. 
            
            Therefore, $\forall (x_n) \to c$, $g(f(x_n)) \to g(f(c))$ so by the Sequential Criterion for Functional Limits, $g \circ f$ is continuous at $c$. $\qed$
        \color{black}
    \end{enumerate}

\end{document}

