\documentclass[12pt]{article} 
\usepackage[utf8]{inputenc}
\usepackage{geometry}
\usepackage{graphicx} 
\usepackage{parskip}
\usepackage{booktabs}
\usepackage{array} 
\usepackage{paralist} 
\usepackage{verbatim}
\usepackage{subfig}
\usepackage{fancyhdr}
\usepackage{sectsty}
\usepackage[shortlabels]{enumitem}

\pagestyle{fancy}
\renewcommand{\headrulewidth}{0pt} 
\lhead{}\chead{}\rhead{}
\lfoot{}\cfoot{\thepage}\rfoot{}

\geometry{
    letterpaper,
    left=0.25in, 
    right=0.25in,
    top=0.25in,
    bottom=0.25in
}


%%% ToC (table of contents) APPEARANCE
\usepackage[nottoc,notlof,notlot]{tocbibind} 
\usepackage[titles,subfigure]{tocloft}
\renewcommand{\cftsecfont}{\rmfamily\mdseries\upshape}
\renewcommand{\cftsecpagefont}{\rmfamily\mdseries\upshape} %

\usepackage{amsmath}
\usepackage{amssymb}
\usepackage{mathtools}
\usepackage{empheq}
\usepackage{xcolor}

\usepackage{tikz}
\usepackage{pgfplots}
\pgfplotsset{compat=1.18}

\newcommand{\ans}[1]{\boxed{\text{#1}}}
\newcommand{\vecs}[1]{\langle #1\rangle}
\renewcommand{\hat}[1]{\widehat{#1}}
\newcommand{\F}[1]{\mathcal{F}(#1)}
\renewcommand{\P}{\mathbb{P}}
\newcommand{\R}{\mathbb{R}}
\newcommand{\E}{\mathbb{E}}
\newcommand{\Z}{\mathbb{Z}}
\newcommand{\N}{\mathbb{N}}
\newcommand{\Q}{\mathbb{Q}}
\newcommand{\ind}{\mathbbm{1}}
\newcommand{\qed}{\quad \blacksquare}
\newcommand{\brak}[1]{\left\langle #1 \right\rangle}
\newcommand{\bra}[1]{\left\langle #1 \right\vert}
\newcommand{\ket}[1]{\left\vert #1 \right\rangle}
\newcommand{\abs}[1]{\left\vert #1 \right\vert}
\newcommand{\mfX}{\mathfrak{X}}
\newcommand{\ep}{\varepsilon}

\usepackage{tcolorbox}
\tcbuselibrary{breakable, skins}
\tcbset{enhanced}
\newenvironment*{tbox}[2][gray]{
    \begin{tcolorbox}[
        parbox=false,
        colback=#1!5!white,
        colframe=#1!75!black,
        breakable,
        title={#2}
    ]}
    {\end{tcolorbox}}


\title{Math 1010 - Homework 10}
\author{}
\date{}

\begin{document}
\maketitle
\vspace*{-1in} 


\section*{Problem 1 }
Let 
\[f_n(x)=\frac{nx}{1+nx^2}\]
\begin{enumerate}
	\item Find the pointwise limit of $(f_n)$ for all $x\in(0,\infty)$.

        \color{blue}
           \[\lim_{n \to \infty} f_n(x) = \lim_{n \to \infty} \frac{nx}{1 + nx^2} = \lim_{n \to \infty} \frac{x}{\frac{1}{n} + x^2} = \frac{x}{x^2} = \frac{1}{x}\]
        \color{black}

	\item Is the convergence uniform on $(0,\infty)$?
	
        \color{blue}
            Let $\ep > 0$. 
            \begin{align*}
                \abs{f_n(x) - f(x)} &= \abs{\frac{nx}{1+nx^2} - \frac{1}{x}}\\ 
                &= \abs{\frac{nx^2}{x(1 + nx^2)} - \frac{1 + nx^2}{x(1 + nx^2)}}\\
                &= \abs{-\frac{1}{x(1 + nx^2)}} = \frac{1}{x(1 + nx^2)}
            \end{align*}
            Therefore, any choice of $n$ to make $\abs{f_n(x) - f(x)}$ will depends on $x$ so the convergence is not uniform on $(0, \infty)$. $\qed$
        \color{black}

	\item Is the convergence uniform on $(0,1)$?
	
        \color{blue}
            Let $\ep = 1$. Then 
            \[\abs{f_n(x) - f(x)} = \frac{1}{x + nx^3}\]

            If we choose $x = \frac{1}{4}$ then 
            \[\abs{f_n(\frac{1}{4}) - f(\frac{1}{4})} = \frac{1}{1/4 + n(1/64)} = \frac{64}{16 + n}\]
            so we need $n > 48$ to make $\abs{f_n(\frac{1}{4}) - f(\frac{1}{4})} < 1$. 

            However, if we chose $x = \frac{1}{2}$, then 
            \[\abs{f_n(\frac{1}{2}) - f(\frac{1}{2})} = \frac{1}{1/2 + n/8} = \frac{8}{4 + n}\]
            so we just need $n > 4$ to make $\abs{f_n(\frac{1}{2}) - f(\frac{1}{2})} < 1$.

            Therefore, the convergence is not uniform on $(0,1)$. $\qed$
        \color{black}

	\item Is the convergence uniform on $(1,\infty)$?
	
        \color{blue}
            $x \geq 1$ implies that 
            \[\abs{f_n(x) - f(x)} = \frac{1}{x + nx^3} < \frac{1}{1 + n} < \frac{1}{n}\] 

            Therefore, to bound $\abs{f_n(x) - f(x)} < \ep$, we need $n > \frac{1}{\ep}$ which is independent of $x$. The convergence is uniform on $(1, \infty)$. $\qed$
        \color{black}
\end{enumerate}

\pagebreak


\section*{Problem 2 }
Using the Cauchy Criterion for convergent sequences of real numbers (Thm. 2.6.4 in the book), supply a proof for the Cauchy Criterion for Uniform Convergence (Thm. 6.2.5 in the book). First define a candidate for $f(x)$ and then argue that $f_n\to f$ uniformly.

    \color{blue}
        The Cauchy Criterion for uniform convergence says a sequence of functions $(f_n)$ defined on $A \subseteq \R$ converges uniformly on $A$ iff $\forall \ep > 0$, $\exists N \in \N$ such that $\abs{f_n(x) - f_m(x)} < \ep$ whenever $n, m \geq N$ and $x \in A$.

        The Cauchy Criterion for convergent sequences of real numbers says a sequence converges iff it is a Cauchy sequence. 
        
        Suppose $(f_n) \to f$ uniformly. Then $\exists N \in \N$ such that $\abs{f_n(x) - f(x)} < \frac{\ep}{2}$ whenever $n \geq N$ and $x \in A$. Choose $m > N$. Then 
        \[\abs{f_n(x) - f_m(x)} = \abs{f_n(x) - f_m(x) - f(x) + f(x)} \leq \abs{f_n(x) - f(x)} + \abs{f_m(x) - f(x)} < \frac{\ep}{2} + \frac{\ep}{2} = \ep\] 
        so $(f_n)$ is a Cauchy sequence.

        Suppose now that $\exists N \in \N$ such that $\abs{f_n(x) - f_m(x)} < \frac{\ep}{2}$ whenever $n, m \geq N$ and $x \in A$. Then, by the Cauchy Criterion for Convergent Sequences, $(f_n)$ converges to some limit $L$. 
        
        Define $f(x) = L$. then 
        \[\abs{f_n(x) - f(x)} = \abs{f_n(x) - f(x) + f_m(x) - f_m(x)} \leq \abs{f_n(x) - f_m(x)} + \abs{f_m(x) - f(x)}\]

        By assumption, $\abs{f_n(x) - f_m(x)} < \frac{\ep}{2}$. By pointwise convergence of $(f_n)$, if we choose $m$ sufficiently large, we have $\abs{f_m(x) - f(x)} = \abs{f_m(x) - L} < \frac{\ep}{2}$. Therefore, $\abs{f_n(x) - f(x)} < \ep$ for all $n \geq N$ and $x \in A$ so $(f_n)$ converges uniformly to $f$. $\qed$
    \color{black}


\pagebreak


\section*{Problem 3 (Arzela-Ascoli Theorem)}
A sequence of functions $(f_n)$ defined on a set $E\subset\mathbb{R}$ is called \textit{equicontinuous} if for every $\ep>0$ there exists a $\delta>0$ such that $|f_n(x)-f_n(y)|<\ep$ for all $n\in\mathbb{N}$ and $|x-y|<\delta$ in $E$. \\ \\
For each $n\in\mathbb{N}$, let $f_n$ be a function defined on $[0,1]$. If $(f_n)$ is bounded on $[0,1]$ -- that is, there exists an $M>0$ such that $|f_n(x)|\leq M$ for all $n\in\mathbb{N}$ and $x\in[0,1]$ -- and if the collection of functions $(f_n)$ is equicontinuous, follow these steps to show that $(f_n)$ contains a uniformly convergent subsequence.
\begin{enumerate}
	\item Explain why the construction in Exercise 6.2.13 (see below) produces a subsequence $(f_{n_k})$ that converges at every rational point in $[0,1]$. To simplify notation, set $g_k=f_{n_k}$. It remains to show that $(g_k)$ converges uniformly on all of $[0,1]$.
	
        \color{blue}
            We proved in class that $\Q$ is countable. Let $A = \{x \in [0, 1] \cap \Q\}$. Since $A \subseteq \Q$, $A$ is countable (not finite by density of $\Q$ in $\R$).

            Further, since $A \subseteq [0, 1]$, $(f_n)$ is defined on $A$ and is bounded on $A$ by assumption. By Exercise 6.2.13, there exists a subsequence $(f_{n_k})$ that converges pointwise on $A$.
        \color{black}
        
	\item Let $\ep>0$. By equicontinuity, there exists a $\delta>0$ such that $$|g_k(x)-g_k(y)|<\frac{\ep}{3}$$
	for all $|x-y|<\delta$ and $k\in\mathbb{N}$. Using this $\delta$, let $r_1,r_2,\cdots,r_m$ be a \textit{finite} collection of rational points with the property that the union of the neighborhoods $V_\delta(r_i)$ contains $[0,1]$. 
    
    Explain why there must exist an $N\in\mathbb{N}$ such that $$|g_s(r_i)-g_t(r_i)|<\frac{\ep}{3}$$ for all $s,t\geq N$ and $r_i$ in the finite subset of $[0,1]$ just described. Why does having the set $\{r_1,r_2,\cdots,r_m \}$ be finite matter?

        \color{blue}
            Since $r_i$ is a rational point in $[0, 1]$, we have that $(g_k)$ converges at $r_i$. Therefore, it is Cauchy so $\exists N_i \in \N$ such that $\abs{g_s(r_i) - g_t(r_i)} < \frac{\ep}{3}$ for all $s, t \geq N$.

            We want the sequence to converge for all $r_i$, so we take $N = \max\{N_1, N_2, \cdots, N_m\}$. However, taking a maximum of a set requires that the set be finite. The supremum would not suffice because our domain is limited to $\N$ (axiom of completeness). 
        \color{black}

	\item Finish the argument by showing that, for an arbitrary $x\in[0,1]$, 
    \[|g_s(x)-g_t(x)|<\ep\] 
     for all $s,t\geq N$.

        \color{blue}
            By the triangle inequality, 
            \begin{align*}
                \abs{g_s(x) - g_t(x)} &= \abs{g_s(x) - g_t(x) + g_s(r_i) - g_s(r_i) + g_t(r_i) - g_t(r_i)}\\ 
                &= \abs{g_s(x) - g_s(r_i)} + \abs{g_t(x) - g_t(r_i)} + \abs{g_s(r_i) - g_t(r_i)}\\
            \end{align*}
            for any $r_i$ in the finite subset of $[0, 1]$.

            By (2), for all $s, t \geq N$, we have $\abs{g_s(r_i) - g_t(r_i)} < \frac{\ep}{3}$. 

            By the definition of $\{r_i\}_{i=1}^m$, we have that any $x \in [0, 1]$ is in $V_\delta(r_i)$ for some $r_i$. Therefore, $\abs{x - r_i} < \delta$. By equicontinuity, $\abs{g_s(x) - g_s(r_i)} < \frac{\ep}{3}$ and $\abs{g_t(x) - g_t(r_i)} < \frac{\ep}{3}$.

            Thus, 
            \[\abs{g_s(x) - g_t(x)} < \frac{\ep}{3} + \frac{\ep}{3} + \frac{\ep}{3} = \ep \qed\]
        \color{black}
\end{enumerate}

\textbf{Result from Exercise 6.2.13}: Let $A=\{x_1,x_2,x_3,\cdots\}$ be a countable set. For each $n\in\mathbb{N}$, let $f_n$ be defined on $A$ and assume there exists an $M>0$ such that $|f_n(x)|\leq M$ for all $n\in\mathbb{N}$ and $x\in A$. Then there exists a subsequence of $(f_n)$ that converges pointwise on $A$. (This a version of the Bolzano--Weierstrass Thm. for bounded sequences of functions).

\pagebreak

\section*{Problem 4 }
Consider the sequence of functions defined by 
\begin{equation*}
	g_n(x)=\frac{x^n}{n}.
\end{equation*}
\begin{enumerate}
	\item Show $(g_n)$ converges uniformly on $[0,1]$ and find $g=\lim g_n$. Show that $g$ is differentiable and compute $g'(x)$ for all $x\in[0,1]$.
	
        \color{blue}
            Let $\ep > 0$. Since $x \leq 1$, $g_n(x) \leq \frac{1}{n}$. Therefore, $N > \frac{1}{\ep}$ will suffice to make $\abs{g_n(x) - 0} < \ep$ for all $n \geq N$ and $x \in [0, 1]$. Thus, $\lim g_n = 0$. 

            Clearly, $g$ is differentiable since $g(x) = 0$ for all $x \in [0, 1]$. Therefore, $g'(x) = 0$ for all $x \in [0, 1]$. $\qed$
        \color{black}

	\item Now, show that $(g'_n)$ converges on $[0,1]$. Is the convergence uniform? Set $h=\lim g_n'$ and compare $h$ and $g'$. Are they the same?

        \color{blue}
            \[\lim_{n \to \infty} g_n' = \lim_{n \to \infty} x^{n-1} = \begin{cases}
                0 & \text{if } 0 \leq x < 1\\
                1 & \text{if } x = 1
            \end{cases}\] 
            
            Notice 
            \begin{align*}
                \abs{g_n'(\frac{1}{2}) - h(\frac{1}{2})} < \frac{1}{3} \implies n \geq 3\\ 
                \abs{g_n'(\frac{9}{10}) - h(\frac{9}{10})} < \frac{1}{3} \implies n \geq 12\\
            \end{align*}
            so the convergence is not uniform.

            We have $g'(x) = 0$ for all $x \in [0, 1]$ and $h = \mathbf{1}_{x=1}(x)$ so $h \neq g'$. $\qed$
            
        \color{black}

\end{enumerate}


\pagebreak
\section*{Problem 5}
Use the Mean Value Theorem to supply a proof for Theorem 6.3.2 from the book. To get started, observe that the triangle inequality implies that, for any $x\in[a,b]$ and $m,n\mathbb{N}$,
\begin{equation*}
	|f_n(x)-f_m(x)|\leq|(f_n(x)-f_m(x))-(f_n(x_0)-f_m(x_0))|+|f_n(x_0)-f_m(x_0)|.
\end{equation*}

    \color{blue}
        \emph{Theorem 6.3.2:} Let $(f_n)$ be a sequence of differentiable functions defined on the closed interval $[a, b]$ and assume $(f_n')$ converges uniformly to $g$ on $[a, b]$. If there exists a point $x_0\in[a,b]$ such that $(f_n(x_0))$ converges, then $(f_n)$ converges uniformly. Moreover, $f = \lim f_n$ is differentiable and $f'(x) = g(x)$ for all $x\in[a,b]$.

        \emph{Proof:} Let $(f_n)$ be differentiable on $[a, b]$ and assume $(f_n') \to g$ uniformly on $[a, b]$. Suppose $\exists x_0 \in [a, b]$ so $(f_n(x_0))$ converges. 
        
        Let $\ep > 0$. We want to show that $(f_n)$ converges uniformly, i.e. $\exists K \in \N$ for which $\forall n, m \geq K$,
        \[\abs{f_n(x) - f_m(x)} < \ep\]

        Using the triangle inequality, 
        \begin{align*}
            \abs{f_n(x) - f_m(x)} &= \abs{f_n(x) - f_m(x) + f_n(x_0) - f_m(x_0) + f_n(x_0) - f_m(x_0)}\\
            &\leq \abs{(f_n(x) - f_m(x)) - (f_n(x_0) - f_m(x_0))} + \abs{f_n(x_0) - f_m(x_0)}
        \end{align*}

        By assumption, $\exists x_0 \in [a, b]$ so $\exists N \in \N$ for which $\forall n, m \geq N$, 
        \[\abs{f_n(x_0) - f_m(x_0)} < \frac{\ep}{2}\] 

        Now let $h(x) = f_n(x) - f_m(x)$. Since $f_n(x)$ is differentiable for all $x \in [a, b]$ and $n \in \N$, $h'(x) = f_n'(x) - f_m'(x)$ by the Algebraic Differentiability Theorem. 
        
        Since $(f_n') \to g$ uniformly, by the Cauchy Criterion, $\exists M \in \N$ such that $\forall m, n \geq M$, 
        \[\abs{f_n'(x) - f_m'(x)} < \ep_0\]
        for all $x \in [a, b]$. 

        Therefore, $\abs{h'(x)} < \ep_0$.  
        
        By the mean value theorem, $\exists c \in [x_0, x]$ such that
        \[h'(c) = \frac{h(x) - h(x_0)}{x - x_0} = \frac{(f_n(x) - f_m(x)) - (f_n(x_0) - f_m(x_0))}{x - x_0}\]

        Therefore, 
        \begin{align*}
            \abs{f_n(x) - f_m(x)} &\leq \abs{(f_n(x) - f_m(x)) - (f_n(x_0) - f_m(x_0))} + \abs{f_n(x_0) - f_m(x_0)}\\ 
            &= \abs{h'(c)(x - x_0)} + \abs{f_n(x_0) - f_m(x_0)}\\
        \end{align*}

        Using our earlier convergence bounds, $\forall n, m \geq \max\{N, M\}$,
        \[\abs{f_n(x) - f_m(x)} \leq \abs{\ep_0(x - x_0)} + \frac{\ep}{2}\]

        Since $x, x_0 \in [a, b]$, $\abs{x - x_0} \leq b - a$. If we let $\ep_0 = \frac{\ep}{2(b - a)}$, then
        \[\abs{f_n(x) - f_m(x)} \leq \frac{\ep}{2} + \frac{\ep}{2} = \ep\]
        so by the Cauchy Criterion, $(f_n)$ converges uniformly.

        Since $(f_n) \to f$ uniformly, it converges pointwise on $[a, b]$ and by assumption $(f_n') \to g$ uniformly. Therefore, by the Differentiable Limit Theorem $f = \lim f_n$ is differentiable and $f'(x) = g(x)$ for all $x\in[a,b]$. $\qed$

    \color{black}


\end{document}