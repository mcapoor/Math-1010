\documentclass[12pt]{article} 
\usepackage[utf8]{inputenc}
\usepackage{geometry}
\geometry{letterpaper, 
    left=0.5in,
    right=0.5in,
    top=0.5in,
    bottom=0.5in
}

\usepackage{graphicx} 
\usepackage{parskip}
\usepackage{booktabs}
\usepackage{array} 
\usepackage{paralist} 
\usepackage{verbatim}
\usepackage{subfig}
\usepackage{fancyhdr}
\usepackage{sectsty}
\usepackage[shortlabels]{enumitem}

\pagestyle{fancy}
\renewcommand{\headrulewidth}{0pt} 
\lhead{}\chead{}\rhead{}
\lfoot{}\cfoot{\thepage}\rfoot{}


%%% ToC (table of contents) APPEARANCE
\usepackage[nottoc,notlof,notlot]{tocbibind} 
\usepackage[titles,subfigure]{tocloft}
\renewcommand{\cftsecfont}{\rmfamily\mdseries\upshape}
\renewcommand{\cftsecpagefont}{\rmfamily\mdseries\upshape} %

\usepackage{amsmath}
\usepackage{amssymb}
\usepackage{mathtools}
\usepackage{empheq}
\usepackage{xcolor}

\usepackage{tikz}
\usepackage{pgfplots}
\pgfplotsset{compat=1.18}

\newcommand{\ans}[1]{\boxed{\text{#1}}}
\newcommand{\vecs}[1]{\langle #1\rangle}
\renewcommand{\hat}[1]{\widehat{#1}}
\newcommand{\F}[1]{\mathcal{F}(#1)}
\renewcommand{\P}{\mathbb{P}}
\newcommand{\R}{\mathbb{R}}
\newcommand{\E}{\mathbb{E}}
\newcommand{\Z}{\mathbb{Z}}
\newcommand{\Q}{\mathbb{Q}}
\newcommand{\N}{\mathbb{N}}
\newcommand{\C}{\mathbb{C}}
\newcommand{\I}{\mathbb{I}}

\newcommand{\ind}{\mathbbm{1}}
\newcommand{\qed}{\quad \blacksquare}
\newcommand{\brak}[1]{\left\langle #1 \right\rangle}
\newcommand{\bra}[1]{\left\langle #1 \right\vert}
\newcommand{\ket}[1]{\left\vert #1 \right\rangle}
\newcommand{\abs}[1]{\left\vert #1 \right\vert}
\newcommand{\mfX}{\mathfrak{X}}

\usepackage{tcolorbox}
\tcbuselibrary{breakable, skins}
\tcbset{enhanced}
\newenvironment*{tbox}[2][gray]{
    \begin{tcolorbox}[
        parbox=false,
        colback=#1!5!white,
        colframe=#1!75!black,
        breakable,
        title={#2}
    ]}
    {\end{tcolorbox}}


\title{Math 1010 - Homework 1}
\author{}
\date{}

\begin{document}
\maketitle
\vspace*{-1in}
To solve these problems, you may use any definition of theorem we have learned or proven in class, but
remember to state what you use!

\section*{Problem 1 (De Morgan's Laws)} 
    Let $A$ and $B$ be subsets of $R$.
    \begin{enumerate}
        \item If $x \in (A \cap B)^c$, explain why $x \in A^c \cup B^c$. This shows that $(A \cap B)^c \subset A^c \cup B^c$.
        
            \color{blue}
                If $x \in (A\cap B)^c$, then $x \notin (A \cap B)$. This can mean that either $x$ is in neither $A$ nor $B$, or it is not in one of the two. Hence, 
                \[(x \notin A) \lor (x \notin B) \implies (x \in A^c) \lor (x \in B^c) \implies x \in (A^c \cup B^c) \implies (A \cap B)^c \subset A^c \cup B^c \qed\]
            \color{black}

        \item Prove the reverse inclusion $(A \cap B)^c \supset A^c \cup B^c$ and conclude that $(A \cap B)^c = A^c \cup B^c.$
            \color{blue}
                \begin{align*}
                    x \in (A^c \cup B^c) &\implies (x \in A^c) \lor (x \in B^c) \\ 
                        &\implies (x \notin A) \lor (x \notin B)\\  
                        & \implies x \notin (A \cap B)\\ 
                        &\implies x \in (A \cap B)^c \\ 
                        &\implies (A^c \cup B^c) \subset (A \cap B)^c \qed
                \end{align*}
            \color{black}

        \item Show $(A \cup B)^c = A^c \cap B^c$ by demonstrating inclusion both ways.
            \color{blue}
                \begin{align*}
                    x \in (A \cup B)^c &\implies x \notin (A \cup B)\\ 
                        &\implies (x \notin A) \land (x \notin B)\\ 
                        &\implies (x \in A^c) \land (x \in B^c)\\ 
                        &\implies x \in (A^c \cap B^c)\\ 
                        &\implies (A \cup B)^c \subset (A^c \cap B^c)\\ 
                    x \in (A^c \cap B^c) &\implies (x \in A^c) \land (x \in B^c)\\ 
                        &\implies (x \notin A) \land (x \notin B)\\ 
                        &\implies x \notin (A \cup B)\\ 
                        &\implies x \in (A \cup B)^c\\ 
                        &\implies (A^c \cap B^c) \subset (A \cup B)^c
                \end{align*}
                \[\begin{array}{c}
                    (A \cup B)^c \subset (A^c \cap B^c)\\ 
                    (A^c \cap B^c) \subset (A \cup B)^c 
                \end{array} \implies (A \cup B)^c = A^c \cap B^c \qed\]
            \color{black}
    \end{enumerate}
    
\pagebreak 
\section*{Problem 2}
    Let $y_1 = 6$, and for each $n \in \N$ define $y_{n+1} = (2y_n - 6)/3.$
    \begin{enumerate}
        \item Use induction to prove that the sequence satisfies $y_n > -6$ for all $n \in \N.$
            
            \color{blue}
                \begin{enumerate}[(a)]
                    \item Base case: 
                    \[y_2 = \frac{2y_1 - 6}{3} = \frac{2(6) - 6}{3} = 2 > -6 \quad \checkmark\]

                    \item Inductive step: Assume $y_n > -6$. Then 
                    \[y_{n+1} = \frac{2y_n - 6}{3} > \frac{2(-6) - 6}{3} = -6 \qed\]

                \end{enumerate}
            \color{black}

        \item Use another induction argument to show the sequence $(y_1, y_2, y_3, \dots)$ is decreasing.
            \color{blue}
                \begin{enumerate}[(a)]
                    \item Base case: 
                        \[y_2 = \frac{2y_1 - 6}{3} = \frac{2(6) - 6}{3} = 2 < y_1 = 6 \quad \checkmark\]

                    \item Inductive step: Assume $y_{n+1} \leq y_n$. Then 
                    \[\frac{2y_{n+1} - 6}{3} \leq \frac{2y_n - 6}{3} \implies y_{n+2} \leq y_{n+1} \qed\]
                \end{enumerate}
            \color{black}

    \end{enumerate}

\pagebreak 

\section*{Problem 3}
    Let $A \subset \R$ be nonempty and bounded above, and let $c \in \R$. Define the set $cA = \{ca : a \in A\}$
    \begin{enumerate}
        \item If $c \geq 0$, show that $\sup(cA) = c \sup A$.
        
            \color{blue}
                We will proceed in two steps: (a) show that $c \sup A$ is an upper bound for $cA$, and (b) show that $c \sup A$ is the least upper bound for $cA$.
                \begin{enumerate}[(a)]
                    \item \emph{$c \sup A$ is an upper bound for $cA$:} Let $x \in cA$. Then $x = ca$ for some $a \in A$. Since $a \leq \sup A$,  $x = ca \leq c \sup A$. Thus, $c \sup A$ is an upper bound for $cA$.

                    \item \emph{$c \sup A$ is the least upper bound for $cA$:} Suppose that $b$ is an upper bound for $cA$ such that $b < c \sup A$. If $c \neq 0$, then $\frac{b}{c} < \sup A$. Definitionally, $\frac{b}{c}$ is not an upper bound for $A$, so there exists $a \in A$ such that $\frac{b}{c} < a$. Thus $b < ca$, and $b$ cannot be an upper bound for $cA$. This is a contradiction. 
                    
                    If $c = 0$, then $cA = \{0\}$, and $0$ is the least upper bound for $cA$. Further, $c \sup A = 0$ so $c \sup A$ is the least upper bound for $cA$.

                    Thus, $c \sup A$ is the least upper bound for $cA$.
                \end{enumerate}
                Since $c\sup A$ is the least upper bound for $cA$, $\sup(cA) = c \sup A$. $\qed$
            \color{black}

        \item Postulate a similar type of statement for $\sup(cA)$ for the case $c < 0$. Hint: this might have something to do with $\inf A$!
        
            \color{blue}
                \emph{Claim:} Let $A \subset \R$ be nonempty and bounded below, and let $c \in \R$. Define the set $cA = \{ca : a \in A\}$. If $c < 0$, then $\sup(cA) = c \inf A$.

                \emph{Proof:} Exactly analogous to the above, we first show that $c \inf A$ is an upper bound for $cA$, and then show that $c \inf A$ is the lowest upper bound for $cA$.

                Let $ca \in cA$ for some $a \in A$. Since $\inf A \leq a$ and $c < 0$, $ca \leq c \inf A$. Thus, $c \inf A$ is an upper bound for $cA$ so $\sup(cA) \leq c \inf A$ 

                Now let $b \in A$. By definition, 
                \[cb \leq \sup(cA) \implies b \geq \frac{\sup(cA)}{c} \quad \forall b \in A\] 
                thus, $\sup(cA)/c$ is a lower bound for $A$. Since $\inf A$ is the greatest lower bound for $A$, $\sup(cA)/c \leq \inf A$. Thus, $\sup(cA) \geq c \inf A$.

                Since we now have $c\inf A \geq \sup(cA)$ and $c \inf A \leq \sup(cA)$, we conclude that $\sup(cA) = c \inf A$. $\qed$
                
            \color{black}
    \end{enumerate}

    \textbf{Remark:} See Example 1.3.7 in the book.
    \pagebreak 

\section*{Problem 4}
    Prove that if $a$ is an upper bound for $A$, and if $a$ is also an element of $A$, then it must be true that $a = \sup A$.

    \color{blue}
        Suppose that $a \neq \sup A$. Since $a$ is an upper bound for $A$, $a$ cannot be less than $\sup A$. Thus, $a > \sup A$. However, since $a \in A$, there exists at least one element of $A$ which is greater than $\sup A$ so $\sup A$ is not an upper bound. This is a contradiction. Thus, $a = \sup A$. $\qed$ 
    \color{black}
\pagebreak 

\section*{Problem 5}
    Recall that $\I$ stands for the set of irrational numbers.
    \begin{enumerate}
        \item Show that if $a \in \Q$ and $t \in \I$, then $a + t \in I$ and $at \in I$ as long as $a \neq 0$
            \color{blue}
                \begin{enumerate}[(a)]
                    \item $a + t \in \I$: Suppose that $a + t \in \Q$. $\Q$ is a field so it is closed under addition. Thus, $(a + t) - a \in \Q \implies t \in \Q$, which is a contradiction. Thus, $a + t \in \I$. 
                    
                    \item $at \in \I$: Suppose that $at \in \Q$. $\Q$ is a field so it is closed under multiplication. Thus (if $a \neq 0$), $\frac{at}{a} = t \in \Q$, which is a contradiction. Thus, $at \in \I$. $\qed$
                \end{enumerate}
            \color{black}

        \item Prove that $\I$ is dense in $\R$ by considering the real numbers $a - \sqrt 2$ and $b - \sqrt 2$ 
        
            \color{blue}
                We want to show that for any $a, b \in \R$ with $a < b$, there exists an irrational number $t$ such that $a < t < b$. 

                Without loss of generality, consider the real numbers $a - \sqrt 2, b - \sqrt 2$ with $a, b \in \R$ and $a < b$.  

                Since $\Q$ is dense in $\R$, there exists a rational number $q$ such that 
                \[a - \sqrt 2 < q < b - \sqrt 2 \implies a < q + \sqrt 2 < b\]
                
                By a lemma proved in class, $\sqrt 2 \notin \Q$. Then by part 1, $q + \sqrt 2 \in \I$  
                
                Thus, for any $a, b \in \R$, $\exists t \in \I$ such that $a < t < b$. Therefore, $\I$ is dense in $\R$. $\qed$
            \color{black}
    \end{enumerate}
    Hint: you are allowed to use the theorem on the density of $\Q$ in $\R$ which we proved in class


\end{document}